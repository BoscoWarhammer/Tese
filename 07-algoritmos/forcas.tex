\begin{algorithm}
    \SetKwInOut{Input}{Entrada}\SetKwInOut{Output}{Saída}
    \Input{posições, velocidades, lista de contatos e perfil de velocidades do fluido}
    \Output{forças e torques atuantes nos corpos}
    \ForEach{corpo}{
        Aplicar gravidade\;
        \ForEach{elemento da lista de contatos}{
            Calcula força normal $\vec{N}$\;
            Calcula força ${F}^{d}$ no rolamento de corpos\;
            \eIf{$|{F}^{d}| < \mu |\vec{N}|$}{
                $\vec{F}^{at} += \vec{F}^{d}\hat{t}$\;
            }{
                $\vec{F}^{at} += \mu \textrm{sinal}(\vec{F}^{d}) N\hat{t}$\;
            }
            Calcula torque\;
        }
        \If{possui fluido}{
            Calcula força de Arquimedes\;
            Calcula força de arrasto\;
        }
    }
    \caption{Aqui são calculadas as resultantes das forças em cada corpo. A força $\vec{N}$ é a força normal, contribuição da força elástica $\vec{F}^{el}$ e força de amortecimento $\vec{F}^{am}$ (equações \ref{equ:forca_elastica} e \ref{equ:forca_amortecimento}), $F^{d}$ é a força de rolamento de um corpo sobre o outro, que deve ser comparado com a força de atrito estático máxima $\mu N$. Retirado e adaptado de \cite{Dissertacao}.}
    \label{alg:forcas}
\end{algorithm}
