\begin{algorithm}
    \SetKwInOut{Input}{Entrada}\SetKwInOut{Output}{Saída}
    \Input{configuração de dados inicial da simulação}
    \Output{resposta e medições de simulação ao longo do tempo}
    \While{não atingida a condição de parada da simulação}{
        \If{chegou a hora de listar os Vizinhos}{
            Determinar a lista de corpos Vizinhos\;
        }
        Preditor\;
        Detectar Contatos\;
        Cálculo de Forças\;
        Corretor\;
        \If{Possui Fluido}{
            Cálculo do Fluido\;
        }
    }
    \caption{Dadas as entradas do problema, como posições iniciais dos corpos, velocidades e acelerações, o algoritmo de Dinâmica Molecular monta uma lista de corpos que são vizinhos delimitados por uma certa região, então prediz a posição e a velocidade dos corpos no próximo instante de tempo, procura os contatos que foram formados com a predição, calcula as forças entre cada corpo em contato e inclui as forças externas, corrige as predições de velocidade e aceleração de cada corpo e calcula a dinâmica do fluido. Assim um passo de Dinâmica Molecular é construído. Retirado de \cite{Dissertacao}.}
    \label{alg:MD}
\end{algorithm}
