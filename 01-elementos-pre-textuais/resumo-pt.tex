% -----------------------------------------------------------------------------
% Resumo
% -----------------------------------------------------------------------------

\begin{resumo}
    A simulação de materiais granulares é estudada nas academias de todo o mundo, também utilizada em indústrias e empresas de engenharia. Para o entendimento e quantificação dos materiais granulares um Método de Elementos Discretos, ou \textit{Discrete Element Method} (DEM), simula o comportamento de materiais granulares.
    Muitos dos desafios de se compreender o comportamento de materiais granulares têm início no fenômeno de segregação de grãos secos. Classicamente, temos o efeito castanha do Pará - \textit{Brazil Nut Effect} (BNE) - que consiste em um material granular confinado contendo grãos de diferentes volumes e que, quando agitados, ocorre segregação, sendo que os grãos maiores separam-se dos grãos menores. Por muitos anos, acreditou-se que esta segregação ocorria devido a presença de paredes que confinam o material. Nesta tese mostramos que em sistemas com condição periódica de contorno também pode ocorrer o BNE.
    Estudamos também o transporte de sedimentos que ocorre na interação entre granulares e fluidos. Para simular o comportamento de materiais granulares imersos em um fluido, utilizamos uma técnica de Fluidodinâmica Computacional, ou \textit{Computational Fluid Dynamics} (CFD). Os sedimentos sólidos se movem em um campo de velocidades transportados pelo fluido. Três parâmetros adimensionais são necessários para descrever o comportamento do transporte: o número de Reynolds, que relaciona as forças inerciais com as forças viscosas, e consequentemente os efeitos de turbulência do fluido; o número de Shields, que está relacionado com a as forças de arraste e as forças inerciais do fluido; e finalmente, a razão de densidade entre o sólido e o fluido. É possível reproduzir os diferentes modos de transporte apenas mudando tais parâmetros adimensionais. Nesta tese, calculamos o tempo de saturação para os modos de transporte.

    \textbf{Palavras Chaves}: Materiais granulares. Simulação computacional. Método de Elemento Discreto (DEM). Fluidodinâmica Computacional (CFD). Efeito castanha do Pará (BNE). Transporte de sedimentos.

%    Síntese do trabalho em texto cursivo contendo um único parágrafo.
%    Para uma Tese de Doutorado o resumo deve conter, no máximo, 500 palavras.
%    Para uma Dissertação de Mestrado o resumo deve conter, no máximo, 250 palavras.
%    Para um Projeto de Qualificação o resumo deve conter, no máximo, 200 palavras.
%    O resumo é a apresentação clara, concisa e seletiva do trabalho.
%    No resumo deve-se incluir, preferencialmente, nesta ordem:brevíssima introdução ao assunto do trabalho de pesquisa (incluindo motivação e justificativa para a realização deste trabalho), o que será feito no trabalho (objetivos), como ele será desenvolvido (metodologia), quais são os principais resultados obtidos ou esperados e a conclusão (compare os resultados com os da literatura e destaque as principais contribuições científicas do trabalho.

%    \textbf{Palavras-chave}: Modelo Latex. Trabalho acadêmico monográfico. Normas ABNT. Outra palavra.
\end{resumo}

% -----------------------------------------------------------------------------
% Escolha de 3 a 6 palavras ou termos que descrevam bem o seu trabalho. As palavras-chaves são utilizadas para indexação.
% A letra inicial de cada palavra deve estar em maiúsculas. As palavras-chave são separadas por ponto.
% -----------------------------------------------------------------------------
