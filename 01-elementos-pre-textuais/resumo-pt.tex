% -----------------------------------------------------------------------------
% Resumo
% -----------------------------------------------------------------------------

\begin{resumo}
    A simulação de materiais granulares é estudada nas academias de todo o mundo, também aplicada em indústrias e empresas de engenharia. Para o entendimento e quantificação das proprieadades dos materiais granulares, o Método de Elementos Discretos, ou \textit{Discrete Element Method} (DEM), é uma das técnicas computacionais mais usada para simular o comportamento de materiais granulares.
    Muitos dos desafios de se compreender o comportamento de materiais granulares têm início no fenômeno de segregação de grãos secos. Classicamente, temos o efeito castanha do Pará - \textit{Brazil Nut Effect} (BNE) - que consiste em um material granular confinado contendo grãos de diferentes volumes e que, quando agitados, exibem segregação, sendo que os grãos maiores ascendem até a superfície. Por muitos anos, acreditou-se que esta segregação ocorria devido a presença de paredes que confinam o material. Na primeira parte desta tese mostramos que em sistemas com condição periódica de contorno também pode ocorrer o BNE. Propomos que o BNE se comporta com efeito ressonante, e diferenciamos os sistemas com paredes do com condição periódica de contorno usando a função de grandes desvios - \textit{Large-Deviation function} (LDF).
    Na segunda parte desta tese estudamos também o transporte de sedimentos, que ocorre na interação entre granulares e fluidos. Para simular o comportamento de materiais granulares imersos em um fluido, utilizamos uma técnica de Fluidodinâmica Computacional, ou \textit{Computational Fluid Dynamics} (CFD). Os sedimentos sólidos se movem em um campo de velocidades transportados pelo fluido. Três parâmetros adimensionais são necessários para descrever o comportamento do transporte: o número de Reynolds, que relaciona as forças inerciais com as forças viscosas, e consequentemente os efeitos de turbulência do fluido; o número de Shields, que está relacionado com a as forças de arraste e as forças inerciais do fluido; e finalmente, a razão de densidade entre o sólido e a fase fluida. É possível reproduzir os diferentes modos de transporte apenas mudando tais parâmetros adimensionais. Nesta tese, calculamos o tempo de saturação para o modo \textit{bedload} no regime viscoso, e também predizemos o tempo de saturação para este modo de transporte. Este estudo de transporte de sedimentos foi possível graças ao doutorado sanduíche realizado no PMMH-ESPCI com a bolsa CAPES No.88881.187077/2018-01.

    \textbf{Palavras Chaves}: Materiais granulares. Simulação computacional. Método de Elemento Discreto (DEM). Fluidodinâmica Computacional (CFD). Efeito castanha do Pará (BNE). Transporte de sedimentos.

\end{resumo}
