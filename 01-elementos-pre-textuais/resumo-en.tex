% -----------------------------------------------------------------------------
% Abstract
% -----------------------------------------------------------------------------

\begin{resumo}[Abstract]
    The simulation of granular materials is studied widely in many research centers around the world, and applied in industries and engineering companies. For the understanding and quantification of granular materials properties, Discrete Element Method (DEM) is used to simulate the behavior of granular materials.
    Many of the challenges to understanding the behavior of granular materials begin in the dry grain segregation phenomenon. Classically, we have the Brazil-Nut Effect (BNE) - which consists of a confined granular material containing grains of different sizes which, when agitated, displays segregation, with the larger grains rising up to surface. For many years, it was believed that this segregation occurred due to the presence of walls that confine the material. In this thesis we show that in systems with periodic boundary conditions (pbc), BNE can also occur. We also proposed that BNE exhibits resonance effect, and we differentiate systems with walls and pbc by using Large-Deviation function (LDF).
    We also studied sediment transport that occurs in the interaction between granules and fluids. To simulate the behavior of granular materials immersed in a fluid, we use a Computational Fluid Dynamics (CFD) technique. The solid sediments move in the velocity field transported by the fluid. Three dimensionless parameters are required to describe the transport behavior: the Reynolds number, which relates the inertial forces to the viscous forces, and consequently the fluid turbulence effects; the number of Shields, which is related to the drag forces and the inertial forces of the fluid; and finally, the density ratio between the solid and the fluid phases. It is possible to reproduce the different modes of transport only by changing such dimensionless parameters. In this thesis, we calculate and characterize the saturation time for bedload transport mode in the viscous regime, and we also predict the saturation length for this transport mode. This sediment transport was possible to studied thanks to the sandwich PhD. done in PMMH-ESPCI with CAPES grant No.88881.187077/2018-01.
%    Translation of the abstract into english, possibly adapting or slightly changing the text in order to adjust it to the grammar of Standard English.
%    Try to stay within the limit of: 500 word for a PhD Thesis;
%    250 words for a Master Dissertation;
%    200 words for a Qualifying Research Project.

    \textbf{Keywords}: Granular materials. Computer simulations. Discrete Element Method (DEM). Computational Fluid Dynamics (CFD). Brazil-Nut Effect. Sediment transport.
\end{resumo}

% -----------------------------------------------------------------------------
% O restante da formatação deve manter-se igual ao do resumo em português, i.e, um único parágrafo.
% -----------------------------------------------------------------------------
