% -----------------------------------------------------------------------------
% Resumo
% -----------------------------------------------------------------------------

\begin{resumo}[Résumé]
    Des simulation de matériaux granulaires ont étudiée en centres de recherche de tout le monde, également elles ont appliquée dans les industries et des société d'ingénierie. Pour comprenez et qualifiez des propriétés des matériaux granulaire, la Méthode de Éléments Discrètes, ou \textit{Discrete Element Method} (DEM), est utilisée pour simuler des comportement des matériaux granulaires.
    De nombreux défis pour comprenez le comportement des matériaux granulaires commencement par le phénomène de ségrégation de grains secs. Classiquement, il y a l'effet noix du Brésil - \textit{Brazil Nut Effect} (BNE) - qui consiste en des matériaux granulaires confinés contenant grains de différents volumes et qui, lorsqu'ils son agités, présentent une ségrégation, le plus gros grains remontent jusqu'à la surface. Pendant de nombreuses années, on a cru que cette ségrégation était due à la présence de murs qui confinement le matériau. Dans cette thèse, nous montrons que dans le systèmes avec de condition aux limites périodiques, le BNE peut également se produire. Nous avons également proposé que le BNE présente un effet de résonance, et nous différencions les systèmes avec murs et de condition aux limites périodiques en utilisant la fonction de grande déviation.
    Nous avons également étudié le transport des sédiments qui se produit dans l'interaction entre le granules et les fluides. Pour simuler le comportement de matériaux granulaires immergés dans un fluide, nous utilisons une technique de Dynamique des Fluides Computationnelle, ou \textit{Computational Fluid Dynamics (CFD)}. Les sédiments solides se déplacent dans un champ de vitesses portées par le fluide. Trois paramètres adimensionnels sont nécessaires pour décrie le comportement de transport: le nombre de Reynolds, qui est lié aux forces d'inertie aux forces visqueuses, et par conséquent aux effets de la turbulence des fluides; le nombre de Shields, qui est lié aux forces de traînée et aux forces d'inertie du fluide; et enfin, le rapport de densité entre le solide et la phase fluide. Il est possible de reproduire les différents modes de transport simplement en modifiant ces paramètres adimensionnelles. Dans cette thèse, nous calculons et caractérisons le temps de saturation pour le modes de transport charrié en régime visquex, et nous prédisons également la longueur de saturation pour ce mode de transport. Ce transport sédimentaire a pu être étudié grâce à le stage de thèse fait au PMMH-ESPCI avec la bourse CAPES 88881.187077/2018-01.

    \textbf{Mots clés}: Matériaux granulaires. Simulation par ordinateur. Méthode des éléments discrets (DEM). Dynamique des fluides computationnelle (CFD). Effet de Noix du Brésil (BNE). Transporte de sédiments.

%    Síntese do trabalho em texto cursivo contendo um único parágrafo.
%    Para uma Tese de Doutorado o resumo deve conter, no máximo, 500 palavras.
%    Para uma Dissertação de Mestrado o resumo deve conter, no máximo, 250 palavras.
%    Para um Projeto de Qualificação o resumo deve conter, no máximo, 200 palavras.
%    O resumo é a apresentação clara, concisa e seletiva do trabalho.
%    No resumo deve-se incluir, preferencialmente, nesta ordem:brevíssima introdução ao assunto do trabalho de pesquisa (incluindo motivação e justificativa para a realização deste trabalho), o que será feito no trabalho (objetivos), como ele será desenvolvido (metodologia), quais são os principais resultados obtidos ou esperados e a conclusão (compare os resultados com os da literatura e destaque as principais contribuições científicas do trabalho.

%    \textbf{Palavras-chave}: Modelo Latex. Trabalho acadêmico monográfico. Normas ABNT. Outra palavra.
\end{resumo}

% -----------------------------------------------------------------------------
% Escolha de 3 a 6 palavras ou termos que descrevam bem o seu trabalho. As palavras-chaves são utilizadas para indexação.
% A letra inicial de cada palavra deve estar em maiúsculas. As palavras-chave são separadas por ponto.
% -----------------------------------------------------------------------------
