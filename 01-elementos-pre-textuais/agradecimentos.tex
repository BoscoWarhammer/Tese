% -----------------------------------------------------------------------------
% Agradecimentos
% -----------------------------------------------------------------------------

\begin{agradecimentos}[Acknowledgements]

First of all, I am greatfull to the creator of the universe, that gaves us inteligence and mysteries, which through science and philosophy will be unveiled.

To my family, I am thankful that they always given me all the encouragement and support necessary to carry out all my human and academic training. To my beloved wife, that supported me in all ways to keep searching and understanding what are the misteries of nature. To my father, I thank who always encouraged me to study scientific. To my mother, I thank, taste and fulfillment for teaching. To my brother I thank all the practice of patience and perseverance, in addition to keeping me in the realization of good practices and the usefulness of our work.

To my friend and adviser Allbens Atman, I thank the time and the efforts to teach me the path of the research and to all lessons inside and outside class, that I take with me.

To my friend and co-adviser Philippe Claudin, I thank for accepting me as his student, besides all disponibility and pacience that leads to this research.

I thank to the support given from the Graduate Program in Mathematical and Computing Modelling - PPGMMC at \textit{Centro Federal de Educação Tecnológica de Minas Gerais} - CEFET-MG, and from the laboratory \textit{Physique et Mécanique des Milieux Hétérogènes} - PMMH at \textit{École Supérieure de Physique et de Chimie Industrielles de la ville de Paris} - ESPCI.

I am also thankful to all my teachers and professors, once without them, I would certainly not be here.

To my friends, I thank their compreension, whom give me support in this project.

And specially, I am greatful to every brazilian citzen, that allow me to do this research in this public institution payed with taxes. I am also greatful to the finacial support given by \textit{Conselho Nacional de Pesquisa} - CNPq grant 88881.187077/2018-01, \textit{Coordenação de aperfeiçoamento de Pessoal de Nível Superior} - CAPES, \textit{Fundação de Amparo à Pesquisa do Estado de Minas Gerais} - FAPEMIG, PPGMMC, CEFET-MG, PMMH, \textit{Centre National de la Recherce Scientifique} - CNRS, and \textit{Fundação CEFETMINAS}.

%É obrigatório o agradecimento às instituições de fomento à pesquisa que financiaram total ou parcialmente o trabalho, inclusive no que diz respeito à concessão de bolsas.

\end{agradecimentos}

\begin{agradecimentos}[Agradecimentos]

Agradeço primeiramente ao grande criador de todo o universo, que nos deu inteligência e os mistérios, que pelas ciências e filosofia os serão desvendados.

À minha família, que sempre deu todo o incentivo e suporte necessários para realizar toda a minha formação humana e acadêmica. À minha grande companheira que sempre me deu apoio e incentivo, direto e indireto, a sempre continuar desvendando os mistérios naturais. Ao meu pai que sempre me incentivou o estudo científico. A minha mãe o gosto e a realização pela docência. Ao meu irmão toda a prática de paciência e perseverança, além de me manter na realização das boas práticas e da utilidade de nossos trabalhos.

Ao meu amigo e orientador Allbens Atman, por todo o tempo e esforço em ensinar os caminhos de como fazer pesquisa e por todas as lições dentro e fora da sala de aula que levo comigo.

Ao meu amigo e co-orientador Philippe Claudin, por ter me aceitado como seu aluno, além de toda a disponibilidade e paciência para que pudessemos chegar neste estágio da pesquisa.

Ao Centro Federal de Educação Tecnológico de Minas Gerais - CEFET-MG, ao Programa de Pós-Graduação em Modelagem Matemática e Computacional - PPGMMC e ao laboratório de \textit{Physique et Mécanique des Milieux Hétérogènes} - PMMH da \textit{École Supérieure de Physique et de Chimie Industrielles de la ville de Paris} - ESPCI que disponibilizaram seus recursos, pessoal e apoio nesta pesquisa.

À todos os meus professores, que certamente contribuíram para a minha chegada até este ponto.

Aos meus amigos e colegas que sempre compreenderam as ausências, e que me deram a ajuda e o suporte para continuar trabalhando neste projeto, sejam como ouvidos para os desabafos, sejam para renovar as ideias da pesquisa.


%É obrigatório o agradecimento às instituições de fomento à pesquisa que financiaram total ou parcialmente o trabalho, inclusive no que diz respeito à concessão de bolsas.

À cada cidadão brasileiro, que me permitiu pesquisar em uma escola paga com os impostos do povo, através de suas agências de fomento à educação e à pesquisa. Ao Conselho Nacional de Pesquisa - CNPq, especialmente na chamada de número 88881.187077/2018-01, à Coordenação de Aperfeiçoamento de Pessoal de Nível Superior - CAPES, à Fundação de Amparo à Pesquisa do Estado de Minas Gerais - FAPEMIG, ao CEFET-MG, ao PMMH e à ESPCI, ao \textit{Centre National de la Recherche Scientifique} - CNRS e à Fundação CEFETMINAS pelo suporte financeiro neste projeto de pesquisa, que teve início desde minha primeira iniciação científica em 2008.

\end{agradecimentos}
