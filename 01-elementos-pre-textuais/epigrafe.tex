% -----------------------------------------------------------------------------
% Epígrafe
% -----------------------------------------------------------------------------

\begin{epigrafe}

%\textit{“Saber que sabemos o que sabemos, e saber que não sabemos o que não sabemos, esta é a verdadeira sabedoria.”}
%(Nicolau Copérnico)

%\textit{“São grandes as vantagens industriais derivadas do princípio econômico da divisão do trabalho, porém, por causa disso, privou-se o trabalho do homem de alma e de vida.”}
%(Johannes Kepler)

\textit{“All truths are easy to understand once they are discovered; the point is discover them.”}

\textit{“Todas as verdades são fáceis de perceber depois de terem sido descobertas; o problema é descobri-las.”}

\textit{"Il est facile comprendre toutes les vérités une fois qu'elles sont découvertes ; le point est de les découvrir."}

(Galileu Galilei)

%\textit{“A gravidade explica os movimentos dos planetas, mas não pode explicar quem colocou os planetas em movimento. Deus governa todas as coisas e sabe tudo que é ou que pode ser feito.”}
%(Isaac Newton)

%\textit{“Ninguém é tão sábio que não tenha algo pra aprender e nem tão tolo que não tenha algo pra ensinar.”}
%(Blaise Pascal)

%\textit{“Devemos admitir com humildade que, ao passo que os números são puramente produtos de nossas mentes, o espaço tem uma realidade fora de nossas mentes, de modo que não podemos descrever completamente suas propriedades a priori.”}
%(Carl Friedrich Gauss)

%\textit{“Leia Euler, leia Euler, ele é o mestre de todos nós.”}
%(Pierre Simon Laplace)

\end{epigrafe}

% -----------------------------------------------------------------------------
% Edite o texto acima para inserir uma epígrafe de sua preferência
% -----------------------------------------------------------------------------
