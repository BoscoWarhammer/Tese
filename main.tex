% -----------------------------------------------------------------------------
%       Centro Federal de Educação Tecnológica de Minas Gerais - CEFET-MG
%
%   Modelo de trabalho acadêmico monográfico de acordo com as normas da ABNT
%   (Tese de Doutorado, Dissertação de Mestrado ou Projeto de Qualificação)
%
%     Projeto hospedado em: https://github.com/cfgnunes/latex-cefetmg
%
%    Autores: Cristiano Fraga G. Nunes <cfgnunes@gmail.com>
%             Henrique E. Borges <henrique@lsi.cefetmg.br>
%             Denise de Souza <densouza@gmail.com>
%             Lauro César <https://code.google.com/p/abntex2/>
%
% -----------------------------------------------------------------------------

\documentclass[%
%    twoside,                    % Impressão em frente (anverso) e verso
    oneside,                    % Impressão apenas no anverso
]{cefetmg}

\usepackage[%
    num,
%    overcite,
    abnt-emphasize=bf,
    bibjustif,
    recuo=0cm,
    abnt-doi=expand,            % Expande um endereço iniciado com doi: para http://dx.doi.org/
    abnt-url-package=url,       % Utiliza o pacote url
    abnt-refinfo=yes,           % Utiliza o estilo bibliográfico abnt-refinfo
    abnt-etal-cite=3,
    abnt-etal-list=3,
    abnt-thesis-year=final
]{abntex2cite}                  % Configura as citações bibliográficas conforme a norma ABNT

% -----------------------------------------------------------------------------
% Pacotes utilizados
% -----------------------------------------------------------------------------
\usepackage[utf8]{inputenc}                                 % Codificação do documento
\usepackage[T1]{fontenc}                                    % Seleção de código de fonte
\usepackage{booktabs}                                       % Réguas horizontais em tabelas
\usepackage{color, colortbl}                                % Controle das cores
\usepackage{float}                                          % Necessário para tabelas/figuras em ambiente multi-colunas
\usepackage{graphicx}                                       % Inclusão de gráficos e figuras
\usepackage{icomma}                                         % Uso de vírgulas em expressões matemáticas
\usepackage{indentfirst}                                    % Indenta o primeiro parágrafo de cada seção
\usepackage{microtype}                                      % Melhora a justificação do documento
\usepackage{multirow, array}                                % Permite tabelas com múltiplas linhas e colunas
\usepackage{subeqnarray}                                    % Permite subnumeração de equações
\usepackage{verbatim}                                       % Permite apresentar texto tal como escrito no documento, ainda que sejam comandos Latex
\usepackage{amsfonts, amssymb, amsmath}                     % Fontes e símbolos matemáticos
\usepackage[algoruled, english]{algorithm2e}                % Permite escrever algoritmos em inglês
\usepackage[scaled]{helvet}                                 % Usa a fonte Helvetica
%\usepackage{times}                                         % Usa a fonte Times
%\usepackage{palatino}                                      % Usa a fonte Palatino
%\usepackage{lmodern}                                       % Usa a fonte Latin Modern
%\usepackage[bottom]{footmisc}                              % Mantém as notas de rodapé sempre na mesma posição
%\usepackage{ae, aecompl}                                   % Fontes de alta qualidade
%\usepackage{latexsym}                                      % Símbolos matemáticos
%\usepackage{lscape}                                        % Permite páginas em modo "paisagem"
%\usepackage{picinpar}                                      % Dispor imagens em parágrafos
%\usepackage{scalefnt}                                      % Permite redimensionar tamanho da fonte
%\usepackage{subfig}                                        % Posicionamento de figuras
%\usepackage{upgreek}                                       % Fonte letras gregas
\usepackage{subcaption}                                     % Permite a utilização de sublegendas em finguras compostas
\usepackage{empheq}                                         % Permite a utilização de subequações
\usepackage{tikz}                                           % Criação e manipulação de figuras
\usepackage{tikzscale}                                      % Inclui as figuras feitas no TIKZ
\usepackage{pdfpages}                                       % Permite a inclusão de arquivos no texto

% Redefine o estilo de citação para o uso dos colchetes
\citebrackets[]

% Redefine a fonte para uma fonte similar a Arial (fonte Helvetica)
\renewcommand*\familydefault{\sfdefault}

% Redefine a cor dos números dos algoritmos para preto
% \SetNlSty{bfseries}{\color{black}}{}

% -----------------------------------------------------------------------------
% Configurações de aparência do PDF final
% -----------------------------------------------------------------------------
\makeatletter
\hypersetup{%
    english,
    colorlinks=true,            % true: "links" coloridos; false: "links" em caixas de texto
    linkcolor=black,            % Define cor dos "links" internos
    citecolor=black,            % Define cor dos "links" para as referências bibliográficas
    filecolor=black,            % Define cor dos "links" para arquivos
    urlcolor=black,             % Define a cor dos "hiperlinks"
    breaklinks=true,
    pdftitle={\@title},
    pdfauthor={\@author},
    pdfkeywords={Granular Materials, DEM, CFD, BNE}
}
\makeatother

% Altera o aspecto da cor azul
\definecolor{blue}{RGB}{41,5,195}

% Redefinição de labels
\renewcommand{\algorithmautorefname}{Algorithm}
\def\equationautorefname~#1\null{Equation~(#1)\null}

% Cria o índice remissivo
\makeindex

% Hifenização de palavras que não estão no dicionário
\hyphenation{%
    qua-dros-cha-ve
}

% -----------------------------------------------------------------------------
% Inclui os arquivos do trabalho acadêmico
% -----------------------------------------------------------------------------

% Insere e constrói alguns elementos pré-textuais para gerar capa, folha de rosto e folha de aprovação
% -----------------------------------------------------------------------------
% Capa
% -----------------------------------------------------------------------------

% -----------------------------------------------------------------------------
% ATENÇÃO:
% Caso algum campo não se aplique ao seu documento - por exemplo, em seu trabalho
% não houve coorientador - não comente o campo, apenas deixe vazio, assim: \campo{}
% -----------------------------------------------------------------------------

% -----------------------------------------------------------------------------
% Dados do trabalho acadêmico
% -----------------------------------------------------------------------------

\titulo{Numerical simulation of granular materials}
%\title{Title in English}
\subtitulo{Brazil nut effect and sediment transport}
\autor{Gustavo Henrique Borges Martins}
\local{Belo Horizonte}
\data{2021 July} % Normalmente se usa apenas mês e ano

% -----------------------------------------------------------------------------
% Natureza do trabalho acadêmico
% Use apenas uma das opções: Tese (p/ Doutorado), Dissertação (p/ Mestrado) ou
% Projeto de Qualificação (p/ Mestrado ou Doutorado), Trabalho de Conclusão de
% Curso (Graduação)
% -----------------------------------------------------------------------------

\projeto{Thesis}

% -----------------------------------------------------------------------------
% Título acadêmico
% Use apenas uma das opções:
% - Se a natureza for Tese, coloque Doutor
% - Se a natureza for Dissertação, coloque Mestre
% - Se a natureza for Projeto de Qualificação, coloque Mestre ou Doutor conforme o caso
% - Se a natureza for Trabalho de Conclusão de Curso, coloque Bacharel
% -----------------------------------------------------------------------------

\tituloAcademico{Doctor of Philosophy}

% -----------------------------------------------------------------------------
% Área de concentração e linha de pesquisa
% OBS: indique o nome da área de concentração e da linha de pesquisa do Programa de Pós-graduação
% nas quais este trabalho se insere
% Se a natureza for Trabalho de Conclusão de Curso, deixe ambos os campos vazios
% -----------------------------------------------------------------------------

\areaconcentracao{Mathematical and Computational Modeling}
\linhapesquisa{Applied Mathematical Methods}

% -----------------------------------------------------------------------------
% Dados da instituição
% OBS: a logomarca da instituição deve ser colocada na mesma pasta que foi colocada o documento
% principal com o nome de "logoInstituicao". O formato pode ser: pdf, jpf, eps
% Se a natureza for Trabalho de Conclusão de Curso, coloque em "programa' o nome do curso de graduação
% -----------------------------------------------------------------------------

\instituicao{Centro Federal de Educação Tecnológica de Minas Gerais}
\programa{Programa de Pós-graduação em Modelagem Matemática e Computacional}
\logoinstituicao{0.2}{./04-figuras/logo-instituicao.pdf} % \logoinstituicao{<escala>}{<nome do arquivo>}

% -----------------------------------------------------------------------------
% Dados do(s) orientador(es)
% -----------------------------------------------------------------------------

\orientador[Advisor:]{Allbens Atman Picardi Faria}
%\orientador[Orientadora:]{Nome da orientadora}
\instOrientador{CEFET-MG}

\coorientador[Co-Advisor:]{Philippe Claudin}
%\coorientador[Coorientadora:]{Nome da coorientadora}
\instCoorientador{ESPCI - France}

\include{./01-elementos-pre-textuais/folha-rosto}
% -----------------------------------------------------------------------------
% Folha de Aprovação
% -----------------------------------------------------------------------------

\textopadraofolhadeaprovacao{This sheet should be replaced by the scanned copy of the approval sheet provided.}

% -----------------------------------------------------------------------------
% Este documento foi mantido apenas para preservar a paginação do trabalho
% acadêmico final, após a inserção da folha de aprovação fornecida
% -----------------------------------------------------------------------------


\begin{document}

% Insere os elementos pré-textuais
\pretextual
\imprimircapa                                               % Comando para imprimir Capa
\imprimirfolhaderosto{}                                     % Comando para imprimir Folha de rosto
\imprimirfolhadeaprovacao{}                                 % Comando para imprimir Folha de aprovação
%% -----------------------------------------------------------------------------
% Dedicatória
% -----------------------------------------------------------------------------

\begin{dedicatoria}

I dedicate this thesis to all teachers that went in my life, 'cause without them, for sure, I wouldn't know that my profession is to be an eternal questioner...

\textit{Dedico este texto a todos os meus professores, pois sem eles, com certeza, não saberia que a minha profissão é ser um eterno aprendiz...}

\textit{Je dédie cette thèse à tous les enseignants et professeurs qui sont allés dans ma vie, car sans eux, surement, je ne saurais pas que ma profession est devenir un éternel questionneur ... Merci a vous !}

\end{dedicatoria}
           % Dedicatória
%% -----------------------------------------------------------------------------
% Agradecimentos
% -----------------------------------------------------------------------------

\begin{agradecimentos}[Acknowledgements]

First of all, I am greatfull to the creator of the universe, that gaves us inteligence and mysteries, which through science and philosophy will be unveiled.

To my family, I am thankful that they always given me all the encouragement and support necessary to carry out all my human and academic training. To my beloved wife, that supported me in all ways to keep searching and understanding what are the misteries of nature. To my father, I thank who always encouraged me to study scientific. To my mother, I thank, taste and fulfillment for teaching. To my brother I thank all the practice of patience and perseverance, in addition to keeping me in the realization of good practices and the usefulness of our work.

To my friend and adviser Allbens Atman, I thank the time and the efforts to teach me the path of the research and to all lessons inside and outside class, that I take with me.

To my friend and co-adviser Philippe Claudin, I thank for accepting me as his student, besides all disponibility and pacience that leads to this research.

I thank to the support given from the Graduate Program in Mathematical and Computing Modelling - PPGMMC at \textit{Centro Federal de Educação Tecnológica de Minas Gerais} - CEFET-MG, and from the laboratory \textit{Physique et Mécanique des Milieux Hétérogènes} - PMMH at \textit{École Supérieure de Physique et de Chimie Industrielles de la ville de Paris} - ESPCI.

I am also thankful to all my teachers and professors, once without them, I would certainly not be here.

To my friends, I thank their compreension, whom give me support in this project.

And specially, I am greatful to every brazilian citzen, that allow me to do this research in this public institution payed with taxes. I am also greatful to the finacial support given by \textit{Conselho Nacional de Pesquisa} - CNPq grant 88881.187077/2018-01, \textit{Coordenação de aperfeiçoamento de Pessoal de Nível Superior} - CAPES, \textit{Fundação de Amparo à Pesquisa do Estado de Minas Gerais} - FAPEMIG, PPGMMC, CEFET-MG, PMMH, \textit{Centre National de la Recherce Scientifique} - CNRS, and \textit{Fundação CEFETMINAS}.

%É obrigatório o agradecimento às instituições de fomento à pesquisa que financiaram total ou parcialmente o trabalho, inclusive no que diz respeito à concessão de bolsas.

\end{agradecimentos}

\begin{agradecimentos}[Agradecimentos]

Agradeço primeiramente ao grande criador de todo o universo, que nos deu inteligência e os mistérios, que pelas ciências e filosofia os serão desvendados.

À minha família, que sempre deu todo o incentivo e suporte necessários para realizar toda a minha formação humana e acadêmica. À minha grande companheira que sempre me deu apoio e incentivo, direto e indireto, a sempre continuar desvendando os mistérios naturais. Ao meu pai que sempre me incentivou o estudo científico. A minha mãe o gosto e a realização pela docência. Ao meu irmão toda a prática de paciência e perseverança, além de me manter na realização das boas práticas e da utilidade de nossos trabalhos.

Ao meu amigo e orientador Allbens Atman, por todo o tempo e esforço em ensinar os caminhos de como fazer pesquisa e por todas as lições dentro e fora da sala de aula que levo comigo.

Ao meu amigo e co-orientador Philippe Claudin, por ter me aceitado como seu aluno, além de toda a disponibilidade e paciência para que pudessemos chegar neste estágio da pesquisa.

Ao Centro Federal de Educação Tecnológico de Minas Gerais - CEFET-MG, ao Programa de Pós-Graduação em Modelagem Matemática e Computacional - PPGMMC e ao laboratório de \textit{Physique et Mécanique des Milieux Hétérogènes} - PMMH da \textit{École Supérieure de Physique et de Chimie Industrielles de la ville de Paris} - ESPCI que disponibilizaram seus recursos, pessoal e apoio nesta pesquisa.

À todos os meus professores, que certamente contribuíram para a minha chegada até este ponto.

Aos meus amigos e colegas que sempre compreenderam as ausências, e que me deram a ajuda e o suporte para continuar trabalhando neste projeto, sejam como ouvidos para os desabafos, sejam para renovar as ideias da pesquisa.


%É obrigatório o agradecimento às instituições de fomento à pesquisa que financiaram total ou parcialmente o trabalho, inclusive no que diz respeito à concessão de bolsas.

À cada cidadão brasileiro, que me permitiu pesquisar em uma escola paga com os impostos do povo, através de suas agências de fomento à educação e à pesquisa. Ao Conselho Nacional de Pesquisa - CNPq, especialmente na chamada de número 88881.187077/2018-01, à Coordenação de Aperfeiçoamento de Pessoal de Nível Superior - CAPES, à Fundação de Amparo à Pesquisa do Estado de Minas Gerais - FAPEMIG, ao CEFET-MG, ao PMMH e à ESPCI, ao \textit{Centre National de la Recherche Scientifique} - CNRS e à Fundação CEFETMINAS pelo suporte financeiro neste projeto de pesquisa, que teve início desde minha primeira iniciação científica em 2008.

\end{agradecimentos}
        % Agradecimentos
%% -----------------------------------------------------------------------------
% Epígrafe
% -----------------------------------------------------------------------------

\begin{epigrafe}

%\textit{“Saber que sabemos o que sabemos, e saber que não sabemos o que não sabemos, esta é a verdadeira sabedoria.”}
%(Nicolau Copérnico)

%\textit{“São grandes as vantagens industriais derivadas do princípio econômico da divisão do trabalho, porém, por causa disso, privou-se o trabalho do homem de alma e de vida.”}
%(Johannes Kepler)

\textit{“All truths are easy to understand once they are discovered; the point is discover them.”}

\textit{“Todas as verdades são fáceis de perceber depois de terem sido descobertas; o problema é descobri-las.”}

\textit{"Il est facile comprendre toutes les vérités une fois qu'elles sont découvertes ; le point est de les découvrir."}

(Galileu Galilei)

%\textit{“A gravidade explica os movimentos dos planetas, mas não pode explicar quem colocou os planetas em movimento. Deus governa todas as coisas e sabe tudo que é ou que pode ser feito.”}
%(Isaac Newton)

%\textit{“Ninguém é tão sábio que não tenha algo pra aprender e nem tão tolo que não tenha algo pra ensinar.”}
%(Blaise Pascal)

%\textit{“Devemos admitir com humildade que, ao passo que os números são puramente produtos de nossas mentes, o espaço tem uma realidade fora de nossas mentes, de modo que não podemos descrever completamente suas propriedades a priori.”}
%(Carl Friedrich Gauss)

%\textit{“Leia Euler, leia Euler, ele é o mestre de todos nós.”}
%(Pierre Simon Laplace)

\end{epigrafe}

% -----------------------------------------------------------------------------
% Edite o texto acima para inserir uma epígrafe de sua preferência
% -----------------------------------------------------------------------------
              % Epígrafe
% -----------------------------------------------------------------------------
% Abstract
% -----------------------------------------------------------------------------

\begin{resumo}[Abstract]
    The simulation of granular materials is studied widely in many research centers around the world, and applied in industries and engineering companies. For the understanding and quantification of granular materials properties, Discrete Element Method (DEM) is one of the most used technique to simulate the behavior of granular materials.
    Many of the challenges to understanding the behavior of granular materials begin in the dry grain segregation phenomenon. Classically, we have the Brazil Nut Effect (BNE) - which consists of a confined granular material containing grains of different sizes which, when agitated, displays segregation, with the larger grains rising up to surface. For many years, it was believed that this segregation occurred due to the presence of walls that confine the material. In the first part of this thesis we show that in systems with periodic boundary conditions (pbc), BNE can also occur. We also proposed that BNE exhibits resonance effect, and we differentiate systems with walls and pbc by using Large-Deviation function (LDF).
    In the second part of this thesis we studied a different area of granular materials: sediment transport. Sediment transport occurs in the interaction between granules and fluids. To simulate the behavior of granular materials immersed in a fluid, we use a Computational Fluid Dynamics (CFD) technique. The solid sediments move in the velocity field transported by the fluid. Three dimensionless parameters are required to describe the transport: the Reynolds number, which relates the inertial forces to the viscous forces, and consequently the fluid turbulence effects; the number of Shields, which is related to the drag forces and the inertial forces of the fluid; and finally, the density ratio between the solid and the fluid phases. It is possible to reproduce the different modes of transport only by changing such dimensionless parameters. In this thesis, we calculate and characterize the saturation time for bedload transport mode in the viscous regime, and we also predict the saturation length for this transport mode. This sediment transport was possible to studied thanks to the sandwich PhD. done in PMMH-ESPCI with CAPES grant No.88881.187077/2018-01.

    \textbf{Keywords}: Granular materials. Computer simulations. Discrete Element Method (DEM). Computational Fluid Dynamics (CFD). Brazil-Nut Effect. Sediment transport.
\end{resumo}
             % Resumo na língua vernácula
% -----------------------------------------------------------------------------
% Resumo
% -----------------------------------------------------------------------------

\begin{resumo}
    A simulação de materiais granulares é estudada nas academias de todo o mundo, também aplicada em indústrias e empresas de engenharia. Para o entendimento e quantificação das proprieadades dos materiais granulares, o Método de Elementos Discretos, ou \textit{Discrete Element Method} (DEM), é usado para simular o comportamento de materiais granulares.
    Muitos dos desafios de se compreender o comportamento de materiais granulares têm início no fenômeno de segregação de grãos secos. Classicamente, temos o efeito castanha do Pará - \textit{Brazil Nut Effect} (BNE) - que consiste em um material granular confinado contendo grãos de diferentes volumes e que, quando agitados, exibem segregação, sendo que os grãos maiores ascendem até a superfície. Por muitos anos, acreditou-se que esta segregação ocorria devido a presença de paredes que confinam o material. Nesta tese mostramos que em sistemas com condição periódica de contorno também pode ocorrer o BNE. Propomos que o BNE se comporta com efeito ressonate, e diferenciamos os sistemas com paredes do com condição periódica de contorno usando a função de grandes desvios - \textit{Large-Deviation function} (LDF).
    Estudamos também o transporte de sedimentos que ocorre na interação entre granulares e fluidos. Para simular o comportamento de materiais granulares imersos em um fluido, utilizamos uma técnica de Fluidodinâmica Computacional, ou \textit{Computational Fluid Dynamics} (CFD). Os sedimentos sólidos se movem em um campo de velocidades transportados pelo fluido. Três parâmetros adimensionais são necessários para descrever o comportamento do transporte: o número de Reynolds, que relaciona as forças inerciais com as forças viscosas, e consequentemente os efeitos de turbulência do fluido; o número de Shields, que está relacionado com a as forças de arraste e as forças inerciais do fluido; e finalmente, a razão de densidade entre o sólido e a fase fluida. É possível reproduzir os diferentes modos de transporte apenas mudando tais parâmetros adimensionais. Nesta tese, calculamos o tempo de saturação para o modo \textit{bedload} no regime viscoso, e também predizemos o tempo de saturação para este mode de transporte. Este estudo de transporte de sedimentos foi possível graças ao doutorado sanduíche realizado no PMMH-ESPCI com a bolsa CAPES No.88881.187077/2018-01.

    \textbf{Palavras Chaves}: Materiais granulares. Simulação computacional. Método de Elemento Discreto (DEM). Fluidodinâmica Computacional (CFD). Efeito castanha do Pará (BNE). Transporte de sedimentos.

%    Síntese do trabalho em texto cursivo contendo um único parágrafo.
%    Para uma Tese de Doutorado o resumo deve conter, no máximo, 500 palavras.
%    Para uma Dissertação de Mestrado o resumo deve conter, no máximo, 250 palavras.
%    Para um Projeto de Qualificação o resumo deve conter, no máximo, 200 palavras.
%    O resumo é a apresentação clara, concisa e seletiva do trabalho.
%    No resumo deve-se incluir, preferencialmente, nesta ordem:brevíssima introdução ao assunto do trabalho de pesquisa (incluindo motivação e justificativa para a realização deste trabalho), o que será feito no trabalho (objetivos), como ele será desenvolvido (metodologia), quais são os principais resultados obtidos ou esperados e a conclusão (compare os resultados com os da literatura e destaque as principais contribuições científicas do trabalho.

%    \textbf{Palavras-chave}: Modelo Latex. Trabalho acadêmico monográfico. Normas ABNT. Outra palavra.
\end{resumo}

% -----------------------------------------------------------------------------
% Escolha de 3 a 6 palavras ou termos que descrevam bem o seu trabalho. As palavras-chaves são utilizadas para indexação.
% A letra inicial de cada palavra deve estar em maiúsculas. As palavras-chave são separadas por ponto.
% -----------------------------------------------------------------------------
             % Resumo em língua inglesa
% -----------------------------------------------------------------------------
% Resumo
% -----------------------------------------------------------------------------

\begin{resumo}[Résumé]
    Des simulation de matériaux granulaires ont étudié en centres de recherche de tout le monde, également ils ont utilisé dans la industrie et des société d'ingénierie. Pour comprenez et qualifiez des matériaux granulaire, un Méthode de Éléments Discrètes, ou \textit{Discrete Element Method (DEM)}, simule des comportement des matériaux granulaires.
    De nombreux défis pour comprenez le comportement des matériaux granulaires commencement par le phénomène de ségrégation de grains secs. Classiquement, il y a l'effet noix du Brésil - \textit{Brazil Nut Effect} - qui consiste en des matériaux granulaires confinés contenant grains de différents volumes et qui, lorsqu'ils son agités, se produit une ségrégation, le plus gros grains se séparant des plus petits grains. Pendant de nombreuses années, on a cru que cette ségrégation était due à la présence de murs qui confinement le matériau. Dans cette thèse, nous montrons que dans le systèmes à condition aux limites périodiques, le BNE peut également se produire.
    Nous avons également étudié le transport des sédiments qui se produit dans l'interaction entre le granules et les fluides. Pour simuler le comportement de matériaux granulaires immergés dans un fluide, nous utilisons une technique de Dynamique des Fluides Computationnelle, ou \textit{Computational Fluid Dynamics (CFD)}. Les sédiments solides se déplacent dans un champ de vitesses portées par le fluide. Trois paramètres adimensionnels sont nécessaires pour décrie le comportement de transport: le nombre de Reynolds, qui est lié aux forces d'inertie aux forces visqueuses, et par conséquent aux effets de la turbulence des fluides; le nombre de Shields, qui est lié aux forces de traînée et aux forces d'inertie du fluide; et enfin, le rapport de densité entre le solide et le fluide. Il est possible de reproduire les différents modes de transport simplement en modifiant ces paramètres adimensionnelles. Dans cette thèse, nous calculons le temps de saturation des modes de transport.

    \textbf{Mots clés}: Matériaux granulaires. Simulation par ordinateur. Méthode des éléments discrets (DEM). Dynamique des fluides computationnelle (CFD). Effet de Noix du Brésil (BNE). Transporte de sédiments.

%    Síntese do trabalho em texto cursivo contendo um único parágrafo.
%    Para uma Tese de Doutorado o resumo deve conter, no máximo, 500 palavras.
%    Para uma Dissertação de Mestrado o resumo deve conter, no máximo, 250 palavras.
%    Para um Projeto de Qualificação o resumo deve conter, no máximo, 200 palavras.
%    O resumo é a apresentação clara, concisa e seletiva do trabalho.
%    No resumo deve-se incluir, preferencialmente, nesta ordem:brevíssima introdução ao assunto do trabalho de pesquisa (incluindo motivação e justificativa para a realização deste trabalho), o que será feito no trabalho (objetivos), como ele será desenvolvido (metodologia), quais são os principais resultados obtidos ou esperados e a conclusão (compare os resultados com os da literatura e destaque as principais contribuições científicas do trabalho.

%    \textbf{Palavras-chave}: Modelo Latex. Trabalho acadêmico monográfico. Normas ABNT. Outra palavra.
\end{resumo}

% -----------------------------------------------------------------------------
% Escolha de 3 a 6 palavras ou termos que descrevam bem o seu trabalho. As palavras-chaves são utilizadas para indexação.
% A letra inicial de cada palavra deve estar em maiúsculas. As palavras-chave são separadas por ponto.
% -----------------------------------------------------------------------------
             % Resumo em língua francesa
\include{./01-elementos-pre-textuais/lista-figuras}         % Lista de figuras
\include{./01-elementos-pre-textuais/lista-tabelas}         % Lista de tabelas
\include{./01-elementos-pre-textuais/lista-quadros}         % Lista de quadros
\include{./01-elementos-pre-textuais/lista-algoritmos}      % Lista de algoritmos
% -----------------------------------------------------------------------------
% Lista de Siglas
% -----------------------------------------------------------------------------

\begin{siglas}
%    \item[ABNT] Associação Brasileira de Normas Técnicas
%    \item[DECOM] Departamento de Computação
    \item[BNE] Brazil-Nut Effect
    \item[CFD] Computational Fluid Dynamics
    \item[DEM] Discrete Element Method
    \item[MD] Molecular Dynamics
    \item[LDF] Large Deviation Function
    \item[pbc] periodic boundary condition
    \item[fw] frictional walls
    \item[flw] frictionless walls
    \item[FDM] Finite Difference Method
\end{siglas}

% -----------------------------------------------------------------------------
% Edite a lista acima para definir "todos" os acrônimos e siglas utilizados neste trabalho
% -----------------------------------------------------------------------------
          % Lista de abreviaturas e siglas
% -----------------------------------------------------------------------------
% Lista de Símbolos
% -----------------------------------------------------------------------------

\begin{simbolos}
    \item[$\mu$] Friction coefficient
    \item[$\mathcal{G}$] Galileos number
    \item[$\mathcal{R}$] Reynolds number
    \item[$\Theta$] Shields number
    \item[$\rho$] Density
    \item[$\Gamma$] Dimensionless number that compares shaken acceleration and gravity
    \item[$\omega$] Oscillation frequency of shaken systems
    \item[$\phi$] Packing fraction
    \item[$m$] Mass
    \item[$\epsilon$] Restitution coefficient
    \item[$\gamma$] Normal damping coefficient
    \item[$k_n$] Normal spring coefficient
    \item[$k_t$] Tangential spring coefficient
    \item[$g$] Gravity
    \item[$\bar{\bar{\sigma}}$] Stress tensor
    \item[$\tau$] Shear stress
    \item[$P$] Pressure
    \item[$q$] Transport flux
    \item[$T_\textrm{sat}$] Saturation time
    \item[$L_\textrm{sat}$] Saturation length
    \item[$n$] Number of transported grains by area
    \item[$\bar{u}^p$] Average velocity of transported grains
\end{simbolos}
        % Lista de símbolos
\include{./01-elementos-pre-textuais/sumario}               % Sumário

% Insere os elementos textuais
\textual
% -----------------------------------------------------------------------------
% Introdução
% -----------------------------------------------------------------------------

\chapter{Introduction}
\label{chap:Introducao}

%    Materiais granulares estão presentes em vários contextos da natureza e das atividades humanas \cite{Sands_Powders_and_Grains, The_Physics_of_Granular_Media, Granular_Physics, Micromechanics_of_Granular_Materials, Granular_Media_Between_Fluid_and_Solid}. Atividades econômicas, como produção agrícola, mineração e tecnologia de construção, são essencialmente ligadas ao uso de materiais granulares \cite{Sands_Powders_and_Grains}. Por muitos anos, os estudos em materiais granulares estiveram presentes principalmente nas engenharias \cite{Versuche_uber_Getreidedruck_in_Silozellen, Janssen}, com o intuito de otimizar os processos de produção, armazenagem, escoamento e aplicações estruturais destes materiais. Hoje, algumas áreas da física, como a mecânica estatística \cite{Unifying_Concepts_in_Granular_Media_and_Glasses}, estudam intensamente a caracterização do comportamento destes materiais e suas aplicações, pela riqueza dos fenômenos observados. Sua ubiquidade reflete a importância dos estudos acerca de seu conhecimento, para que haja a manipulação destes elementos nas mais diversas situações.

    Granular materials are present in various contexts of nature and in many human activities \cite{Sands_Powders_and_Grains, The_Physics_of_Granular_Media, Granular_Physics, Micromechanics_of_Granular_Materials, Granular_Media_Between_Fluid_and_Solid}. Economic activities, like agricultural production, mining and building technology, are essentially linked to the usage of granular materials \cite{Sands_Powders_and_Grains}. For many years, research in granular materials were linked manly to engineering \cite{Versuche_uber_Getreidedruck_in_Silozellen, Janssen}, in order to optimize production process, storing, flowing, and structural applications to these materials. Nowadays, some areas of physics, such as statistical mechanics \cite{Unifying_Concepts_in_Granular_Media_and_Glasses}, study intensively the characterization and behavior of these materials, and their applications, due to the richness of observed phenomena. Its ubiquity reflects the importance of studies about its knowledge, so that there is manipulation of these elements in the most diverse situations.

%    Materiais granulares podem ser caracterizados como um aglomerado de corpos maiores que algumas centenas de micrometros até o tamanho de asteroides \cite{Sands_Powders_and_Grains, The_Physics_of_Granular_Media}. Além do tamanho, outra característica dos corpos é se apresentarem no estado sólido. Suas interações resultam em dissipação de energia, seja por atrito, seja pela inelasticidade da interação. Não estão sujeitos à variações no movimento causadas por flutuações térmicas, e portanto, não exibem movimentos Brownianos. Mais caracterizações dos materiais granulares podem ser encontradas no capítulo \ref{chap:Trabalhos-Relacionados} desta tese.

    Granular materials can be characterized as a cluster of bodies larger than a few hundred micrometers up to the size of asteroids \cite{Sands_Powders_and_Grains, The_Physics_of_Granular_Media}. In addition to the size, another feature of bodies is that they are individually in solid state. Their interactions result in energy dissipation, either by friction or by inelastic collision interaction. They are not subject to various movement caused by thermal fluctuations, and therefore, do not exhibit Brownian movements. More characterizations of granular materials can be found in the chapter \ref{chap:Trabalhos-Relacionados} of this thesis.

%    A proposta de estudos deste trabalho baseia-se na realização de simulações computacionais de materiais granulares, utilizando o Método de Elemento Discreto, ou \textit{Discrete Element Method} (DEM), baseado no método de Dinâmica Molecular, ou \textit{Molecular Dynamics} (MD) \cite{Computer_Simulation_of_Liquids}. As simulações estão em 2D, os grãos tem geometria circular, estão sob a ação da gravidade, e possuem potencial de repulsão quando estão em contato. No contato também levamos em conta o atrito entre as partículas. Definidas as propriedades dos materiais, como dureza, atrito, massa, posição e raio, aplicamos as leis de Newton para realizar a simulação. Detalhamos tal equacionamento e peculiaridades da simulação no capítulo \ref{chap:DEM}. Em relação ao fluido, a descrição detalhada do equacionamento, considerações do fluido no problema de transporte e da Fluidodinâmica Computacional, ou \textit{Computational Fluid Dynamics (CFD)}, pode ser encontrada também no capítulo \ref{chap:DEM}.

    The aim of this work is to computationally simulate granular materials, using Discrete Element Method (DEM), based on the Molecular Dynamics (MD) \cite{Computer_Simulation_of_Liquids}. The simulations are in 2D, with grains that have circular geometry, have potential repulsion when in contact, and are under the action of gravity. In contact, we also take into account the friction between the grains. Once the properties of the materials are defined, such as hardness, friction, mass, position and radius, we apply Newton's laws to perform the simulation. We detail these equations and peculiarities of the simulation in the chapter \ref{chap:DEM}. The detailed description of the fluid equation, and its considerations, can also be found in chapter \ref{chap:DEM}, likewise the transport law and Computational Fluid Dynamics (CFD).

%    Dos fenômenos apresentados pelos materiais granulares, estudamos nesta tese, o efeito castanha do Pará, ou \textit{Brazil Nut Effect} (BNE), relacionado à segregação de grãos confinados quando são submetidos à vibração e em presença de um campo gravitacional. Grãos maiores segregam-se no topo, enquanto grãos menores afundam. O capítulo \ref{chap:BNE} fornece mais detalhes a respeito do \textit{BNE}, tanto do ponto de vista fenomenológico, quanto das simulações propostas.

    From the phenomena presented by the dry granular materials, we researched in this thesis, the Brazil Nut Effect (BNE), related to the segregation of confined grains when they are submitted to the vibration and in the presence of a gravitational field. Larger grains segregate at the top, while smaller grains sink. The chapter \ref{chap:BNE} provides more details about BNE, both from a phenomenological point of view and from the proposed simulations.

%    Estudamos também o fenômeno do transporte de sedimentos imersos em fluidos. Com as equações que regem os fluidos, a equação de Navier-Stokes \cite{Physical_Hydrodynamics, Fluid_Mechanics} é utilizada neste trabalho para modelar o fluido que escoa e carrega consigo parte do material granular. Existem alguns modos de transporte que são caracterizados pela maneira como os grãos são trasportados pelo fluido, os quais estão descritos no capítulo \ref{chap:Transporte-Sedimentos}.

    We also researched the phenomenon of sediment transport immersed in fluids. With the equations that govern fluids, the Navier-Stokes \cite{Physical_Hydrodynamics, Fluid_Mechanics} equation is used in this work to model the fluid that flows and carries part of the granular materials with. There are some transport modes that are characterized by the way the grains are transported by the fluid, which are briefly described in the chapter \ref{chap:Transporte-Sedimentos}. In this thesis, we focused in the bedload transportation mode.

\section{Justification}
\label{sec:justificativa}

%    No contexto da engenharia, necessita-se compreender como os processos são elaborados, de forma a ajustá-los para otimizar os custos de produção, transporte e armazenamento de materiais vitais às atividades humanas, como alimentos e minérios. Neste sentido, o entendimento do comportamento dos materiais, quando submetidos a certas condições, permite manipulá-los da forma de maior interesse, seja por uma necessidade de conservação do material, seja pelo transporte mais rápido ou pela eficiência de outro parâmetro na qual se pretende gastar menos recursos ou ter o maior retorno financeiro, energético ou social.

    In the context of engineering, it is necessary to understand how the processes are elaborated, in order to adjust them to optimize the production costs, transportation and storage of materials vital to human activities, such as food and ores. In this sense, the understanding of behavior of materials, when subjected to certain conditions, allows them to be manipulated in the way of greatest interest, whether due to the need to conserve the material, either due to faster transport or the efficiency of another parameter in which it is intended spend less resources or have the greatest financial, energy or social return.

%    Na segregação de grãos, problemas relacionados a entupimentos podem ocorrer dependendo da geometria dos materiais \cite{Caio-Tese}. Quando estes materiais são submetidos à vibração, ou quando escoam, naturalmente separam-se os maiores dos menores, facilitando assim a filtração, porém dificultando a mistura. Estes conjuntos de aglomerados podem trazer consequências negativas aos processos industriais, como desgaste de silos relacionado à resistência dos materiais, corrosão do silo ou do granular, apodrecimento ou envelhecimento de alimentos estocados, etc. \cite{Silo_failures}.

    In grain segregation, problems related to clogging may occur depending on the geometry of the materials \cite{Caio-Tese}. When these materials are subjected to vibration, or when they flow, the largest are naturally separated of the smallest, thus facilitating the filtration, but making it difficult to mix. These sets of agglomerates can have negative consequences for industrial process, such as wear of silos related to the resistance of materials, corrosion of the silo or granular, rotting or aging of stored food, etc. \cite{Silo_failures}.

%    Assim, compreender como os materiais interagem e como são transportados, sejam transportados em uma esteira, sejam levados pelas correntezas de um rio, dá a possibilidade de controlar seus possíveis efeitos, ou prever suas consequências, como por exemplo a quantidade de resíduos remanescentes nos rios devido ao rompimento das barragens de Fundão em Mariana-MG em novembro de 2015 \cite{Mariana_en, Mariana_pt, Mariana_fr}, e a barragem de Brumadinho em janeiro de 2019 \cite{Brumadinho_en, Brumadinho_pt, Brumadinho_fr}.

    Thus, understanding how the materials interact and how they are transported, whether transported on a conveyor belt, or carried by the currents of a river, gives the possibility to control its possible effects, or to predict its consequences, such as the amount of residues remaining in in the rivers due to the collapse of the Fundão dams in Mariana-MG in November 2015 \cite{Mariana_en, Mariana_pt, Mariana_fr}, and the Brumadinho dam in January 2019 \cite{Brumadinho_en, Brumadinho_pt, Brumadinho_fr}.

\section{Motivation}
\label{sec:motivacao}

%    Por percebermos que ainda falta compreensão nos fenômenos envolvendo materiais granulares, propomos estudar nesta tese as técnicas que possam predizer o comportamento do aglomerado, ou caracterizar alguma propriedade emergente do sistema que ainda não tenha sido relatada ou documentada, bem como utilizar resultados já conhecidos dos materiais e procurar aplicá-los em outras áreas que a medição faz-se dificultada.
    
    Because we realize that there is still a lack of understanding in the phenomena involving granular materials, we propose to study in this thesis the techniques that can predict the behavior of the conglomerate, or to characterize some emerging properties of the system that has not yet been reported or documented, as well as to use results already known from the materials, and try to apply them in other areas where measurement is difficult.

%    Especificamente, trataremos de um assunto que ainda não havia sido reproduzido na literatura: o BNE em duas dimensões com condição periódica de contorno. Para analisar este problema utilizamos a técnica de \textit{Large Deviation Function} (LDF) \cite{Large_Deviations_in_Physics}.

    Specifically, we will deal with a subject that has not yet been reproduced in the literature: BNE in two dimensions with periodic boundary conditions. To analyze this problem we use the Large Deviation Function (LDF) \cite{Large_Deviation_in_Physics} technique.

%    Medimos também a medida de escala de tempo de saturação\footnote{Tempo de saturação em transporte de sedimentos indica o tempo característico que o material leva para entrar em regime permanente, saindo de uma configuração e chegando na configuração final.} do material quando transportado por um fluido.

    We also measure the saturation time scale\footnote{Saturation time in sediment transport indicates the characteristic time it takes for the material to enter a steady state, leaving a configuration and arriving at the final configuration.} of the material when transported by a fluid.

\section{Work organiztion}
\label{sec:organizacaoTrabalho}

%    Este trabalho divide-se em um capítulo de revisão bibliográfica sobre materiais granulares, descrito no capítulo \ref{chap:Trabalhos-Relacionados}, com fenomenologia sobre o estudo de materiais granulares. As descrições do capítulo \ref{chap:DEM} dizem respeito ao equacionamento e da modelagem do sistema, subdividido em duas partes principais: a primeira diz respeito das forças de interação dos grãos e como são implementadas as equações regentes neste sistema, enquanto a segunda parte trata do fluido e como implementá-lo. O capítulo \ref{chap:BNE} refere-se em especial ao fenômeno conhecido como BNE. O capítulo \ref{chap:Transporte-Sedimentos} traz as caracterizações de cada modo de transporte, e as descrições do transporte de materiais por fluidos em regime laminar. No capítulo \ref{chap:Resultados} contém as discussões sobre os resultados obtidos, divididos em duas secções: a primeira referente ao BNE, a segunda referente ao transporte de sedimentos. O capítulo \ref{chap:Conclusao} apresenta as conclusões desta tese e perspectivas de continuação deste trabalho.

    This work is divided into a bibliographic review chapter on granular materials, described in chapter \ref{chap:Trabalhos-Relacionados}, with phenomenology on the study of granular materials. The descriptions of the chapter \ref{chap:DEM} relate to the equation and modeling of the system, subdivided into two main parts: the first concerns the interaction forces of the grains and how the governing equations are implemented in this system, while the second part deals with the fluid and how to implement it. The chapter \ref{chap:BNE} refers in particular to the phenomenon known as BNE. The chapter \ref{chap:Transporte-Sedimentos} represents the characterizations of each transport mode, and the descriptions of the transport of materials by fluids in laminar regime. The chapter \ref{chap:Resultados} contains discussions on the results obtained, divided into two sections: the first referring to the BNE, the second referring to the transport of sediments. The chapter \ref{chap:Conclusao} presents the conclusions of this thesis and perspectives for the continuation of this work.
                % Introdução
% -----------------------------------------------------------------------------
% Trabalhos Relacionados
% -----------------------------------------------------------------------------

\chapter{Granular Materials}
\label{chap:Trabalhos-Relacionados}

%Cada capítulo deve conter uma pequena introdução (tipicamente, um ou dois parágrafos), em seção não numerada, que deve deixar claro o objetivo e o que será discutido no capítulo, bem como a organização do capítulo.

%    Materiais granulares são conjunto de corpos sólidos, que individualmente podem ser compostos de um mesmo material ou de diferentes materiais, de geometria das mais variadas, podendo ter várias densidades, coeficiente de atrito, dureza, e várias outras propriedades físicas que os materiais possuem, mas todos são maiores que $100\mu m$, e portanto visíveis a olho nu \cite{Sands_Powders_and_Grains}. Interagem entre si quando estão em contato uns com os outros, perdendo energia, tanto na forma de dissipação inelástica, quanto no atrito entre os grãos.
%    Os corpos sólidos que constituem os materiais granulares são grandes o suficiente para não apresentarem influência cinemática em função da temperatura termodinâmica. Assim sendo, movimentos Brownianos são ausentes nesse tipo de sistema.

    Granular Materials are sets of solid bodies, which individually can be composed by same material or different materials. They can have variety geometry, different densities, friction coefficient, hardness, etc., but an individual grain must be larger than $100\mu m$ \cite{Sands_Powders_and_Grains} due to the athermous nature. The solid bodies that composes granular materials are large enough to do not present kinetic fluctuation induced by thermodynamic temperature. Therefore, Brownian motion do not appear in those systems. Granular materials interact each other when they are in contact, loosing energy in inelastic collision\footnote{Inelastic collision is a loss of kinetic energy due to the contact of the bodies, in which they have transformed part of the energy to heat and they may deform in the process \cite{Halliday}. We are modeling the inelastic collision between two grains in section \ref{subsubchap:Reologia}.}, as well as friction.

%    Segundo a \href{http://webofknowledge.com}{\textit{Web of Science}}, o número de produções publicadas com a palavra chave \textit{"Granular Materials"}, até $03/04/2018$, é de $8.618$ e segue a distribuição apresentada na figura \ref{fig:articles-year} ao longo dos anos. O estudo de materiais granulares também segue uma tendência crescente ao longo dos anos.
    
%\begin{figure}
%    \centering
%    \includegraphics[width=0.8\textwidth]{04-figuras/articles-year.png}
%    \caption{Produção científica acerca de materiais granulares com as palavras chave \textit{"Granular Materials"} ao longo dos anos.}
%    \label{fig:articles-year}
%\end{figure}

\section{Theory}
\label{subsection:Teoria}

%    Como exemplos de materiais granulares temos areia, pedras, solos, fármacos, minérios, alimentos em grãos (arroz, milho, soja, etc.), e até mesmo o cinturão de asteroides e os anéis de Saturno. Só a areia compõe $10\%$ dos materiais da superfície do planeta Terra. Além disso, estima-se que o segundo material mais utilizado nas indústrias são materiais granulares, utilizando aproximadamente $10\%$ de toda a energia do planeta, sendo que o material mais utilizado é a própria água \cite{Sands_Powders_and_Grains}.

    Examples of granular materials includes sand, stones, soils, drugs, ores, grain foods (rice, corn, soybeans, etc.), even the asteroid belt and Saturn's rings. The sand alone makes up $10\%$ of the materials on surface of planet Earth. Besides that, it is estimated that the second most used material in industries are granular materials, using approximately $10\%$ of all the energy on the planet, with the most used material being water itself \cite{Sands_Powders_and_Grains}.

    \begin{figure}
        \begin{minipage}{.45\linewidth}
            \centering
            \includegraphics[width=0.9\textwidth]{04-figuras/Exemplo_Alimento.png}
            \subcaption{Grain foods.}
            \label{subfig:exemplo_alimento}
        \end{minipage}
        \begin{minipage}{.45\linewidth}
            \centering
            \includegraphics[width=0.9\textwidth]{04-figuras/Exemplo_Medicamento.png}
            \subcaption{Drugs.}
            \label{subfig:exemplo_medicamento}
        \end{minipage}
        \begin{minipage}{.45\linewidth}
            \centering
            \includegraphics[width=0.9\textwidth]{04-figuras/Exemplo_Açucar.png}
            \subcaption{Sugar.}
            \label{subfig:exemplo_acucar}
        \end{minipage}
        \begin{minipage}{.45\linewidth}
            \centering
            \includegraphics[width=0.9\textwidth]{04-figuras/Exemplo_Saturno.png}
            \subcaption{Saturn and its rings.}
            \label{subfig:exemplo_saturno}
        \end{minipage}
        \caption{Examples of granular material. Figures taken from \cite{Granular_Media_Between_Fluid_and_Solid}.}
    \end{figure}

%    Pela ausência de movimentos Brownianos, bem como pela dissipação de energia, sistemas granulares não sofrem relaxação espontânea de suas configurações estáveis na ausência de perturbações externas, principalmente na forma de vibrações, e portanto não apresentam ergodicidade\footnote{Um sistema ergódico tem característica de transitar entre seus microestados de energia espontaneamente, em intervalos de tempos, implicando que seus estados são todos equiprováveis quando analisados em um longuíssimo tempo \cite{Dissertacao, Srdjan-Tese, Granular_Solids_Liquids_and_Gases}.}.

    Due to the absence of Brownian movements, as well as the dissipation of energy in the contact, granular systems does not undergo spontaneous relaxation of its stable configurations in the absence of external disturbances, and therefore do no have ergodicity\footnote{An ergodic system has the characteristic of moving between their micro-states of energy spontaneously, in intervals of time, implying that their states are all equiprobable when analyzed in a very long time \cite{Srdjan-Tese, Unifying_Concepts_in_Granular_Media_and_Glasses}.}.

    To demonstrate this non-ergodicity, we can think about a dry sand pile that rests at a base. If this base does not oscillate, the structure of the pile does not change, the structure of internal forces will remain unchanged, even if it is heated or cooled. This means that sand grains cannot transit between all equipotential states spontaneously, and than this sand pile will rest with internal configurations (chain-forces, stress tensors, grain contact, etc.) unchanged. In the section \ref{subchap:Fenomenologia}, one can find more details.

%    Materiais granulares apresentam também particularidades quanto às suas fases. Apresentam-se individualmente em corpos sólidos, e quando o conglomerado está próximo do repouso, constituem a fase sólida do sistema. Porém, se o sistema é agitado, ou configurado além de um limiar crítico do ângulo de repouso, pode apresentar-se nas fases de granular líquido\footnote{Granulares líquidos podem possuir uma camada limite que flui sobre a camada sólida do sistema.} ou granular gasoso. Tal classificação ainda está em aberto na literatura, apesar de existirem proposições para o que seria a temperatura granular do sistema \cite{Granular_Solids_Liquids_and_Gases}.

    Granular materials also have particularities regarding their phases, analogously to the state of matter. They are presented individually in solid bodies, and when the conglomerate is close to rest, they constitute the equivalent solid phase of the state of matter. However, if the granular system is slightly agitated, or configured beyond a critical threshold of angle of rest, it can be interpreted in similarity of the liquid state of matter\footnote{Liquid granules can have a boundary layer that flows over the sold layer of the system.}. Granular gases are granular systems that are vigorously agitated, with low packing fraction\footnote{Packing fraction is the measure of the occupied space by the solid portion in relation of the total space.}, and they tend to occupy a large part of the recipe which contains them. An example of the granular state is shown in figure \ref{fig:exemplo_fases}, where the solid phase is static in the bottom, the liquid phase is flowing though layers in the middle, and the gaseous phase is flowing in a higher disordered portion at top. Such classification is still open in the literature, although there are proposals for what would be the granular temperature of the system, in analogous of thermal temperature \cite{Granular_Solids_Liquids_and_Gases}.

    \begin{figure}
        \centering
        \includegraphics[width=0.9\textwidth]{04-figuras/Exemplo_Fases.png}
        \label{fig:exemplo_fases}
        \caption{Example of three granular phases according their kinetic energy. Figures taken from \cite{Granular_Media_Between_Fluid_and_Solid}.}
    \end{figure}

%    Uma diferenciação entre sistemas granulares pode ser resultado direto das forças de interações entre os grãos. São chamados de granulares secos os sistemas que possuem apenas interações repulsivas, enquanto os granulares molhados apresentam forças de van der Waals nas interações grão a grão. Nesta tese, consideraremos apenas as interações repulsivas de contato, apesar de que em alguns casos, existe fluido envolvendo o material. Consideramos que todo o material que está envolvido pelo mesmo fluido não sofre forças de atração entre os mesmos grãos, e portanto, não está inclusa força de van der Waals na interação entre os grãos.

    A differentiation between granular systems can be a direct result of the interaction forces between grains. Systems that have only repulsive interactions are called dry granulars, while wet granulars have van der Waals forces in grain-to-grain interactions. In this thesis, we will only consider repulsive contact interactions, although in some cases, there is fluid surrounding the material. We consider that all the material that is involved by the same fluid does not suffer forces of attraction between the same grains, and therefore, van der Waals force is not included in the interaction between the grains.

\section{Fenomenologia de Materiais Granulares}
\label{subchap:Fenomenologia}

%Pilha de areia
    Talvez a primeira ideia sobre materiais granulares remeta ao empilhamento de areira. Nesse caso, tem-se uma pilha estática de areia\footnote{A pilha estática está no estado sólido da fase granular \cite{Granular_Solids_Liquids_and_Gases}.}, amontoada sobre uma superfície. Note que em uma pilha como essa, os grãos sempre atingem uma determinada altura, e quando coloca-se mais material sobre a pilha, em algum momento, as camadas superiores da pilha escorrem até a base. Sempre que a pilha ultrapassar o ângulo crítico de repouso \cite{Granular_Physics}, ocorrerá uma avalanche, restaurando o sistema a um outro ângulo característico. Esta propriedade é a de auto-organização\footnote{Um sistema que não possui controlador central, regido por vários agentes que interagem entre si, com regras conhecidas na interação dos agentes e exibem propriedade não prevista pelas interações entre os agentes caracterizam um Sistema Complexo. Uma propriedade característica de Sistemas Complexos e de materiais granulares é a auto-organização. Alguns autores \cite{Mixing_and_Segregation_of_Granular_Materials, Measuring_the_flowing_properties_of_powders_and_grains, Revisiting_localized_deformation_in_sand_with_complex_systems, Granular_matter_and_networks, Patterns_and_collective_behavior_in_granular_media} classificam materiais granulares dentro da área de estudo de Sistemas Complexos.} da pilha de areia pelo ângulo crítico de repouso. Uma boa aproximação do ângulo de repouso é dada pela equação \ref{equ:atrito}:
\begin{equation}
    \label{equ:atrito}
    tan(\theta) = \mu _s ,
\end{equation}
onde $\theta$ é o ângulo crítico de repouso, e $\mu _s$ é o coeficiente de atrito do material.

%Biestabilidade da pilha
    Um pouco mais curioso ainda sobre as pilhas de grãos é que o histórico de preparação do sistema reflete-se no ângulo de repouso \cite{Dynamics_at_the_angle_of_repose}. Este histórico de preparação permite o sistema configurar-se diferentemente, e portanto, o ângulo de repouso assume valores diferentes utilizando o mesmo material. Existe então um ângulo de repouso mínimo $\theta _r$, e um ângulo máximo $\theta _m$, em que o empilhamento pode configurar-se: $\theta _r < \theta < \theta _m$ \cite{Granular_Physics}. Ter uma faixa de ângulos estáveis entre o ângulo mínimo de repouso e o ângulo máximo recebe o nome de biestabilidade do ângulo de repouso.

%Diferentes preparações, diferentes pressões
    Uma evidência de que o histórico de preparação altera a configuração do material é descrita nos artigos "\textit{Sensitivity of Stress Response Function to Packing Preparation}" e "\textit{Memories in sand: Experimental tests of construction history on stress distributions under sandpiles}" \cite{Sensitivity_of_Stress_Response_Function_to_Packing_Preparation, Memories_in_Sand}. A base circular apresentada na figura \ref{fig:pile_stress} foi feita com duas formas de deposição diferentes. No experimento, a pressão medida na base varia de acordo com a deposição, sendo que a deposição feita a partir do funil possui um pico de máximo de pressão em torno de $0,25$ e $0,5$ do raio da mesa, enquanto na deposição feita a partir da peneira apresenta pressão como espécie de platô entre o centro da mesa e $0,25$ do raio, com o máximo próximo do centro.

\begin{figure}
    \centering
    \begin{minipage}{.45\linewidth}
        \centering
        \includegraphics[width=0.9\textwidth]{04-figuras/Sand_Pile_GG_Experiment.png}
        \subcaption{Empilhamento a partir do funil.}
        \label{fig:pressure_pile:GG}
    \end{minipage}
    \begin{minipage}{.45\linewidth}
        \centering
        \includegraphics[width=0.9\textwidth]{04-figuras/Sand_Pile_GG_Pressure.png}
        \subcaption{Pressão na montagem a partir do funil.}
        \label{fig:pressure_response:GG}
    \end{minipage}
    \begin{minipage}{.45\linewidth}
        \centering
        \includegraphics[width=0.9\textwidth]{04-figuras/Sand_Pile_RL_Experiment.png}
        \subcaption{Empilhamento a partir da peneira.}
        \label{fig:pressure_pile:RL}
    \end{minipage}
    \begin{minipage}{.45\linewidth}
        \centering
        \includegraphics[width=0.9\textwidth]{04-figuras/Sand_Pile_RL_Pressure.png}
        \subcaption{Pressão na montagem a partir da peneira.}
        \label{fig:pressure_response:RL}
    \end{minipage}
    \caption{A preparação das pilhas de areia reflete nas pressões medidas na base da pilha. Nas figuras \ref{fig:pressure_pile:GG} e \ref{fig:pressure_response:GG} a deposição a partir do funil cria um perfil de pressões que tem o pico fora do centro da pilha, enquanto nas figuras \ref{fig:pressure_pile:RL} e \ref{fig:pressure_response:RL} a deposição a partir da peneira cria um perfil de pressões que tem um platô e depois decai. Figuras retiradas de \cite{Memories_in_Sand}.}
    \label{fig:pile_stress}
\end{figure}    

%Função resposta em diferentes configurações
    Um estudo feito por Atman \textit{et al.} \cite{Sensitivity_of_Stress_Response_Function_to_Packing_Preparation} mostra que diferentes geometrias de materiais granulares resultam em diferentes funções respostas\footnote{Função resposta é a diferença entre duas configurações, uma antes de aplicar-se a carga de teste e após a aplicação da carga, mostrando-se a distribuição de forças sobre o material, ou a compressão do sistema \cite{The_Physics_of_Granular_Media}.}. Como exemplo, a figura \ref{fig:stress_response} mostra duas funções respostas para sistemas com geometria circular e pentagonal.

\begin{figure}
    \centering
    \begin{minipage}{.45\linewidth}
        \centering
        \includegraphics[width=0.9\textwidth]{04-figuras/Funcao_Resposta1.png}
        \subcaption{Grãos circulares.}
        \label{fig:stress_response:circle}
    \end{minipage}
    \begin{minipage}{.45\linewidth}
        \centering
        \includegraphics[width=0.9\textwidth]{04-figuras/Funcao_Resposta2.png}
        \subcaption{Grãos pentagonais.}
        \label{fig:stress_response:pentagon}
    \end{minipage}
    \caption{A aplicação de uma força em diferentes sistemas granulares resulta em diferentes funções respostas. A diferença entre estes sistemas é que a figura \ref{fig:stress_response:circle} possui grãos de geometria circular e possui maior ordem, enquanto a figura \ref{fig:stress_response:pentagon} possui geometria pentagonal e maior desordem. Figuras retiradas de \cite{Sensitivity_of_Stress_Response_Function_to_Packing_Preparation}.}
    \label{fig:stress_response}
\end{figure}

%Cadeias de forças em diferentes pilhas
    Já que citamos as diferentes funções respostas nos empilhamentos de grãos, não podemos deixar de citar as cadeias de forças\footnote{Cadeias de forças consistem na rede de contatos entre os grãos que possuem força acima da força média do sistema. Em geral, mede-se as cadeias de forças são medidas a partir da função resposta \cite{The_Physics_of_Granular_Media}.}. A importância experimental da visualização das cadeias de forças se dá no entendimento da distribuição das forças internas que sustentam o material. Como exemplo, a figura \ref{fig:force_chain} indicia a cadeia de forças associada à função resposta de uma força puntual aplicada sobre o topo do material granular.

\begin{figure}
    \centering
    \includegraphics[width=0.5\textwidth]{04-figuras/Cadeia_Forca.png}
    \caption{A aplicação de uma força puntual no topo do material resulta na cadeia de forças, que pode ser vista através da função resposta do sistema. Neste caso, o sistema é preparado com grãos fotoelásticos em uma distribuição bidimensional. Quanto mais escuras, maiores são as tensões no material. Figura retirada de \cite{Sensitivity_of_Stress_Response_Function_to_Packing_Preparation}.}
    \label{fig:force_chain}
\end{figure}

%Formação de arcos
    As cadeias de forças são importantes para entender o fenômeno que está presente nos arcos de sustentação que utilizamos. Arcos são estruturas coletivas que possuem sustentação mútua, e, consequentemente, uma cadeia de forças ligando toda a estrutura, sendo capaz de sustentar o peso próprio e de todos os grãos acima, impedindo que os mesmos escoem. Na formação de arcos podem ocorrer efeitos de segregação, como verificado por Magalhães, C. e Magalhães, F. \cite{Caio-Tese, Felipe-Tese}.

%Pressão e tensão
    As medidas em materiais granulares geralmente tentam caracterizar o material em duas escalas diferentes que se relacionam: microescala, que diz respeito das medidas na escala dos grãos, como atrito; e macroescala, que diz respeito das medidas na escala do sistema, como pressão e tensão de cisalhamento.

%Dilatação
    Uma curiosidade sobre os materiais granulares é que quando estão submetidos a uma pressão e seu coeficiente de compactação\footnote{\label{foot:packingfraction}O coeficiente de compactação é dado pela razão da soma dos volumes individuais dos grãos pelo volume de ocupação no espaço.} está próximo do engarrafamento \cite{Non-Gaussian_behavior_in_jamming_unjamming_transition_in_dense_granular_materials}, uma dilatação tende a ocorrer, expandindo-se pelas bordas das fronteiras que confinam o material ou pelas outras direções de liberdade que o confinamento apresenta. Muitas vezes, ao aplicar-se uma pressão no material confinado, o coeficiente de compactação final pode ser menor que o inicial, indicando uma expansão volumétrica do sistema \cite{Felipe-Tese}.

%Escoamento de granular
    Aproveitando o exemplo do empilhamento, observa-se que na formação da pilha, após os grãos atingem o ângulo critico, ocasiona-se uma avalanche do material. Na avalanche, a camada superior entra em movimento, enquanto as camadas abaixo continuam estáticas. Na movimentação da camada superior, o material granular se apresenta no estado líquido, enquanto a camada estática abaixo encontra-se no estado sólido \cite{Granular_Solids_Liquids_and_Gases}. A figura \ref{fig:inclinacao} exemplifica a transição de fase sólido líquido entre as camadas.

\begin{figure}
    \centering
    \begin{minipage}{.45\linewidth}
        \centering
        \includegraphics[width=0.9\textwidth]{04-figuras/Pilha1.png}
        \subcaption{Pilha estática.}
        \label{fig:inclinacao:solido}
    \end{minipage}
    \begin{minipage}{.45\linewidth}
        \centering
        \includegraphics[width=0.9\textwidth]{04-figuras/Pilha2.png}
        \subcaption{Pilha escorrendo.}
        \label{fig:inclinacao:liquido}
    \end{minipage}
    \caption{Com o aumento do ângulo da pilha, nota-se que a camada superior desliza sobre a camada inferior (da figura \ref{fig:inclinacao:solido} para a figura \ref{fig:inclinacao:liquido}). Figuras retiradas de \cite{Granular_Solids_Liquids_and_Gases}.}
    \label{fig:inclinacao}
\end{figure}

    Escoamentos podem ocorrer então por tensões aplicadas no material, seja em uma inclinação da base, seja pela vibração do material. Como a mudança de configurações do material está relacionada a taxa de cisalhamento do material, mas a tensão de cisalhamento não é necessariamente proporcional a taxa de cisalhamento, este escoamento pode ser classificado como fluido não newtoniano.

%Segregação
%    Um fenômeno muito estudado é a segregação dos materiais granulares. O efeito de segregação ocorre em diferentes geometrias de material, densidades e coeficientes de atrito.

%Vibração
%    Vibrações no material granular fornecem energia ao sistema

    No próximo capítulo descreveremos as equações e os procedimentos para realizar as simulações dos materiais granulares, desde o modelo de contatos até a inserção do fluido no sistema.
    % Trabalhos relacionados
\include{./02-elementos-textuais/fundamentacao-teorica1}    % Fundamentação teórica - DEM
\include{./02-elementos-textuais/fundamentacao-teorica2}    % Fundamentação teórica - BNE
\include{./02-elementos-textuais/fundamentacao-teorica3}    % Fundamentação teórica - Fluido
%\include{./02-elementos-textuais/metodologia}               % Metodologia
%\include{./02-elementos-textuais/resultados}                % Resultados
%---------------------------------- CAPITULO VI ------------------------------%
%\thispagestyle{empty}
\chapter{Cronograma e Plano de Estudos}
\section{Cronograma}
\label{ch:Chronogram}

    Apresentamos uma proposta de doutorado sanduíche para iniciar dos trabalhos com o prof. Dr. Philippe Claudin, do laboratório \textit{Physique et Méchanique des Milliex Hétérogènes} (PMMH) da \textit{École Superieure de Physique et Chemie Industrielles de la ville de Paris} (ESPCI) em Setembro de 2018 e retornar ao Brasil em Julho de 2019. A tabela \ref{tab:CronogramaFRA} contém a proposta das atividades do doutorado sanduíche. Pretende-se contribuir com o desenvolvimento científico através da escrita de ao menos um artigo, publicado em periódico que possua alto fator de impacto ao final desta parceria.

\setlength{\tabcolsep}{3pt}

\begin{table}[h]
    \begin{tabular}{| l|c|c|c|c|c|c|c|c|c|c|c |}
        \textbf{Atividades} & \multicolumn{11}{c}{\textbf{Mês do ano}}                                       \\
                                           & Set & Out & Nov & Dez & Jan & Fev & Mar & Abr & Mai & Jun & Jul \\
\hline        Revisão bibliográfica        &  X  &  X  &  X  &  X  &  X  &  X  &  X  &  X  &     &     &     \\
\hline        Equacionamento do modelo     &  X  &  X  &  X  &  X  &  X  &     &     &     &     &     &     \\
\hline        Escrita do código fonte      &     &  X  &  X  &  X  &  X  &  X  &  X  &  X  &  X  &     &     \\
\hline        Validação do modelo          &     &     &  X  &  X  &  X  &  X  &  X  &  X  &  X  &  X  &     \\
\hline        Escrita da tese              &     &     &  X  &  X  &  X  &  X  &  X  &  X  &  X  &  X  &  X  \\
\hline        Resultados preliminares      &     &     &  X  &  X  &     &     &     &     &     &     &     \\
\hline        Ajustes dos parâmetros       &     &     &     &     &     &  X  &  X  &  X  &     &     &     \\
\hline        Análise dos resultados finais&     &     &     &     &     &     &     &  X  &  X  &  X  &  X  \\
\hline \hline Participação em congresso    &     &     &     &     &  X  &  X  &     &     &     &     &     \\
\hline        Escrita do artigo            &     &     &     &     &     &     &     &     &     &  X  &  X  
    \end{tabular}
    \caption{Atividades programadas para a realização do doutorado sanduíche.}
    \label{tab:CronogramaFRA}
\end{table}

    Após o retorno ao Brasil, pretendemos seguir com o cronograma apresentado na tabela \ref{tab:CronogramaBRA}, que contempla o encerramento desta tese pela compilação dos resultados obtidos durante o tempo de desenvolvimento do sanduíche e anteriormente.

\begin{table}[h]
    \begin{tabular}{| l|c|c|c|c|c |}
        \textbf{Atividades} & \multicolumn{5}{c}{\textbf{Mês do ano}}    \\
                                           & Ago & Set & Out & Nov & Dez \\
\hline        Compilação dos resultados    &  X  &  X  &     &     &     \\
\hline        Escrita da tese              &  X  &  X  &  X  &     &     \\
\hline        Marcação da banca            &     &     &     &  X  &     \\
\hline \hline Defesa da tese               &     &     &     &     &  X
    \end{tabular}
    \caption{Atividades programadas para a realização após o retorno do doutorado sanduíche.}
    \label{tab:CronogramaBRA}
\end{table}

\section{Plano de estudos do doutorado sanduíche}
\label{ch:Estudos}

Faremos uma revisão da bibliografia baseada nos métodos de simulação de materiais granulares, o qual o candidato, o orientador e o coorientador já possuem experiência e publicações internacional. A adição as técnicas de simulação da mecânica dos fluidos ao problema caracteriza-o como um problema de transporte. Tal revisão tem o intuito de aprimorar as técnicas computacionais e o melhor embasamento no equacionamento da interação entre fluido e granular, além de verificar o estado da arte e as últimas tendências do problema.

Como a definição do trabalho e de validação prévia, revemos a forma de equacionar a mecânica dos fluidos. O intuito é de encontrar uma forma computacional mais estável para resolver o FEM. Estudamos o sistema descrito no artigo \textit{"Numerical simulation of turbulent sediment transport, from bed load to saltation."}, publicado na \textit{Physics of Fluids}, de autoria do Dr. Philippe Claudin, o coorientador \cite{Numerical_simulation_of_turbulent_sediment_transport}.

Aprimoraremos o código fonte, adequando das equações que regem o sistema. Validaremos o modelo baseando-se nos resultados da literatura, como os modos de transportes e suas propriedades. Utilizaremos as métricas já descritas pela geografia física e pela mecânica estatística.

Analisaremos os resultados preliminares do transporte de grãos como base para o início da validação das equações do modelo e das regras que regem o sistema, a fim de obter resultados que exprimam a realidade. Em seguida, ajustaremos os parâmetros necessários para a realização específica das propriedades na qual desejamos observar e documentar em forma de artigos e na base da tese a ser escrita.

A participação em um congresso na área e a publicação em periódico internacional de relevância tornam-se de importantes para a divulgação das ideias e dos resultados. O congresso tem objetivo de fazer contatos com outros grupos de pesquisa, aprimorando assim as habilidades e a colaboração entre os assuntos tratados em âmbito internacional. O artigo promove boa oportunidade posicionar bem o Brasil e o CEFET-MG com as revistas de relevância para a ciência internacional. O código fonte é um dos produtos diretos da pesquisa, e que pode ser patenteado após a conclusão da mesma.

                % Cronograma
% -----------------------------------------------------------------------------
% Conclusão
% -----------------------------------------------------------------------------

\chapter{Conclusões Parciais}
\label{chap:Conclusao}

    Para os as simulações do BNE conseguimos reproduzir as propriedades de ascessão do intruso com diferentes densidades, diferentes amplitudes de vibração e diferentes frequências de vibração, observando a importância do atrito nas paredes do sistema. Mais do que isso, conseguimos realizar o BNE em um sistema que possui condição periódica de contorno e suas diferenças para o sistema de caixa fechada.

    Para o sedimento de transportes, conseguimos validar as propriedades físicas que regem o sistema, condizendo simulação com a conservação de movimento. Validamos o fluido de acordo com a literatura e acoplamos grão e fluido de forma a interagirem sobre as leis da física.

%Procure fazer uma análise crítica de seu trabalho, destacando os principais resultados e as contribuições deste trabalho para a área de pesquisa.

%\section{Trabalhos Futuros}
%\label{sec:trabalhosFuturos}

%Também deve indicar, se possível e/ou conveniente, como este trabalho pode ser estendido ou aprimorado.

%\section{Considerações Finais}
%\label{sec:consideracoesFinais}

%As derradeiras palavras para encerramento do seu trabalho acadêmico.

% -----------------------------------------------------------------------------
% OBS: a norma ABNT estabelece que em qualquer tipo de trabalho acadêmico monográfico
% deve haver um capítulo de conclusão
% -----------------------------------------------------------------------------
                 % Conclusão

% Insere os elementos pós-textuais
\postextual
\include{./03-elementos-pos-textuais/referencias}           % Referências
\begin{apendicesenv}
\partapendices

\chapter{Artigos publicados}
\label{chap:Artigo}
    A seguir os artigos publicados desde o início desta pesquisa. O primeiro artigo apresentado refere-se a publicação feita sobre este doutoramento, com resultados mistos das técnicas utilizadas na dissertação de mestrado \cite{Dissertacao} e este projeto de tese. O segundo e terceiro artigos apresentados referem-se a publicações feitas durante a dissertação, mas que expressam as técnicas utilizadas neste projeto de tese.

\section{\textit{Large-deviation quantification of boundary conditions on the Brazil nut effect}}
\label{appendix:BNE}
    This paper was published on the Physical Review E, and it is one of the main themes of this thesis, refering to chapter \ref{chap:BNE}.

\section{\textit{Methods of parallel computation applied on granular simulations}}

    Este artigo foi publicado no quatrienal do congresso \textit{Powders \& Grains 2017}, que é o maior congresso sobre materiais granulares, e que está em sua 8ª edição.

\includepdf[pages=-]{./08-apendice/ArtigoPG2017.pdf}

\section{\textit{Mechanical properties of inclined frictional granular layers}}

    Este artigo foi publicado na \textit{Granular Matter}, uma das maiores revistas sobre material granular e é uma revista A2 segundo a classificação \textit{qualis} da CAPES e possui fator de impacto de $1,762$.

\includepdf[pages=-]{./08-apendice/ArtigoGM.pdf}

\section{\textit{Non-Gaussian behavior in jamming / unjamming transition in dense granular materials}}

    Este artigo foi publicado no quatrienal do congresso \textit{Powders \& Grains 2013}, que é o maior congresso sobre materiais granulares, e que estava em sua 7ª edição.

\includepdf[pages=-]{./08-apendice/ArtigoPG2013.pdf}

\chapter{Códigos}

    Coloquei os códigos utilizados para este projeto de tese em um GIT para a maior comodidade e facilidade do acesso. O endereço eletrônico é \url{https://github.com/BoscoWarhammer/Doutorado}.

\end{apendicesenv}
             % Apêndices
%% -----------------------------------------------------------------------------
% Anexos
% -----------------------------------------------------------------------------

\begin{anexosenv}
\partanexos

% -----------------------------------------------------------------------------
% Solving the diffusion equation
% -----------------------------------------------------------------------------

\chapter{Solving the diffusion equation}
\label{chap:Calculations}
    In this part, I will show the calculations we did for previous parts in Chapter \ref{chap:CFD}, with some details and some tricks to handle with it, if it is necessary. In this sense, we can start with the linear diffusion equation (equations \ref{equ:diff0}), and its solution is presented at \cite{Boyce}, chapter 10.

    \begin{subequations}
        \begin{align}
            \rho \frac{\partial u}{\partial t} &= \frac{\partial \tau}{\partial z}, \\
            \tau &= \rho \nu \frac{\partial u}{\partial z}.
        \end{align}
        \label{equ:diff0}
    \end{subequations}

    To solve this system, we need to know two boundary conditions and the initial condition. As we have discussed in section \ref{sec:viscous_steady}, the boundary conditions gives us the initial and the final condition, once we are going to change between one known steady state to another known steady state with a difference of $\Delta u_*$ between them (equation \ref{equ:viscous_steady1} has all information to give us initial and final condition). To solve equations \ref{equ:diff0}, we apply a variable separation on the function by guessing the solution is in the form of:
    \begin{equation}
        u(t,z) = Z(z) T(t).
        \label{equ:diff_solution0}
    \end{equation}

    Rewriting equations \ref{equ:diff0} with guess \ref{equ:diff_solution0}, we get:
    \begin{equation}
        \frac{1}{T}\frac{\mathrm{d} T}{\mathrm{d} t} = \frac{\nu}{Z}\frac{\mathrm{d}^2 Z}{\mathrm{d} z^2} = -\lambda^2.
    \end{equation}

    As we know the steady state, and it must fulfill as solution of the equation, we will remove it from transient. Calling $s(z)=u(z,t \to \infty)$, the final steady state, and $w(z,t) = u(z,t)-s(z)$ the transient solution of the diffusion equation, we get:
    \begin{subequations}
        \begin{align}
            \frac{\partial u}{\partial t} = \frac{\partial}{\partial t}\left[s(z)+w(z,t)\right] & \implies \frac{\partial u}{\partial t} = \frac{\partial w}{\partial t} ,\\
            \frac{\partial^2 u}{\partial z^2} = \frac{\partial^2 }{\partial z^2}\left[s(z)+w(z,t)\right] & \implies \frac{\partial^2 u}{\partial z^2} = \frac{\partial^2 s}{\partial z^2} + \frac{\partial^2 w}{\partial z^2},
        \end{align}
    \end{subequations}
and since $s(z)=z (u_*+\Delta u_*)^2/\nu$, $\partial^2 s/\partial z^2 = 0$. Applying all condition, we get the following:
    \begin{subequations}
        \begin{align}
            w(z=0,t) = u(0,t)-s(0) & \implies w(z=0,t) = 0 & \text{ Bottom wall}, \\
            \frac{\partial w(z=h,t)}{\partial z} = \frac{\partial u(z=h,t)}{\partial z} -\frac{\partial s(z=h)}{\partial z} & \implies \frac{\partial w(z=h,t)}{\partial z} = 0 & \text{ Top shear}, \\
            w(z,t=0) = u(z,t=0) -s(z) & \implies w(z,t=0) = -z\frac{2u_*\Delta u_* +\left(\Delta u_*\right)^2}{\nu} & \text{ Initial}.
        \end{align}
    \end{subequations}

    The solutions for the transient regime are, considering that $w=w_t(t) w_z(z)$:
    \begin{subequations}
        \begin{align}
            \frac{\dot{w}_t}{\nu w_t} = -\lambda^2 & \implies w_t = Ae^{-\lambda^2 \nu t}, \\
            \frac{{w''}_z}{\nu w_z} = -\lambda^2 & \implies w_z = B_1 \sin \left(\lambda z\right) + B_2 \cos \left(\lambda z\right).
        \end{align}
    \end{subequations}

    Now we can see that solutions must have the form of an exponential decay according the transition between one steady state and another, and at same time, they should be written in modes of $\sin$ and $\cos$. To determine some missing coefficients, we need to apply boundary conditions to this solutions:
    \begin{subequations}
        \begin{align}
            w(z=0,t) = 0 & \implies B_2 = 0, \\
            \frac{\partial w(z=h,t)}{\partial z} = 0 & \implies \lambda = \pi \frac{2n-1}{2h},
        \end{align}
    \end{subequations}
and initial condition
    \begin{equation}
        w(z,t=0) = -z\frac{2u_*\Delta u_* +\left(\Delta u_*\right)^2}{\nu} \implies A.B_1 = \frac{2}{h}\int_0^h\left[-z\frac{2u_*\Delta u_* +\left(\Delta u_*\right)^2}{\nu}\right]\sin\left(\lambda z\right)\mathrm{d}z,
    \end{equation}
since we have rewritten initial condition to same base of solution, by a Fourier's series. Fourier modes appear in initial condition "caused" by boundary conditions in the nature of the equation. Only certain frequencies can appear in the solution, and they are given by this Fourier's frequencies of final solution:
    \begin{equation}
        \frac{2}{h}\int_0^h\left[-z\frac{2u_*\Delta u_* +\left(\Delta u_*\right)^2}{\nu}\right]\sin\left(\lambda z\right)\mathrm{d}z = 
        \frac{8h}{\pi^2}\left[\frac{2u_*\Delta u_* +\left(\Delta u_*\right)^2}{\nu}\right]\sum_{n=1}^\infty\frac{\left(-1\right)^n}{\left(2n-1\right)^2}\sin\left(\lambda z\right),
    \end{equation}
just another way to describe the initial condition in same base of allowed solutions.

Final solution is then in the form of:
    \begin{equation}
        u(z,t) = \frac{\left(u_*+\Delta u_*\right)^2}{\nu}z +\frac{8h}{\pi^2}\left[\frac{2u_*\Delta u_* +\left(\Delta u_*\right)^2}{\nu}\right]\sum_{n=1}^\infty\frac{\left(-1\right)^n}{\left(2n-1\right)^2}\sin\left(\lambda z\right)e^{-\lambda^2\nu t},
    \end{equation}
by summing steady state and transient state.

% -----------------------------------------------------------------------------
% Runge-Kutta
% -----------------------------------------------------------------------------

\chapter{Runge-Kutta}
\label{sec:RK}
    The main advantage here is the precision in calculations than in Euler method. Runge-Kutta method uses the explicit way to calculate the first approximation order and implicit way to correct the higher orders. If the differential function is independent of the function, than Runge-Kutta is the $1/6$ Simpson integration rule. The method consist in calculate the differential equation as:
    \begin{equation}
        y_{n+1} = y_n + \frac{k_{na}+2k_{nb}+2k_{nc}+k_{nd}}{6}\Delta x,
        \label{equ:RK}
    \end{equation}
where
    \begin{subequations}
        \begin{align}
            k_{na} &= f(x_n, y_n),\\
            k_{nb} &= f\left(x_n +\frac{\Delta x}{2}, y_n+\frac{\Delta x k_{na}}{2}\right),\\
            k_{nc} &= f\left(x_n +\frac{\Delta x}{2}, y_n+\frac{\Delta x k_{nb}}{2}\right),\\
            k_{nd} &= f(x_n +\Delta x, y_n +\Delta x k_{nc}),
        \end{align}
    \end{subequations}
and the global error associated with the calculation is in the order of $(\Delta x)^4$.

    Also, we can extend the method to solve first-order ODEs system. The system must be differentiated at the same variable, and for each sub-step, we repeat the process of calculation for each first-order equation. This is useful when we use high-order ODE, cause we always can rewrite high-order ODE into $n$ first-order ODEs. An example of a 2 equation system:
    \begin{subequations}
        \begin{align}
            \frac{\mathrm{d}y}{\mathrm{d}x} = f(x,y,z),\\
            \frac{\mathrm{d}z}{\mathrm{d}x} = g(x,y,z),
        \end{align}
    \end{subequations}
with discretization onto variable $x$, $y$ and $z$ becomes
    \begin{subequations}
        \begin{align}
            y_{n+1} = y_n + \frac{k_{na}+2k_{nb}+2k_{nc}+k_{nd}}{6}\Delta x,\\
            z_{n+1} = z_n + \frac{l_{na}+2l_{nb}+2l_{nc}+l_{nd}}{6}\Delta x,
        \end{align}
    \end{subequations}
where
    \begin{subequations}
        \begin{align}
            k_{na} &= f(x_n, y_n, z_n),\\
            l_{na} &= g(x_n, y_n, z_n),\\
            k_{nb} &= f\left(x_n +\frac{\Delta x}{2}, y_n+\frac{\Delta x k_{na}}{2}, z_n+\frac{\Delta x k_{na}}{2}\right),\\
            l_{nb} &= g\left(x_n +\frac{\Delta x}{2}, y_n+\frac{\Delta x k_{na}}{2}, z_n+\frac{\Delta x l_{na}}{2}\right),\\
            k_{nc} &= f\left(x_n +\frac{\Delta x}{2}, y_n+\frac{\Delta x k_{nb}}{2}, z_n+\frac{\Delta x k_{nb}}{2}\right),\\
            l_{nc} &= g\left(x_n +\frac{\Delta x}{2}, y_n+\frac{\Delta x k_{nb}}{2}, z_n+\frac{\Delta x k_{nb}}{2}\right),\\
            k_{nd} &= f(x_n +\Delta x, y_n +\Delta x k_{nc}, z_n +\Delta x l_{nc}),\\
            l_{nd} &= g(x_n +\Delta x, y_n +\Delta x k_{nc}, z_n +\Delta x l_{nc}),
        \end{align}
    \end{subequations}
and the global error associated with the calculation is still in the order of $(\Delta x)^4$.

\end{anexosenv}
                % Anexos
\include{./03-elementos-pos-textuais/indice-remissivo}      % Índice Remissivo

\end{document}
