% -----------------------------------------------------------------------------
% Anexos
% -----------------------------------------------------------------------------

\begin{anexosenv}
\partanexos

% -----------------------------------------------------------------------------
% Solving the diffusion equation
% -----------------------------------------------------------------------------

\chapter{Solving the diffusion equation}
\label{chap:Calculations}
    In this part, I will show the calculations we did for previous parts in Chapter \ref{chap:CFD}, with some details and some tricks to handle with it, if it is necessary. In this sense, we can start with the linear diffusion equation (equations \ref{equ:diff0}), and its solution is presented at \cite{Boyce}, chapter 10.

    \begin{subequations}
        \begin{align}
            \rho \frac{\partial u}{\partial t} &= \frac{\partial \tau}{\partial z}, \\
            \tau &= \rho \nu \frac{\partial u}{\partial z}.
        \end{align}
        \label{equ:diff0}
    \end{subequations}

    To solve this system, we need to know two boundary conditions and the initial condition. As we have discussed in section \ref{sec:viscous_steady}, the boundary conditions gives us the initial and the final condition, once we are going to change between one known steady state to another known steady state with a difference of $\Delta u_*$ between them (equation \ref{equ:viscous_steady1} has all information to give us initial and final condition). To solve equations \ref{equ:diff0}, we apply a variable separation on the function by guessing the solution is in the form of:
    \begin{equation}
        u(t,z) = Z(z) T(t).
        \label{equ:diff_solution0}
    \end{equation}

    Rewriting equations \ref{equ:diff0} with guess \ref{equ:diff_solution0}, we get:
    \begin{equation}
        \frac{1}{T}\frac{\mathrm{d} T}{\mathrm{d} t} = \frac{\nu}{Z}\frac{\mathrm{d}^2 Z}{\mathrm{d} z^2} = -\lambda^2.
    \end{equation}

    As we know the steady state, and it must fulfill as solution of the equation, we will remove it from transient. Calling $s(z)=u(z,t \to \infty)$, the final steady state, and $w(z,t) = u(z,t)-s(z)$ the transient solution of the diffusion equation, we get:
    \begin{subequations}
        \begin{align}
            \frac{\partial u}{\partial t} = \frac{\partial}{\partial t}\left[s(z)+w(z,t)\right] & \implies \frac{\partial u}{\partial t} = \frac{\partial w}{\partial t} ,\\
            \frac{\partial^2 u}{\partial z^2} = \frac{\partial^2 }{\partial z^2}\left[s(z)+w(z,t)\right] & \implies \frac{\partial^2 u}{\partial z^2} = \frac{\partial^2 s}{\partial z^2} + \frac{\partial^2 w}{\partial z^2},
        \end{align}
    \end{subequations}
and since $s(z)=z (u_*+\Delta u_*)^2/\nu$, $\partial^2 s/\partial z^2 = 0$. Applying all condition, we get the following:
    \begin{subequations}
        \begin{align}
            w(z=0,t) = u(0,t)-s(0) & \implies w(z=0,t) = 0 & \text{ Bottom wall}, \\
            \frac{\partial w(z=h,t)}{\partial z} = \frac{\partial u(z=h,t)}{\partial z} -\frac{\partial s(z=h)}{\partial z} & \implies \frac{\partial w(z=h,t)}{\partial z} = 0 & \text{ Top shear}, \\
            w(z,t=0) = u(z,t=0) -s(z) & \implies w(z,t=0) = -z\frac{2u_*\Delta u_* +\left(\Delta u_*\right)^2}{\nu} & \text{ Initial}.
        \end{align}
    \end{subequations}

    The solutions for the transient regime are, considering that $w=w_t(t) w_z(z)$:
    \begin{subequations}
        \begin{align}
            \frac{\dot{w}_t}{\nu w_t} = -\lambda^2 & \implies w_t = Ae^{-\lambda^2 \nu t}, \\
            \frac{{w''}_z}{\nu w_z} = -\lambda^2 & \implies w_z = B_1 \sin \left(\lambda z\right) + B_2 \cos \left(\lambda z\right).
        \end{align}
    \end{subequations}

    Now we can see that solutions must have the form of an exponential decay according the transition between one steady state and another, and at same time, they should be written in modes of $\sin$ and $\cos$. To determine some missing coefficients, we need to apply boundary conditions to this solutions:
    \begin{subequations}
        \begin{align}
            w(z=0,t) = 0 & \implies B_2 = 0, \\
            \frac{\partial w(z=h,t)}{\partial z} = 0 & \implies \lambda = \pi \frac{2n-1}{2h},
        \end{align}
    \end{subequations}
and initial condition
    \begin{equation}
        w(z,t=0) = -z\frac{2u_*\Delta u_* +\left(\Delta u_*\right)^2}{\nu} \implies A.B_1 = \frac{2}{h}\int_0^h\left[-z\frac{2u_*\Delta u_* +\left(\Delta u_*\right)^2}{\nu}\right]\sin\left(\lambda z\right)\mathrm{d}z,
    \end{equation}
since we have rewritten initial condition to same base of solution, by a Fourier's series. Fourier modes appear in initial condition "caused" by boundary conditions in the nature of the equation. Only certain frequencies can appear in the solution, and they are given by this Fourier's frequencies of final solution:
    \begin{equation}
        \frac{2}{h}\int_0^h\left[-z\frac{2u_*\Delta u_* +\left(\Delta u_*\right)^2}{\nu}\right]\sin\left(\lambda z\right)\mathrm{d}z = 
        \frac{8h}{\pi^2}\left[\frac{2u_*\Delta u_* +\left(\Delta u_*\right)^2}{\nu}\right]\sum_{n=1}^\infty\frac{\left(-1\right)^n}{\left(2n-1\right)^2}\sin\left(\lambda z\right),
    \end{equation}
just another way to describe the initial condition in same base of allowed solutions.

Final solution is then in the form of:
    \begin{equation}
        u(z,t) = \frac{\left(u_*+\Delta u_*\right)^2}{\nu}z +\frac{8h}{\pi^2}\left[\frac{2u_*\Delta u_* +\left(\Delta u_*\right)^2}{\nu}\right]\sum_{n=1}^\infty\frac{\left(-1\right)^n}{\left(2n-1\right)^2}\sin\left(\lambda z\right)e^{-\lambda^2\nu t},
    \end{equation}
by summing steady state and transient state.

% -----------------------------------------------------------------------------
% Runge-Kutta
% -----------------------------------------------------------------------------

\chapter{Runge-Kutta}
\label{sec:RK}
    The main advantage here is the precision in calculations than in Euler method. Runge-Kutta method uses the explicit way to calculate the first approximation order and implicit way to correct the higher orders. If the differential function is independent of the function, than Runge-Kutta is the $1/6$ Simpson integration rule. The method consist in calculate the differential equation as:
    \begin{equation}
        y_{n+1} = y_n + \frac{k_{na}+2k_{nb}+2k_{nc}+k_{nd}}{6}\Delta x,
        \label{equ:RK}
    \end{equation}
where
    \begin{subequations}
        \begin{align}
            k_{na} &= f(x_n, y_n),\\
            k_{nb} &= f\left(x_n +\frac{\Delta x}{2}, y_n+\frac{\Delta x k_{na}}{2}\right),\\
            k_{nc} &= f\left(x_n +\frac{\Delta x}{2}, y_n+\frac{\Delta x k_{nb}}{2}\right),\\
            k_{nd} &= f(x_n +\Delta x, y_n +\Delta x k_{nc}),
        \end{align}
    \end{subequations}
and the global error associated with the calculation is in the order of $(\Delta x)^4$.

    Also, we can extend the method to solve first-order ODEs system. The system must be differentiated at the same variable, and for each sub-step, we repeat the process of calculation for each first-order equation. This is useful when we use high-order ODE, cause we always can rewrite high-order ODE into $n$ first-order ODEs. An example of a 2 equation system:
    \begin{subequations}
        \begin{align}
            \frac{\mathrm{d}y}{\mathrm{d}x} = f(x,y,z),\\
            \frac{\mathrm{d}z}{\mathrm{d}x} = g(x,y,z),
        \end{align}
    \end{subequations}
with discretization onto variable $x$, $y$ and $z$ becomes
    \begin{subequations}
        \begin{align}
            y_{n+1} = y_n + \frac{k_{na}+2k_{nb}+2k_{nc}+k_{nd}}{6}\Delta x,\\
            z_{n+1} = z_n + \frac{l_{na}+2l_{nb}+2l_{nc}+l_{nd}}{6}\Delta x,
        \end{align}
    \end{subequations}
where
    \begin{subequations}
        \begin{align}
            k_{na} &= f(x_n, y_n, z_n),\\
            l_{na} &= g(x_n, y_n, z_n),\\
            k_{nb} &= f\left(x_n +\frac{\Delta x}{2}, y_n+\frac{\Delta x k_{na}}{2}, z_n+\frac{\Delta x k_{na}}{2}\right),\\
            l_{nb} &= g\left(x_n +\frac{\Delta x}{2}, y_n+\frac{\Delta x k_{na}}{2}, z_n+\frac{\Delta x l_{na}}{2}\right),\\
            k_{nc} &= f\left(x_n +\frac{\Delta x}{2}, y_n+\frac{\Delta x k_{nb}}{2}, z_n+\frac{\Delta x k_{nb}}{2}\right),\\
            l_{nc} &= g\left(x_n +\frac{\Delta x}{2}, y_n+\frac{\Delta x k_{nb}}{2}, z_n+\frac{\Delta x k_{nb}}{2}\right),\\
            k_{nd} &= f(x_n +\Delta x, y_n +\Delta x k_{nc}, z_n +\Delta x l_{nc}),\\
            l_{nd} &= g(x_n +\Delta x, y_n +\Delta x k_{nc}, z_n +\Delta x l_{nc}),
        \end{align}
    \end{subequations}
and the global error associated with the calculation is still in the order of $(\Delta x)^4$.

\end{anexosenv}
