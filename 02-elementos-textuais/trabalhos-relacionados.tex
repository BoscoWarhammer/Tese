% -----------------------------------------------------------------------------
% Trabalhos Relacionados
% -----------------------------------------------------------------------------

\chapter{Materiais Granulares}
\label{chap:Trabalhos-Relacionados}

%Cada capítulo deve conter uma pequena introdução (tipicamente, um ou dois parágrafos), em seção não numerada, que deve deixar claro o objetivo e o que será discutido no capítulo, bem como a organização do capítulo.

    Materiais granulares são conjunto de corpos sólidos, que individualmente podem ser compostos de um mesmo material ou de diferentes materiais, de geometria das mais variadas, podendo ter várias densidades, coeficiente de atrito, dureza, e várias outras propriedades físicas que os materiais possuem, mas todos são maiores que $100\mu m$, e portanto visíveis a olho nu \cite{Sands_Powders_and_Grains}. Interagem entre si quando estão em contato uns com os outros, perdendo energia, tanto na forma de dissipação inelástica, quanto no atrito entre os grãos.
    Os corpos sólidos que constituem os materiais granulares são grandes o suficiente para não apresentarem influência cinemática em função da temperatura termodinâmica. Assim sendo, movimentos Brownianos são ausentes nesse tipo de sistema.

    Segundo a \href{http://webofknowledge.com}{\textit{Web of Science}}, o número de produções publicadas com a palavra chave \textit{"Granular Materials"}, até $03/04/2018$, é de $8.618$ e segue a distribuição apresentada na figura \ref{fig:articles-year} ao longo dos anos. O estudo de materiais granulares também segue uma tendência crescente ao longo dos anos.
    
\begin{figure}
    \centering
    \includegraphics[width=0.8\textwidth]{04-figuras/articles-year.png}
    \caption{Produção científica acerca de materiais granulares com as palavras chave \textit{"Granular Materials"} ao longo dos anos.}
    \label{fig:articles-year}
\end{figure}

\section{Teoria}

    Como exemplos de materiais granulares temos areia, pedras, solos, fármacos, minérios, alimentos em grão como arroz, milho e soja, e até mesmo o cinturão de asteroides e os anéis de Saturno. Só a areia, compõe $10\%$ dos materiais da superfície do planeta Terra. Além disso, estima-se que o segundo material mais utilizado nas indústrias são materiais granulares, utilizando aproximadamente $10\%$ de toda a energia do planeta, sendo que o material mais utilizado é a própria água \cite{Sands_Powders_and_Grains}.

    Pela ausência de movimentos Brownianos, bem como pela dissipação de energia, sistemas granulares não sofrem relaxação espontânea de suas configurações estáveis na ausência de perturbações externas, principalmente na forma de vibrações, e portanto não apresentam ergodicidade\footnote{Um sistema ergódico tem característica de transitar entre seus microestados de energia espontaneamente, em intervalos de tempos, implicando que seus estados são todos equiprováveis quando analisados em um longuíssimo tempo \cite{Dissertacao, Srdjan-Tese, Granular_Solids_Liquids_and_Gases}.}.

    Materiais granulares apresentam também particularidades quanto às suas fases. Apresentam-se individualmente em corpos sólidos, e quando o conglomerado está próximo do repouso, constituem a fase sólida do sistema. Porém, se o sistema é agitado, ou configurado além de um limiar crítico do ângulo de repouso, pode apresentar-se nas fases de granular líquido\footnote{Granulares líquidos podem possuir uma camada limite que flui sobre a camada sólida do sistema.} ou granular gasoso. Tal classificação ainda está em aberto na literatura, apesar de existirem proposições para o que seria a temperatura granular do sistema \cite{Granular_Solids_Liquids_and_Gases}.

    Uma diferenciação entre sistemas granulares pode ser resultado direto das forças de interações entre os grãos. São chamados de granulares secos os sistemas que possuem apenas interações repulsivas, enquanto os granulares molhados apresentam forças de van der Waals nas interações grão a grão. Nesta tese, consideraremos apenas as interações repulsivas de contato, apesar de que em alguns casos, existe fluido envolvendo o material. Consideramos que todo o material que está envolvido pelo mesmo fluido não sofre forças de atração entre os mesmos grãos, e portanto, não está inclusa força de van der Waals na interação entre os grãos.

    Ainda neste capítulo, concentraremos em apenas alguns fenômenos de materiais granulares que não são o objetivo direto desta tese, mas que contextualizam o leitor acerca de fenômenos importantes observados em materiais granulares. Em especial, o capítulo \ref{chap:BNE} retrata do efeito castanha do Pará \textit{(BNE)} com os detalhes fenomenológicos e suas caracterizações, enquanto o capítulo \ref{chap:Transporte-Sedimentos} descreve os modos do transporte de sedimentos.

\section{Fenomenologia de Materiais Granulares}
\label{subchap:Fenomenologia}

%Pilha de areia
    Talvez a primeira ideia sobre materiais granulares remeta ao empilhamento de areira. Nesse caso, tem-se uma pilha estática de areia\footnote{A pilha estática está no estado sólido da fase granular \cite{Granular_Solids_Liquids_and_Gases}.}, amontoada sobre uma superfície. Note que em uma pilha como essa, os grãos sempre atingem uma determinada altura, e quando coloca-se mais material sobre a pilha, em algum momento, as camadas superiores da pilha escorrem até a base. Sempre que a pilha ultrapassar o ângulo crítico de repouso \cite{Granular_Physics}, ocorrerá uma avalanche, restaurando o sistema a um outro ângulo característico. Esta propriedade é a de auto-organização\footnote{Um sistema que não possui controlador central, regido por vários agentes que interagem entre si, com regras conhecidas na interação dos agentes e exibem propriedade não prevista pelas interações entre os agentes caracterizam um Sistema Complexo. Uma propriedade característica de Sistemas Complexos e de materiais granulares é a auto-organização. Alguns autores \cite{Mixing_and_Segregation_of_Granular_Materials, Measuring_the_flowing_properties_of_powders_and_grains, Revisiting_localized_deformation_in_sand_with_complex_systems, Granular_matter_and_networks, Patterns_and_collective_behavior_in_granular_media} classificam materiais granulares dentro da área de estudo de Sistemas Complexos.} da pilha de areia pelo ângulo crítico de repouso. Uma boa aproximação do ângulo de repouso é dada pela equação \ref{equ:atrito}:
\begin{equation}
    \label{equ:atrito}
    tan(\theta) = \mu _s ,
\end{equation}
onde $\theta$ é o ângulo crítico de repouso, e $\mu _s$ é o coeficiente de atrito do material.

%Biestabilidade da pilha
    Um pouco mais curioso ainda sobre as pilhas de grãos é que o histórico de preparação do sistema reflete-se no ângulo de repouso \cite{Dynamics_at_the_angle_of_repose}. Este histórico de preparação permite o sistema configurar-se diferentemente, e portanto, o ângulo de repouso assume valores diferentes utilizando o mesmo material. Existe então um ângulo de repouso mínimo $\theta _r$, e um ângulo máximo $\theta _m$, em que o empilhamento pode configurar-se: $\theta _r < \theta < \theta _m$ \cite{Granular_Physics}. Ter uma faixa de ângulos estáveis entre o ângulo mínimo de repouso e o ângulo máximo recebe o nome de biestabilidade do ângulo de repouso.

%Diferentes preparações, diferentes pressões
    Uma evidência de que o histórico de preparação altera a configuração do material é descrita nos artigos "\textit{Sensitivity of Stress Response Function to Packing Preparation}" e "\textit{Memories in sand: Experimental tests of construction history on stress distributions under sandpiles}" \cite{Sensitivity_of_Stress_Response_Function_to_Packing_Preparation, Memories_in_Sand}. A base circular apresentada na figura \ref{fig:pile_stress} foi feita com duas formas de deposição diferentes. No experimento, a pressão medida na base varia de acordo com a deposição, sendo que a deposição feita a partir do funil possui um pico de máximo de pressão em torno de $0,25$ e $0,5$ do raio da mesa, enquanto na deposição feita a partir da peneira apresenta pressão como espécie de platô entre o centro da mesa e $0,25$ do raio, com o máximo próximo do centro.

\begin{figure}
    \centering
    \begin{minipage}{.45\linewidth}
        \centering
        \includegraphics[width=0.9\textwidth]{04-figuras/Sand_Pile_GG_Experiment.png}
        \subcaption{Empilhamento a partir do funil.}
        \label{fig:pressure_pile:GG}
    \end{minipage}
    \begin{minipage}{.45\linewidth}
        \centering
        \includegraphics[width=0.9\textwidth]{04-figuras/Sand_Pile_GG_Pressure.png}
        \subcaption{Pressão na montagem a partir do funil.}
        \label{fig:pressure_response:GG}
    \end{minipage}
    \begin{minipage}{.45\linewidth}
        \centering
        \includegraphics[width=0.9\textwidth]{04-figuras/Sand_Pile_RL_Experiment.png}
        \subcaption{Empilhamento a partir da peneira.}
        \label{fig:pressure_pile:RL}
    \end{minipage}
    \begin{minipage}{.45\linewidth}
        \centering
        \includegraphics[width=0.9\textwidth]{04-figuras/Sand_Pile_RL_Pressure.png}
        \subcaption{Pressão na montagem a partir da peneira.}
        \label{fig:pressure_response:RL}
    \end{minipage}
    \caption{A preparação das pilhas de areia reflete nas pressões medidas na base da pilha. Nas figuras \ref{fig:pressure_pile:GG} e \ref{fig:pressure_response:GG} a deposição a partir do funil cria um perfil de pressões que tem o pico fora do centro da pilha, enquanto nas figuras \ref{fig:pressure_pile:RL} e \ref{fig:pressure_response:RL} a deposição a partir da peneira cria um perfil de pressões que tem um platô e depois decai. Figuras retiradas de \cite{Memories_in_Sand}.}
    \label{fig:pile_stress}
\end{figure}    

%Função resposta em diferentes configurações
    Um estudo feito por Atman \textit{et al.} \cite{Sensitivity_of_Stress_Response_Function_to_Packing_Preparation} mostra que diferentes geometrias de materiais granulares resultam em diferentes funções respostas\footnote{Função resposta é a diferença entre duas configurações, uma antes de aplicar-se a carga de teste e após a aplicação da carga, mostrando-se a distribuição de forças sobre o material, ou a compressão do sistema \cite{The_Physics_of_Granular_Media}.}. Como exemplo, a figura \ref{fig:stress_response} mostra duas funções respostas para sistemas com geometria circular e pentagonal.

\begin{figure}
    \centering
    \begin{minipage}{.45\linewidth}
        \centering
        \includegraphics[width=0.9\textwidth]{04-figuras/Funcao_Resposta1.png}
        \subcaption{Grãos circulares.}
        \label{fig:stress_response:circle}
    \end{minipage}
    \begin{minipage}{.45\linewidth}
        \centering
        \includegraphics[width=0.9\textwidth]{04-figuras/Funcao_Resposta2.png}
        \subcaption{Grãos pentagonais.}
        \label{fig:stress_response:pentagon}
    \end{minipage}
    \caption{A aplicação de uma força em diferentes sistemas granulares resulta em diferentes funções respostas. A diferença entre estes sistemas é que a figura \ref{fig:stress_response:circle} possui grãos de geometria circular e possui maior ordem, enquanto a figura \ref{fig:stress_response:pentagon} possui geometria pentagonal e maior desordem. Figuras retiradas de \cite{Sensitivity_of_Stress_Response_Function_to_Packing_Preparation}.}
    \label{fig:stress_response}
\end{figure}

%Cadeias de forças em diferentes pilhas
    Já que citamos as diferentes funções respostas nos empilhamentos de grãos, não podemos deixar de citar as cadeias de forças\footnote{Cadeias de forças consistem na rede de contatos entre os grãos que possuem força acima da força média do sistema. Em geral, mede-se as cadeias de forças são medidas a partir da função resposta \cite{The_Physics_of_Granular_Media}.}. A importância experimental da visualização das cadeias de forças se dá no entendimento da distribuição das forças internas que sustentam o material. Como exemplo, a figura \ref{fig:force_chain} indicia a cadeia de forças associada à função resposta de uma força puntual aplicada sobre o topo do material granular.

\begin{figure}
    \centering
    \includegraphics[width=0.5\textwidth]{04-figuras/Cadeia_Forca.png}
    \caption{A aplicação de uma força puntual no topo do material resulta na cadeia de forças, que pode ser vista através da função resposta do sistema. Neste caso, o sistema é preparado com grãos fotoelásticos em uma distribuição bidimensional. Quanto mais escuras, maiores são as tensões no material. Figura retirada de \cite{Sensitivity_of_Stress_Response_Function_to_Packing_Preparation}.}
    \label{fig:force_chain}
\end{figure}

%Formação de arcos
    As cadeias de forças são importantes para entender o fenômeno que está presente nos arcos de sustentação que utilizamos. Arcos são estruturas coletivas que possuem sustentação mútua, e, consequentemente, uma cadeia de forças ligando toda a estrutura, sendo capaz de sustentar o peso próprio e de todos os grãos acima, impedindo que os mesmos escoem. Na formação de arcos podem ocorrer efeitos de segregação, como verificado por Magalhães, C. e Magalhães, F. \cite{Caio-Tese, Felipe-Tese}.

%Pressão e tensão
    As medidas em materiais granulares geralmente tentam caracterizar o material em duas escalas diferentes que se relacionam: microescala, que diz respeito das medidas na escala dos grãos, como atrito; e macroescala, que diz respeito das medidas na escala do sistema, como pressão e tensão de cisalhamento.

%Dilatação
    Uma curiosidade sobre os materiais granulares é que quando estão submetidos a uma pressão e seu coeficiente de compactação\footnote{\label{foot:packingfraction}O coeficiente de compactação é dado pela razão da soma dos volumes individuais dos grãos pelo volume de ocupação no espaço.} está próximo do engarrafamento \cite{Non-Gaussian_behavior_in_jamming_unjamming_transition_in_dense_granular_materials}, uma dilatação tende a ocorrer, expandindo-se pelas bordas das fronteiras que confinam o material ou pelas outras direções de liberdade que o confinamento apresenta. Muitas vezes, ao aplicar-se uma pressão no material confinado, o coeficiente de compactação final pode ser menor que o inicial, indicando uma expansão volumétrica do sistema \cite{Felipe-Tese}.

%Escoamento de granular
    Aproveitando o exemplo do empilhamento, observa-se que na formação da pilha, após os grãos atingem o ângulo critico, ocasiona-se uma avalanche do material. Na avalanche, a camada superior entra em movimento, enquanto as camadas abaixo continuam estáticas. Na movimentação da camada superior, o material granular se apresenta no estado líquido, enquanto a camada estática abaixo encontra-se no estado sólido \cite{Granular_Solids_Liquids_and_Gases}. A figura \ref{fig:inclinacao} exemplifica a transição de fase sólido líquido entre as camadas.

\begin{figure}
    \centering
    \begin{minipage}{.45\linewidth}
        \centering
        \includegraphics[width=0.9\textwidth]{04-figuras/Pilha1.png}
        \subcaption{Pilha estática.}
        \label{fig:inclinacao:solido}
    \end{minipage}
    \begin{minipage}{.45\linewidth}
        \centering
        \includegraphics[width=0.9\textwidth]{04-figuras/Pilha2.png}
        \subcaption{Pilha escorrendo.}
        \label{fig:inclinacao:liquido}
    \end{minipage}
    \caption{Com o aumento do ângulo da pilha, nota-se que a camada superior desliza sobre a camada inferior (da figura \ref{fig:inclinacao:solido} para a figura \ref{fig:inclinacao:liquido}). Figuras retiradas de \cite{Granular_Solids_Liquids_and_Gases}.}
    \label{fig:inclinacao}
\end{figure}

    Escoamentos podem ocorrer então por tensões aplicadas no material, seja em uma inclinação da base, seja pela vibração do material. Como a mudança de configurações do material está relacionada a taxa de cisalhamento do material, mas a tensão de cisalhamento não é necessariamente proporcional a taxa de cisalhamento, este escoamento pode ser classificado como fluido não newtoniano.

%Segregação
%    Um fenômeno muito estudado é a segregação dos materiais granulares. O efeito de segregação ocorre em diferentes geometrias de material, densidades e coeficientes de atrito.

%Vibração
%    Vibrações no material granular fornecem energia ao sistema

    No próximo capítulo descreveremos as equações e os procedimentos para realizar as simulações dos materiais granulares, desde o modelo de contatos até a inserção do fluido no sistema.
