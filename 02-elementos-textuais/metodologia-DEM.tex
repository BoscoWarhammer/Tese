\chapter{Discrete Element Method - DEM}
\label{chap:DEM}
%    Uma forma muito utilizada para estudar sistemas granulares é a realização de simulações numéricas, que têm um papel importante na complementação das informações experimentais, o que aumenta ainda mais a compreensão dos fenômenos da física granular. Uma justificativa é o controle preciso dos parâmetros de entrada das simulações e do nível de complexidade acerca do objeto de estudo. Outra vantagem é a facilidade da extração de dados próxima da escala dos grãos, como as cadeias de forças, até a escala do sistema, como o cisalhamento do material, evidenciando eventuais propriedades emergentes e suas causas.

    Numerical simulations are widely used to study granular systems, which play an important role in complementing experimental information, which further enhances the understanding of granular physics phenomena. A justification to the use of numerical simulations is the precise control of the input parameters of the simulations and the level of complexity about the object of study. Another advantage is the ease extraction of data, from the grain scale, such as force chains, up to the system scale, such as material shear, showing possible emerging properties and their causes.

%    A técnica de simulação de materiais granulares que utilizamos neste trabalho é um \textit{DEM} conhecido na literatura como Dinâmica Molecular, ou \textit{Molecular Dynamics (MD)}. O método consiste em solucionar numericamente as equações de movimento quando aplicadas forças sobre os elementos a serem simulados. Uma vantagem deste método é que qualquer força que utilize parâmetros dentro da simulação e que possa ser descrita na interação com os elementos é aceita neste método.

    The technique to simulate of granular materials that we use in this work is a DEM known in the literature as Molecular Dynamics (MD). The method consists of numerically solving the Newton's laws of motion. An advantage of this method is that any force that can describe the interaction with the elements is accepted in this method.

%    A técnica descrita na referência \cite{Computer_Simulation_of_Liquids} utiliza-se dos formalismos da mecânica analítica através dos potenciais de interação entre os agentes, sejam potenciais lagrangianos, sejam potenciais hamiltonianos, para estabelecer a força que atua sobre cada agente. A desvantagem deste tipo de descrição é que forças dissipativas podem não aparecer, uma vez que a descrição das forças está relacionada diretamente com potenciais. Formalmente, o sistema deve obedecer ao conjunto de equações \ref{equ:lagrange} descritas pela função lagrangiana do sistema:
    The technique described in the reference \cite{Computer_Simulation_of_Liquids} uses the formalisms of analytical mechanics through the interaction potentials between agents, whether Lagrangian or Hamiltonian potentials, to establish the forces acting on each agent. The disadvantage of this type of description is that dissipative forces may not appear, since the description of forces is directly related to potentials. Formally, the system must obey the set of Equations \label{equ:lagrange} described by the Lagrangian function of the system: 
\begin{subequations}
    \label{equ:lagrange}
    \begin{empheq}[left={}]{align}
        \label{equ:lagrange1}
        \mathcal{L} = \mathcal{T} - \mathcal{V},\\
        \label{equ:lagrange2}
        \sum_{k} \left[\frac{d}{dt} \left( \frac{\partial \mathcal{L}}{\partial \dot{q_k}} \right) - \left(\frac{\partial \mathcal{L}}{\partial q_k} \right)\right] = 0,\\
        \label{equ:lagrange_forca}
        \vec{F}_{i} = \nabla \mathcal{L} = -\nabla \mathcal{V},
    \end{empheq}
\end{subequations}
%sendo que a notação descrita pelo conjunto de equações \ref{equ:lagrange}, $\mathcal{L}$ representa a função lagrangiana que rege a dinâmica do sistema, $\mathcal{T}$ a energia cinética, $\mathcal{V}$ a energia potencial, $k$ o número de coordenadas generalizadas do sistema, $q_{k}$ as coordenadas generalizadas, $\dot{q_{k}}$ as velocidades generalizadas, $\overrightarrow{F_{i}}$ a força exercida na partícula $i$ originada pelo gradiente do potencial $\mathcal{V}$.
where $\mathcal{L}$ represents the Lagrangian function that governs the dynamics of the system, $\mathcal{T}$ the kinetic energy, $\mathcal{V}$ the potential energy, $k$ the number of generalized coordinates of the system, $q_{k}$ the generalized coordinates, $\dot{q_{k}}$ the generalized velocities, $\vec{F}_{i}$ the force exerted on the particle $i$ originated by the gradient of the potential $\mathcal{V}$. 

%    Já outras referências \cite{Dissertacao, Abraao-Dissertacao, Caio-Dissertacao, Caio-Tese, Bouzid-Tese, Wassgren-Tese, Felipe-Tese, Srdjan-Tese, Luding-Tese, Computational_Granular_Dynamics} utilizam o modelo diretamente das forças que agem sobre cada elemento.

    Other references \cite{Dissertacao, Abraao-Dissertacao, Caio-Tese, Srdjan-Tese, Felipe-Tese, Nathalia-Dissertacao, Leticia-Dissertacao, Fabiola-Dissertacao, Luding-Tese, Caio-Dissertacao, Bouzid-Tese, Wassgren-Tese, Computational_Granular_Dynamics} use the model directly from the acting forces about each element.

\section{Equations of motion}
%    Para a realização da simulação, o conjunto de equações \ref{equ:movimento} deve ser satisfeito, o que leva em consideração as leis de Newton. Assim, tem-se a informação dos estados dos agentes em função do tempo.
    To carry out the simulation, the set of Equations \ref{equ:movimento} must be satisfied, which takes into account Newton's laws of motion. Thus, there is information on the agents' states as a function of time. 
\begin{subequations}
    \label{equ:movimento}
    \begin{empheq}[left={Translational}\empheqlbrace]{align}
        \label{equ:posicao_linear}
        \vec{r}_{i}(t) &= \vec{r}_{i}(0) + \int_{0}^{t} \vec{v}_{i}(t) dt,\\
        \label{equ:velocidade_linear}
        \vec{v}_{i}(t) &= \vec{v}_{i}(0) + \int_{0}^{t} \vec{a}_{i}(t) dt,\\
        \label{equ:aceleracao_linear}
        \vec{a}_{i}(t) &= \sum_{j} \frac{\vec{F}_{i,j}(t)}{m_{i}},
    \end{empheq}
    \begin{empheq}[left={Rotational}\empheqlbrace]{align}
        \label{equ:posicao_angular}
        \theta^{k}_{i}(t) &= \theta^{k}_{i}(0) + \int_{0}^{t} \vec{\omega}^{k}_{i}(t) dt,\\
        \label{equ:velocidade_angular}
        \vec{\omega}^{k}_{i}(t) &= \vec{\omega}^{k}_{i}(0) + \int_{0}^{t} \vec{\alpha}^{k}_{i}(t) dt,\\
        \label{equ:aceleracao_angular}
        \vec{\alpha}^{k}_{i}(t) &= {I^{k}_{i}}^{-1} \sum_{j} \vec{\tau}^{k}_{i,j}(t),
    \end{empheq}
\end{subequations}
%em que $i$ é a i-ésima partícula do sistema, $\vec{r}_{i}(t)$ é o vetor de posição do centro de massa do corpo $i$ no instante de tempo $t$, $\vec{v}_{i}(t)$ ou $\vec{\dot{r}}_{i}(t)$ é o vetor de velocidade do centro de massa do corpo, $\vec{a}_{i}(t)$ ou $\vec{\dot{v}}_{i}(t)$ ou $\vec{\ddot{r}}_{i}(t)$ é o vetor de acelerações do centro de massa do corpo, $\vec{F}_{i,j}(t)$ é a componente da força que o centro de massa do corpo sofre por interagir com outro corpo ou campo $j$, $m_{i}$ é a massa do corpo, $\theta^{k}_{i}(t)$ é a base das coordenadas de rotação do corpo expressas na base $k$ do sistema, $\vec{\omega}^{k}_{i}(t)$ é o pseudovetor de velocidades angulares do corpo expressas na base $k$ do sistema, $\vec{\alpha}^{k}_{i}(t)$ é o pseudovetor de acelerações angulares do corpo, ${I^{k}_{i}}^{-1}$ é o inverso do tensor de inércia do corpo e $\vec{\tau}^{k}_{i,j}(t)$ é o vetor de torques que o corpo sofre por interagir com outro corpo ou campo. Lembrando que a relação entre o torque e a força que o causa pode ser descrita pela equação \ref{equ:torque}:
where $i$ is the i-th particle of the system, $\vec{r}_{i}(t)$ is the position vector of the center of mass of the body $i$ at the instant of time $t$, $\vec{v}_{i}(t)$ or $\vec{\dot{r}}_{i}(t)$ is the velocity vector of the center of mass of the body, $\vec{a}_{i}(t)$ or $\vec{\dot{v}}_{i}(t)$ or $\vec{\ddot{r}}_{i}(t)$ is the vector of accelerations of the center of mass of the body, $\vec{F}_{i,j}(t)$ is the component of the force that the center of mass of the body suffers from interacting with another body or field $j$, $m_{i}$ is the body mass, $\theta^{k}_{i}(t)$ is the basis of the body rotation coordinates expressed in the system's $k$ basis, $\vec{\omega}^{k}_{i}(t)$ is the pseudovector of angular velocities of the body expressed on the basis $k$ of the system, $\vec{\alpha}^{k}_{i}(t)$ is the pseudovector of angular accelerations of the body, ${I^{k}_{i}}^{-1}$ is the inverse of the inertia tensor of the body and $\vec{\tau}^{k}_{i,j}(t)$ is the torque vector that the body suffers from interacting with another body or field. Remembering that the relationship between torque and the force that causes it can be described by the Equation \ref{equ:torque}:
\begin{equation}
    \label{equ:torque}
    \vec{\tau}_{i,j}(t) = \vec{\chi}_{i,j}(t) \times \vec{F}_{i,j}(t),
\end{equation}
%sendo que o vetor $\vec{\tau}_{i,j}(t)$ é o produto vetorial entre o vetor $\vec{\chi}_{i,j}(t)$, que liga o centro de massa da partícula $i$ ao ponto de aplicação da força, e o vetor $\vec{F}_{i,j}(t)$, o vetor da força causada por interagir com outro corpo ou campo $j$. As equações \ref{equ:aceleracao_linear} e \ref{equ:aceleracao_angular} expressam a segunda lei de Newton.
where the vector $\vec{\tau}_{i,j}(t)$ is the cross product of the vector $\vec{\chi}_{i,j}(t)$, which connects the center of mass of the particle $i$ to the point of application of the force, and the vector $\vec{F}_{i,j}(t)$, the vector of the force caused by interacting with another body or field $j$. The Equations \ref{equ:aceleracao_linear} and \ref{equ:aceleracao_angular} express Newton's second law. 

%    A formulação descrita pelo conjunto de equações \ref{equ:movimento} abrange espaços em 1D, 2D e 3D, porém esta tese foca apenas na formulação de sistemas em 2D.

    The formulation described by the set of Equations \ref{equ:movimento} covers spaces in 1D, 2D and 3D, but this thesis focuses only on the formulation of 2D systems.

\subsection{Force model}
\label{subchap:Modelo_Forcas}
%    As forças presentes nos sistemas modelados nesta tese incluem as forças de contato entre os agentes, que pertencem ao modelo reológico dos grãos, as forças de interação entre grão e fluido e a força gravitacional.

    The forces present in the systems modeled in this Chapter include the contact forces between agents, which belong to the rheological model of grains and the gravitational force. The interaction forces between grain and fluid is described in Chapter \ref{chap:CFD}.

\subsubsection{Rheological model of grains}
\label{subsubchap:Reologia}
%    O modelo reológico dos agentes utilizado na simulação de \textit{MD} para materiais granulares foi proposto por Kelvin-Voigt \cite{Dissertacao}, com geometria circular pelos agentes. A reologia de Kelvin-Voigt modela a força de contato por uma mola e um amortecedor em paralelo na direção normal do contato, como exemplificado na figura \ref{fig:forcas}. A parcela da mola representa a contribuição elástica do material, relacionado com o módulo de Young, enquanto o amortecedor tem a função de dissipar a energia na colisão inelástica entre os grãos. Adicionalmente, um elemento parecido com uma mola é inserido na direção tangencial. Um modelo proposto em \cite{Caio-Tese} adiciona um elemento parecido com um amortecedor em paralelo à mola tangencial, modelando a resistência ao rolamento. Por causa da geometria circular, toda a variação de momento angular é causada pelo torque devido à força tangencial.

    The rheological model of the contact between grains we used to simulate granular materials was the rheological model proposed by Kelvin-Voigt \cite{Kelvin, Voigt}. Kelvin-Voigt rheology models the contact force between two grains by a spring and a damper in parallel in the normal direction of contact, as exemplified in Figure \ref{fig:forcas}. The spring portion represents the elastic contribution of the material, related to the Young's modulus, while the damper has the function of dissipate the energy in the inelastic collision between the grains. Additionally, a spring-like element is inserted in the tangential direction. A model proposed in \cite{Caio-Tese} adds a damper-like element in parallel to the tangential spring, modeling the rolling resistance. We chose to use circular geometry for the grains. Because of the circular geometry, all angular momentum variation is caused by torque due to tangential force.

\begin{figure}
    \begin{minipage}{.45\linewidth}
        \centering
        \includegraphics[width=0.9\textwidth]{04-figuras/Modelo_Forcas.png}
        \subcaption{Contact force model between agents. }
        \label{fig:forcas_modelo}
    \end{minipage}
    \begin{minipage}{.45\linewidth}
        \centering
        \includegraphics[width=0.9\textwidth]{04-figuras/Contato.tikz}
        \subcaption{Contact representation between grains.}
        \label{fig:forcas_contato}
    \end{minipage}
    \caption[Contact force model.]{Force model and representation between two circular grains. Figures taken from \cite{Dissertacao}.}
    \label{fig:forcas}
\end{figure}

%    Uma peculiaridade do DEM é a permissão de interpenetração entre os grãos. Portanto, neste modelo não há deformação no contato entre dois corpos. A interpenetração máxima que permitimos em nosso modelo é controlada pelo parâmetro de dureza do material e impomos penetração máxima de $0.5\%$ dos raios.

    A peculiarity of DEM is that it allows interpenetration between the grains. Therefore, in this model there is no deformation in the contact between two bodies. The maximum interpenetration we allow in our model is controlled by the material's hardness parameter and we impose a maximum penetration of $0.5\%$ of the radius.

%    Para encontrar o valor da interpenetração $\delta$, na geometria circular, a equação \ref{equ:interpenetracao}:
    To determine the value of the interpenetration, $\delta$, in circular geometry, the Equation \ref{equ:interpenetracao}:
\begin{equation}
    \label{equ:interpenetracao}
    \delta_{i,j}^{\perp} = \left(R_{i}+R_{j}-\left|\vec{r}_{j}-\vec{r}_{i}\right|\right)\mathcal{H}(R_{i}+R_{j}-\left|\vec{r}_{j}-\vec{r}_{i}\right|),
\end{equation}
%em que $\delta_{i,j}^{\perp}$ é o valor da interpenetração entre os grãos $i$ e $j$, $R_{i}$ é o raio do corpo $i$, $R_{j}$ do corpo $j$, $\vec{r}_{i}$ é o vetor de posição do corpo $i$, $\vec{r}_{j}$ é o vetor de posição do corpo $j$ e $\mathcal{H}$ é a função de degrau de Heaviside. Então, quando a distância entre os corpos for maior que a soma dos raios, os corpos não estarão em contato e a função degrau de Heaviside indica que a interpenetração entre os grãos é nula.
where $\delta_{i,j}^{\perp}$ is the value of the interpenetration between the grains $i$ and $j$, $R_{i}$ is the radius of the body $i$, $R_{j}$ is the radius of the body $j$, $\vec{r}_{i}$ is the position vector of the center of the body $i$, $\vec{r}_{j}$ is the position vector of the center of the body $j$ and $\mathcal{H}$ is the Heaviside step function. So, when the distance between the bodies is greater than the sum of the radii, the bodies will not be in contact and the Heaviside step function indicates that the interpenetration between the grains is null. If the distance between the bodies is lesser than the sum of the radii, the bodies will be in contact and the Heaviside step function indicates that there is interpenetration between grains, by its value being equals to one.

%    Com o contato entre os grãos, a consequência direta da interpenetração é o surgimento de uma força elástica repulsiva ao contato, e dependente da função de interpenetração $\delta^{\perp}$. A expressão da força pode ser calculada pela equação \ref{equ:forca_elastica}:
    With grains being in contact, the direct consequence of the interpenetration is the appearance of an elastic repulsive force, and the force depends on the interpenetration function $\delta^{\perp}$. The force expression can be calculated by the Equation \ref{equ:forca_elastica}: 
\begin{equation}
    \label{equ:forca_elastica}
    \vec{F}_{i,j}^{el} = -k_{n}\left(\delta_{i,j}^{\perp}\right)^{\frac{D}{2}}\hat{n}_{i,j},
\end{equation}
%em que $\vec{F}_{i,j}^{el}$ é a força elástica que o corpo $i$ reage ao entrar em contato com o corpo $j$, $k_{n}$ é a constante relacionada à elasticidade do material na direção do contato, $\delta_{i,j}^{\perp}$ é a interpenetração entre os corpos $i$ e $j$, $D$ é a dimensão do sistema (no caso, $D=2$) e $\hat{n}_{i,j}$ é a direção normal do contato \cite{Dissertacao, Caio-Tese, Landau}. Pode-se escrever um potencial para esta força elástica como: $\mathcal{V} = \frac{1}{2}k_{n}{\delta_{i,j}^{\perp}}^2$.
where $\vec{F}_{i,j}^{el}$ is the elastic force that the body $j$ causes to the body $i$ when they come in contact, $k_{n}$ is the constant related to the elasticity of the material in the direction of contact, $\delta_{i,j}^{\perp}$ is the interpenetration between the bodies $i$ and $j$, $D$ is the dimension of the system (in this case, $D=2$) and $\hat{n}_{i,j}$ is the normal direction of the contact \cite{Dissertacao, Caio-Tese, Landau}. One can write a potential for this elastic force as: $\mathcal{V} = \frac{1}{2}k_{n}\left({\delta_{i,j}^{\perp}}\right)^2$.

%    Associado à força elástica, também está presente a força de amortecimento. Por se tratar de uma força dissipativa, não se pode associar um potencial à força de amortecimento. A maior parcela da perda de energia dos materiais granulares está na colisão, e esta é a responsável. A equação \ref{equ:forca_amortecimento} descreve seu comportamento:
    Associated with the elastic force, the damping force is also present. As it is a dissipative force, a potential cannot be associated with the damping force. Most of the energy loss of granular materials is in collision. The Equation \ref{equ:forca_amortecimento} describes its behavior: 
\begin{equation}
    \label{equ:forca_amortecimento}
    \vec{F}_{i,j}^{am} = -\gamma \left(\vec{v}_{i,j}.\hat{n}_{i,j}\right)\hat{n}_{i,j},
\end{equation}
%em que $\vec{F}_{i,j}^{am}$ é a força de amortecimento que o corpo $i$ sente ao entrar em contato com o corpo $j$, $\gamma$ é a constante de amortecimento relacionada a inelasticidade da colisão, $\vec{v}_{i,j}$ é a velocidade relativa entre os centros de massa dos corpos $i$ e $j$ e $\hat{n}_{i,j}$ é a direção normal do contato \cite{Dissertacao, Caio-Tese, Computational_Granular_Dynamics}.
where $\vec{F}_{i,j}^{am}$ is the damping force that the body $j$ causes to the body $i$ when they come in contact, $\gamma$ is the damping constant related to the inelastic collision, $\vec{v}_{i,j}$ is the relative velocity between the centers of mass of bodies $i$ and $j$ and $\hat{n}_{i,j}$ is the normal contact direction \cite{Dissertacao, Caio-Tese, Computational_Granular_Dynamics}.

%    A constante de amortecimento está diretamente ligada ao coeficiente de restituição e pode ser utilizada equivalentemente através da transformação mostrada em \cite{Dissertacao}. Alguns autores utilizam o coeficiente de restituição nas simulações, como \cite{Computational_Granular_Dynamics, Luding-Tese, Srdjan-Tese}. Aqui, utilizaremos a constante de amortecimento.

    The damping constant is directly linked to the restitution coefficient and can be used equivalently through the transformation shown in the set of Equations \ref{equ:restituicao}. Some authors use the restitution coefficient in simulations, such as \cite{Srdjan-Tese, Luding-Tese, Computational_Granular_Dynamics}. In this thesis we will use the damping constant.

\begin{subequations}
    \label{equ:restituicao}
    \begin{empheq}[left={}]{align}
        \epsilon &= \exp\left(\frac{-\pi}{\sqrt{\frac{4k_{n}m}{\gamma}-1}}\right),\\
        \gamma &= \sqrt{\frac{4k_{n}m}{\left(\frac{\pi}{\ln\left(\epsilon\right)}\right)^2+1}},
    \end{empheq}
\end{subequations}
where $\epsilon$ is the restitution coefficient, $\gamma$ is the damping constant, $k_{n}$ is the constant related to the elasticity of the material in the direction of contact and $m$ is the reduced mass $m=\frac{m_{i}m_{j}}{m_{i}+m_{j}}$.

%    A força de atrito também está presente no modelo de simulação. Como as superfícies estão em contato, existirá uma força de atrito entre elas, se houver tendência de movimento uma em relação à outra. Em especial, devido à geometria circular, as forças de atrito agirão apenas na direção tangencial. A velocidade de deslocamento entre os pontos de contato dos corpos é dada pela equação \ref{equ:velocidade_relativa} a seguir:
    The friction force is also present in the simulation model. As the surfaces are in contact, there will be a frictional force between them if they tend to move each other. In particular, due to the circular geometry, friction forces will only act in the tangential direction. The relative velocity between the contact point of the bodies is given by the Equation \ref{equ:velocidade_relativa} below: 
\begin{equation}
    \label{equ:velocidade_relativa}
    \delta_{i,j}^{\parallel} = \vec{v}_{ij}.\hat{t}_{ij} - R_{i}\omega_{i} - R_{j}\omega_{j},
\end{equation}
%em que $\delta_{i,j}^{\parallel}$ é a velocidade relativa dos pontos de contato dos corpos $i$ e $j$, $\vec{v}_{ij}$ é a velocidade relativa dos centros de massa dos corpos $i$ e $j$, $\hat{t}_{ij}$ é o vetor tangencial às superfícies de contato dos corpos $i$ e $j$, $R_{i}$ é o raio do corpo $i$, $R_{j}$ é o raio do corpo $j$, $\omega_{i}$ é a velocidade angular do corpo $i$ e $\omega_{j}$ é a velocidade angular do corpo $j$.
where $\delta_{i,j}^{\parallel}$ is the relative velocity of the contact point of the bodies $i$ and $j$, $\vec{v}_{ij}$ is the relative velocity of the centers of mass of the bodies $i$ and $j$, $\hat{t}_{ij}$ is the tangential vector to the contact surfaces of the bodies $i$ and $j$, $R_{i}$ is the radius of the body $i$, $R_{j}$ is the radius of the body $j$, $\omega_{i}$ is the angular velocity of the body $i$ and $\omega_{j}$ is the angular velocity of the body $j$.

%    Para a força tangencial, é necessário saber o deslocamento relativo dos pontos de contato, como dada pela equação \ref{equ:velocidade_relativa}, aplicados no sistema de equações \ref{equ:forca_atrito}, que modela a força de atrito com saturação, e é dada por:
    For the tangential force, it is necessary to know the relative displacement of the contact point, as given by the equation \ref{equ:velocidade_relativa}, applied in the system of Equations \ref{equ:forca_atrito}, which models the friction force with saturation, and is given by:
\begin{equation}
    \label{equ:forca_atrito}
    \vec{F}_{i,j}^{at} = \left\{
    \begin{array}{l l l}
        \displaystyle -\int_{t_{0}}^{t_{f}} k_{t} \delta_{i,j}^{\parallel} \hat{t}_{ij}\, \mathrm{d} t, & \quad \textrm{if } k_{t} \left| \delta_{i,j}^{\parallel} \right| \leq \mu \left|\vec{F}_{i,j}^{n}\right| & \textrm{ (Static friction)} \\
        \displaystyle -\frac{\delta_{i,j}^{\parallel}}{\left|\delta_{i,j}^{\parallel}\right|} \mu \left| \vec{F}_{i,j}^{n} \right| \hat{t}_{ij}\, & \quad \textrm{if } k_{t} \left| \delta_{i,j}^{\parallel} \right| > \mu \left| \vec{F}_{i,j}^{n} \right| & \textrm{ (Kinetic friction)}
    \end{array}
    \right.\ ,
\end{equation}
%em que $\vec{F}_{i,j}^{at}$ é a força de atrito entre os corpos $i$ e $j$, $k_{t}$ é a constante elástica do material na direção tangencial, $\delta_{i,j}^{\parallel}$ é a velocidade relativa entre os pontos de contato dos corpos $i$ e $j$, $\hat{t}_{ij}$ é o vetor tangencial às superfícies de contato dos corpos $i$ e $j$, $\mu$ é o coeficiente de atrito entre as superfícies dos corpos $i$ e $j$ e $\vec{F}_{i,j}^{n} = \vec{F}_{i,j}^{el} +\vec{F}_{i,j}^{am}$ é a força normal às superfícies dos corpos $i$ e $j$.
where $\vec{F}_{i,j}^{at}$ is the friction force between the bodies $i$ and $j$, $k_{t}$ is the elastic constant of the material in the tangential direction, $\delta_{i,j}^{\parallel}$ is the relative velocity between the contact point of the bodies $i$ and $j$, $\hat{t}_{ij}$ is the tangential vector to contact surfaces of bodies $i$ and $j$, $\mu$ is the friction coefficient between the surfaces of bodies $i$ and $j$ and $\vec{F}_{i,j}^{n} = \vec{F}_{i,j}^{el} +\vec{F}_{i,j}^{am}$ is the force normal to the surfaces of bodies $i$ and $j$.

    We chose to model static and dynamic friction to be a single value for simplicity, presented the friction coefficient $\mu$. Figure \ref{fig:atrito} describe the Coulomb friction we are using in the simulations.

\begin{figure}
    \centering
    \includegraphics[width=0.5\textwidth]{04-figuras/Atrito.png}
    \caption[Friction.]{Friction versus relative velocity between contact. Figure taken from \cite{Caio-Tese}.}
    \label{fig:atrito}
\end{figure}

\subsubsection{The external force: Gravity}
\label{subsubchap:Gravidade}
%    Para este modelo, a influência gravitacional é aproximada por uma constante, já que a simulação não leva em conta que a influência da massa dos corpos é muito pequena, se comparada com a massa do planeta em que está situada a simulação, bem como a variação de altura do sistema simulado é muito pequena e está próximo à superfície do planeta, quando comparada com o raio do planeta. Por conveniência, normalizamos a gravidade como valor unitário.
    For this model, the gravitational influence is approximated by a constant, since the simulation does not take into account that the influence of the bodies' mass is very small, if compared to the mass of the planet where the simulation is located, as well as the variation of height of the simulated system is very small and is close to the surface of the planet when compared to the radius of the planet. For convenience, we normalize gravity as a unit value.

\subsection{Temporal discretization}
%    Para a simulação computacional dos corpos sólidos, as equações da cinemática devem ser reescritas como expansões da série de Taylor, interpolando o sistema de equações de velocidade pelo algoritmo conhecido como \textit{Velocity Verlet} \cite{Verlet, Computer_Simulation_of_Liquids}. As equações de movimento discretizadas no tempo, em função do passo de tempo $\Delta t$, tornam-se como no conjunto de equações \ref{equ:movimento_discreto}:
    For the computational simulation of solid bodies, the kinematics equations must be rewritten as Taylor series expansions, and we chose the interpolating the velocity equation system by the algorithm known as Velocity Verlet \cite{Verlet, Computer_Simulation_of_Liquids}. The equations of motion discretized in time, as a function of the time step $\Delta t$, become as in the set of equations \ref{equ:movimento_discreto}:
\begin{subequations}
    \label{equ:movimento_discreto}
    \begin{empheq}[left={Translacional}\empheqlbrace]{align}
        \label{equ:posicao_linear_discreta}
        {\vec{r}_{i}}^{\;n+1} &= {\vec{r}_{i}}^{\;n} + {\vec{v}_{i}}^{\;n} \Delta t + \frac{{\vec{a}_{i}}^{\;n}}{2} (\Delta t)^{2},\\
        \label{equ:velocidade_linear_discreta}
        {\vec{v}_{i}}^{\;n+1} &= {\vec{v}_{i}}^{\;n} + \frac{{\vec{a}_{i}}^{\;n}+{\vec{a}_{i}}^{\;n+1}}{2} \Delta t,\\
        \label{equ:aceleracao_linear_discreta}
        {\vec{a}_{i}}^{\;n+1} &= \frac{\sum_{j} {\vec{F}_{i,j}}^{\;n+1} + \sum {\vec{F}_{i,ext}}^{\;n+1}}{m_{i}},
    \end{empheq}
    \begin{empheq}[left={Rotacional}\empheqlbrace]{align}
        \label{equ:posicao_angular_discreta}
        {\theta_{i}}^{n+1} &= {\theta_{i}}^{n} + {\vec{\omega}_{i}}^{\;n} \Delta t + \frac{{\vec{\alpha}_{i}}^{\;n}}{2} (\Delta t)^{2},\\
        \label{equ:velocidade_angular_discreta}
        {\vec{\omega}_{i}}^{\;n+1} &= {\vec{\omega}_{i}}^{\;n} +\frac{{\vec{\alpha}_{i}}^{\;n}+{\vec{\alpha}_{i}}^{\;n+1}}{2} \Delta t,\\
        \label{equ:aceleracao_linear_discreta}
        {\vec{\alpha}_{i}}^{\;n+1} &= {I_{i}}^{-1} \sum_{j} {\vec{\tau}_{i,j}}^{\;n},
    \end{empheq}
\end{subequations}
%em que $i$ é o índice do corpo que movimenta, $j$ é o índice do corpo em contato com o corpo $i$, $n$ é o passo de tempo, $\vec{r}$ é a posição do corpo, $\vec{v}$ é a velocidade do corpo, $\vec{a}$ é a aceleração do corpo, $\Delta t$ é o tamanho do passo de tempo, $\vec{F}_{i,j}$ é a força de contato entre os corpos $i$ e $j$, $\vec{F}_{ext}$ são as forças externas, como gravidade e força que o fluido exerce sobre o corpo, $m$ é a massa do corpo, $\theta$ é a posição angular do corpo, $\vec{\omega}$ é a velocidade angular do corpo, $\vec{\alpha}$ é a aceleração angular do corpo, $\vec{\tau}$ é o torque sobre o corpo, $I$ é o momento de inércia do corpo.
where $i$ is the index of the moving body, $j$ is the index of the body in contact with the body $i$, $n$ is the time step, $\vec{r}$ is the position of the body, $\vec{v}$ is the velocity of the body, $\vec{a}$ is the acceleration of the body, $\Delta t$ is the size of the time step, $\vec{F}_{i,j}$ is the contact force between the bodies $i$ and $j$, $\vec{F}_{ext}$ are the external forces, such as gravity, $m$ is the mass of the body, $\theta$ is the angular position of the body, $\vec{\omega}$ is the angular velocity of the body, $\vec{\alpha}$ is the angular acceleration of the body, $\vec{\tau}$ is the torque on the body, $I$ is the moment of inertia of the body. 

%    O conjunto de equações \ref{equ:movimento_discreto} está escrito para o sistema 2D, uma vez que só há um grau de liberdade para a rotação, e consequentemente todas as equações passam a ser escritas em função de um único parâmetro. A aproximação da velocidade como a ponderação entre as acelerações no instante de tempo atual e futuro é a chave para a minimização da imprecisão gerada pela discretização \cite{Computer_Simulation_of_Liquids}.

    The set of Equations \ref{equ:movimento_discreto} is written for the 2D system, since there is only one degree of freedom for the rotation, and consequently all equations are written as a function of a single parameter. The velocity approximation as the weighting between the accelerations in the current and future instants of time is the key to the minimization of the imprecision generated by the discretization \cite{Computer_Simulation_of_Liquids}.

\section{Algorithm}
%    Além das equações que regem o sistema, uma série de procedimentos devem ser realizados para que a simulação possa ocorrer. Cada um destes passos são essenciais para que a simulação ocorra, e são dependentes uns dos outros. O algoritmo \ref{alg:MD} determina as rotinas para a execução da simulação. Utilizamos o \textit{Gear Predictor-Corrector} de 3ª ordem com o \textit{Velocity Verlet} para realizar as simulações \cite{Computer_Simulation_of_Liquids}.
    In addition to the equations that govern the system, a series of procedures must be carried out so that the simulation can take place. Each of these steps are essential for the simulation to take place, and are dependent on each other. The Algorithm \ref{alg:MD} determines the routines for executing the simulation. We use the 3rd order Gear Predictor-Corrector with the Velocity Verlet to perform the simulations \cite{Computer_Simulation_of_Liquids}.

\begin{algorithm}
    \SetKwInOut{Input}{Entrada}\SetKwInOut{Output}{Saída}
    \Input{configuração de dados inicial da simulação}
    \Output{resposta e medições de simulação ao longo do tempo}
    \While{não atingida a condição de parada da simulação}{
        \If{chegou a hora de listar os Vizinhos}{
            Determinar a lista de corpos Vizinhos\;
        }
        Preditor\;
        Detectar Contatos\;
        Cálculo de Forças\;
        Corretor\;
        \If{Possui Fluido}{
            Cálculo do Fluido\;
        }
    }
    \caption{Dadas as entradas do problema, como posições iniciais dos corpos, velocidades e acelerações, o algoritmo de Dinâmica Molecular monta uma lista de corpos que são vizinhos delimitados por uma certa região, então prediz a posição e a velocidade dos corpos no próximo instante de tempo, procura os contatos que foram formados com a predição, calcula as forças entre cada corpo em contato e inclui as forças externas, corrige as predições de velocidade e aceleração de cada corpo e calcula a dinâmica do fluido. Assim um passo de Dinâmica Molecular é construído. Retirado de \cite{Dissertacao}.}
    \label{alg:MD}
\end{algorithm}


%    As condições de parada do algoritmo dependem do objetivo da simulação. Alguns exemplos, como estabilidade de pilhas estáticas, flutuações de energia, quebra das cadeias de forças, velocidade média do sistema, número de passos de tempo, entre vários outros parâmetros medíveis dentro da simulação podem se tornar o critério de parada da simulação. Nesta tese, utilizamos como principal critério de parada o número de passos de tempo de simulação.

    The algorithm stopping conditions depend on the purpose of the simulation. Some examples, such as static pile stability, energy fluctuations, breaking of force chains, average system velocity, number of time steps, among several other measurable parameters within the simulation can become the stopping criteria of the simulation. In this thesis, we use the number of simulation time steps as the main stopping criterion. 

%    Faremos uma breve discussão a respeito de cada uma das rotinas do algoritmo \ref{alg:MD}. Para maiores detalhes, as referências \cite{Dissertacao, Computer_Simulation_of_Liquids, Computational_Granular_Dynamics} possuem maiores explanações sobre as rotinas, com exemplos e algoritmos.

    We will briefly discuss each of the routines Algorithm \ref{alg:MD}. For more details, the references \cite{Dissertacao, Computer_Simulation_of_Liquids, Computational_Granular_Dynamics} have further explanations about the routines, with examples and detailed algorithms. 

\subsection{Neighbors}
%    Apesar de não ser a forma mais simples do algoritmo de localização de vizinhos, esta é mais eficiente, e está descrita em \cite{Dissertacao}. Consiste criar a lista de todos os corpos que pertencem à uma certa região de possível interação. A criação da lista minimiza o número de comparações feitas durante a execução, o que proporciona o maior desempenho computacional. O artigo \textit{"Methods of parallel computation applied on granular simulations"}\cite{Methods_of_Parallel_Computation_Applied_on_Granular_Simulations} revela as diferenças entre alguns métodos da criação das listas de corpos que interagem entre si. Este artigo foi escrito durante a elaboração deste projeto de tese e está no apêndice \ref{chap:Artigo}. O algoritmo \ref{alg:vizinhos} refere-se a criação da lista dos corpos que tem a possibilidade de interação entre si.

    Although not the simplest form of the Neighbor-Finding algorithm, it is the most efficient, and is described in \cite{Dissertacao}. It consists of creating a list of all bodies that belong to a certain region of possible interaction. Creating the list minimizes the number of comparisons during execution, which provides the highest computational performance. The article "Methods of parallel computation applied on granular simulations" \cite{Methods_of_Parallel_Computation_Applied_on_Granular_Simulations} reveals the differences between some methods of creating lists of interacting bodies. This article was written during the preparation of this thesis project and is in the Appendix \ref{chap:Artigo}. The Algorithm \ref{alg:vizinhos} refers to the creation of a list of bodies that have the possibility of interacting with each other.

\begin{algorithm}
    \SetKwInOut{Input}{Input}\SetKwInOut{Output}{Output}
    \KwIn{body position}
    \KwOut{neighbor list}
    Divide the space in regions\;
    \ForEach{body}{
        Insert the body in the list of the region it belongs to\;
        Insert the body into adjacent lists of the region it belongs to\;
    }
    \caption[Neighbor list.]{Algorithm for creating the list of neighbors. Algorithm taken from \cite{Dissertacao}.}
    \label{alg:vizinhos}
\end{algorithm}


\begin{figure}
    \centering
    \includegraphics[width = 0.40\textwidth]{04-figuras/Vizinhos.tikz}
    \caption[Neighbor search.]{A busca realizada no algoritmo \ref{alg:vizinhos} ocorre entre os corpos com sua região de vizinhança imediatamente adjacente. Figura retirada de \cite{Dissertacao}.}
    \label{fig:vizinhos}
\end{figure}

%    A figura \ref{fig:vizinhos} mostra as regiões que o corpo marcado deve estar listado. Para maiores detalhes, as referências \cite{Dissertacao, Computer_Simulation_of_Liquids}.

    The figure \ref{fig:vizinhos} shows the regions that the marked body should be listed. For more details, see the references \cite{Dissertacao, Computer_Simulation_of_Liquids}. 

\subsection{Predictor}
%    A rotina de predição atualiza as posições e as velocidades dos corpos, permitindo que todas as forças sejam calculadas em função dos novos valores. No conjunto de equações \ref{equ:movimento_discreto}, equações que envolvem os termos com índice $n$ são atualizados nesta rotina. O algoritmo \ref{alg:preditor} mostra a estrutura da rotina de predição.

    The prediction routine updates the positions and velocities of the bodies, allowing all forces to be calculated based on the new values. In the set of Equations \ref{equ:movimento_discreto}, equations involving terms with index $n$ are updated in this routine. The \ref{alg:preditor} algorithm shows the structure of the prediction routine.

\begin{algorithm}
    \SetKwInOut{Input}{Entrada}\SetKwInOut{Output}{Saída}
    \Input{posições, velocidades, acelerações e o passo de tempo $\Delta t$}
    \Output{posições, parcela das velocidades}
    \ForAll{corpos}{
        Calcular as posições\;
        Prever as velocidades\;
    }
    \caption{Rotina de predição das variáveis de estado dos corpos. Retirado de \cite{Dissertacao}.}
    \label{alg:preditor}
\end{algorithm}


\subsection{Detect contacts}
%    A rotina de detecção de contatos utiliza da lista de vizinhos gerada pelo algoritmo \ref{alg:vizinhos} para checar se o par listado corpo/vizinho possuem interpenetração, descrita na equação \ref{equ:interpenetracao}, e então gera uma nova lista de corpos que se interpenetram para ser utilizadas no algoritmo \ref{alg:forcas}. O algoritmo \ref{alg:contatos} descreve esta operação.

    The contact detection routine uses the list of neighbors generated by the Algorithm \ref{alg:vizinhos} to check whether the listed body/neighbor pair has interpenetration, described in equation \ref{equ:interpenetracao}, and then generates a new list of bodies that interpenetrate each other to be used in the Algorithm \ref{alg:forcas}. The Algorithm \ref{alg:contatos} describes this operation.

\begin{algorithm}
    \SetKwInOut{Input}{Input}\SetKwInOut{Output}{Output}
    \Input{Neighbor list}
    \Output{Contact list}
    \ForAll{neighbor bodies}{
        Calculate the Interpenetration $\delta_{i,j}$ between bodies $i$ and $j$\;
        \If{$\delta_{i,j} > 0$}{
            Insert the pair $i$ and $j$ in the contact list\;
        }
    }
    \caption[Detect contacts.]{Detect contacts routine. Algorithm taken from \cite{Dissertacao}.}
    \label{alg:contatos}
\end{algorithm}


\subsection{Force calculation}
%    A rotina de calcular as forças utiliza da lista de contatos gerada pelo algoritmo \ref{alg:contatos} para calcular as forças de contato entre os corpos, como forças elásticas (equação \ref{equ:forca_elastica}), forças de amortecimento (equação \ref{equ:forca_amortecimento}) e forças de atrito (equação \ref{equ:forca_atrito}). Além das forças de contato, os corpos sofrem a força gravitacional e a interação com o fluido (equações \ref{equ:arquimedes} e \ref{equ:arraste}). O algoritmo \ref{alg:forcas} contém a execução do calculo das forças.

    The routine to calculate the forces uses the contact list generated by the Algorithm \ref{alg:contatos} to calculate the contact forces between the bodies, such as elastic forces (Equation \ref{equ:forca_elastica}), damping forces (Equation \ref{equ:forca_amortecimento}) and friction forces (Equation \ref{equ:forca_atrito}). In addition to contact forces, bodies are subjected to gravitational force. The Algorithm \ref{alg:forcas} contains the execution of the calculation of the forces. 

\begin{algorithm}
    \SetKwInOut{Input}{Input}\SetKwInOut{Output}{Output}
    \Input{positions, velocities and contact list}
    \Output{acting forces and torques in the bodies}
    \ForEach{body}{
        Apply gravity force\;
        \ForEach{body in the contact list}{
            Calculate the normal forces $\vec{N}$\;
            Calculate the rolling forces ${F}^{d}$\;
            \eIf{$|{F}^{d}| < \mu |\vec{N}|$}{
                $\vec{F}^{at} += \vec{F}^{d}\hat{t}$\;
            }{
                $\vec{F}^{at} += \mu \textrm{sign}(\vec{F}^{d}) N\hat{t}$\;
            }
            Calculate torque\;
        }
    }
%    \caption[Force calculation.]{Aqui são calculadas as resultantes das forças em cada corpo. A força $\vec{N}$ é a força normal, contribuição da força elástica $\vec{F}^{el}$ e força de amortecimento $\vec{F}^{am}$ (equações \ref{equ:forca_elastica} e \ref{equ:forca_amortecimento}), $F^{d}$ é a força de rolamento de um corpo sobre o outro, que deve ser comparado com a força de atrito estático máxima $\mu N$. Retirado e adaptado de \cite{Dissertacao}.}
    \caption[Force calculation]{In this routine, the resultant forces are calculated for each body. The force $\vec{N}$ is the normal force, contribution of the elastic force $\vec{F}^{el}$ and the damping force $\vec{F}^{am}$ (Equations \ref{equ:forca_elastica} and \ref{equ:forca_amortecimento}), $F^{d}$ is the rolling force of one body on the other, which must be compared with the maximum static friction force $\mu N$. Algorithm taken from \cite{Dissertacao}.}
    \label{alg:forcas}
\end{algorithm}


\subsection{Corrector}
%    A rotina de correção atualiza as velocidades e as acelerações dos corpos. As forças calculadas no cálculo de forças é utilizada aqui para realizar o \textit{Velocity Verlet} e determinar as acelerações do próximo passo de tempo. No conjunto de equações \ref{equ:movimento_discreto}, as equações que envolvem os termos com índice $n+1$ são atualizadas nesta rotina. O algoritmo \ref{alg:corretor} mostra a estrutura da rotina de correção.

    The correction routine updates the speeds and accelerations of the bodies. The forces calculated in the force calculation are used here to perform the Velocity Verlet and determine the velocities and accelerations for the next time step. In the set of Equations \ref{equ:movimento_discreto}, the equations involving terms with index $n+1$ are updated in this routine. The Algorithm \ref{alg:corretor} shows the structure of the correction routine.

\begin{algorithm}
    \SetKwInOut{Input}{Entrada}\SetKwInOut{Output}{Saída}
    \Input{resultante das forças e o passo de tempo $\Delta t$}
    \Output{estado dos corpos prontos para o próximo passo de tempo}
    \ForEach{corpo}{
        Calcular as acelerações\;
        Corrigir as velocidades\;
    }
    \caption{Rotina de correção das variáveis dos corpos. Retirado de \cite{Dissertacao}.}
    \label{alg:corretor}
\end{algorithm}


\section{Important parameters}
%    Em decorrência do modelo de forças apresentado, alguns parâmetros são importantes para as simulações. Por serem regidos por equações de diferenças em função do parâmetro temporal, alguns critérios devem ser obedecidos para que a simulação seja estável. Um dos parâmetros é a constante de tempo $\Delta t$, que em nossas simulações possui relação direta com com o período de oscilação do modelo massa mola, dado por $\Delta t = \zeta \sqrt{m_{min}/k}$, em que $\zeta$ é um valor de ajuste, $m_{min}$ é a menor massa do sistema e $k$ é a constante de elasticidade. Os fatores que estabilizam as simulações que possuem o modelo massa mola devem ter $\zeta < 1/10$ \cite{Dissertacao, Caio-Tese, Computer_Simulation_of_Liquids}. Nesta tese utilizaremos o fator de $1/10$ para sistemas que não são vibrados e $1/100$ para sistemas vibrados.

    Due to the presented force model, some parameters are important for the simulations. As they are governed by difference equations as a function of the temporal parameter, some criteria must be obeyed for the simulation to be stable. One of the parameters is the time constant $\Delta t$, which in our simulations has a direct relationship with the oscillation period of the spring mass model (Kelvin-Voigt rheology), given by $\Delta t = \zeta \sqrt{m_{min}/k_{n}}$, where $\zeta$ is an adjustment value, $m_{min}$ is the smallest mass of the system, and $k_{n}$ is the spring constant. The factors that stabilize the simulations, they must have $\zeta < 1/10$ \cite{Dissertacao, Caio-Thesis, Computer_Simulation_of_Liquids}. In this thesis we will use the factor of $1/10$ for non-vibrated systems and $1/100$ for vibrated systems. 

%    Outro importante parâmetro é o fator de amortecimento $\gamma$, ou o coeficiente de restituição $\epsilon$. Pela natureza dissipativa dos materiais granulares, utilizam-se $\epsilon \simeq 0$\footnote{A relação entre o coeficiente de restituição e o coeficiente de amortecimento podem ser encontrados em \cite{Dissertacao}.}, o que aproxima de $\gamma \simeq 2\sqrt{k {m}_{min}}$ pois teremos regimes críticos na equação massa mola quando utilizarmos a menor massa dos dois corpos, e para todos os outros, o amortecimento será subcrítico \cite{Bouzid-Tese, Luding-Tese}.

    Another important parameter is the damping factor $\gamma$, or the restitution coefficient $\epsilon$. Due to the dissipative nature of granular materials, $\epsilon \simeq 0$, which approximates $\gamma \simeq 2\sqrt{k_{n}{m}_{min}}$, because we will have critical regimes in the spring mass equation when we use the smallest mass of the two bodies, and for all others, the damping will be subcritical \cite{Bouzid-Thesis, Luding-Thesis}. 

    We model the walls for the BNE problem as smooth walls. The technique we use is to create a virtual body that has only the fixed component that it does a boundary, while it is free to move in the other direction, coinciding to be in the closest position of the free bodies.

%    No próximo capítulo descreveremos o efeito castanha do Pará (\textit{BNE}) e como montamos a simulação que leva a este efeito.
    In the next chapter we will describe the Brazil nut Effect (BNE) and how we set up the simulation that leads to this effect.
