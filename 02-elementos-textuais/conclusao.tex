% -----------------------------------------------------------------------------
% Conclusão
% -----------------------------------------------------------------------------

\chapter{Conclusões Parciais}
\label{chap:Conclusao}

    Para os as simulações do BNE conseguimos reproduzir as propriedades de ascessão do intruso com diferentes densidades, diferentes amplitudes de vibração e diferentes frequências de vibração, observando a importância do atrito nas paredes do sistema. Mais do que isso, conseguimos realizar o BNE em um sistema que possui condição periódica de contorno e suas diferenças para o sistema de caixa fechada.

    Para o sedimento de transportes, conseguimos validar as propriedades físicas que regem o sistema, condizendo simulação com a conservação de movimento. Validamos o fluido de acordo com a literatura e acoplamos grão e fluido de forma a interagirem sobre as leis da física.

%Procure fazer uma análise crítica de seu trabalho, destacando os principais resultados e as contribuições deste trabalho para a área de pesquisa.

%\section{Trabalhos Futuros}
%\label{sec:trabalhosFuturos}

%Também deve indicar, se possível e/ou conveniente, como este trabalho pode ser estendido ou aprimorado.

%\section{Considerações Finais}
%\label{sec:consideracoesFinais}

%As derradeiras palavras para encerramento do seu trabalho acadêmico.

% -----------------------------------------------------------------------------
% OBS: a norma ABNT estabelece que em qualquer tipo de trabalho acadêmico monográfico
% deve haver um capítulo de conclusão
% -----------------------------------------------------------------------------
