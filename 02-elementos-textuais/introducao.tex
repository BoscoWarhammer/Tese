% -----------------------------------------------------------------------------
% Introdução
% -----------------------------------------------------------------------------

\chapter{Introduction}
\label{chap:Introducao}
    Granular materials are present in various contexts of nature and in many human activities \cite{Granular_Media_Between_Fluid_and_Solid, Sands_Powders_and_Grains, The_Physics_of_Granular_Media, Granular_Physics, Micromechanics_of_Granular_Materials}. Economic activities, like agricultural production, mining and civil engineering, are essentially linked to the usage of granular materials \cite{Sands_Powders_and_Grains}. For many years, research in granular materials were linked manly to engineering \cite{Abraao-Dissertacao, Versuche_uber_Getreidedruck_in_Silozellen, Janssen}, in order to optimize production processes, storing, transportation, and structural applications to these materials. Nowadays, some areas of physics, such as statistical mechanics \cite{Unifying_Concepts_in_Granular_Media_and_Glasses}, study intensively the characterization and behavior of these materials, and as well applications, due to the richness of observed phenomena. Its ubiquity reflects the importance of studies about it to better understand their manipulation in the most diverse situations.

    Granular materials can be characterized as a cluster of bodies larger than a few hundred micrometers up to the size of asteroids \cite{Sands_Powders_and_Grains, The_Physics_of_Granular_Media}. In addition to the size, another feature of bodies is that they are individually in solid state. Their interactions result in energy dissipation, either by friction or by inelastic collisions interaction. They are not subject to movement caused by thermal fluctuations, and therefore, do not exhibit Brownian motion. More characterizations of granular materials can be found in the Chapter \ref{chap:Trabalhos-Relacionados} of this thesis.

    The aim of this work is to computationally simulate granular materials phenomena, using Discrete Element Method (DEM), specifically based on the Molecular Dynamics (MD) \cite{Computer_Simulation_of_Liquids}. The simulations are in 2D, with grains that have circular geometry, hard-core potential repulsion when in contact, and are under the action of gravity. To simulate the contacts, we also take into account the Coulomb friction between the grains. Once the properties of the materials are defined, such as hardness, friction, mass, position and radius, we apply Newton's laws of motion to perform the simulation. We detail these equations and peculiarities of the simulation of dry granular materials in the Chapter \ref{chap:DEM}.

    Among the phenomena presented by the dry granular materials, we were interested this thesis in the Brazil Nut Effect (BNE), related to the segregation of confined grains when submitted to the vibration, and in the presence of a gravitational field. Larger grains segregate to surface, while smaller grains sink to bottom. The Chapter \ref{chap:BNE} provides more details about the phenomenology of BNE, as well the numerical approach employed. The results and the discussion about BNE phenomena we studied are presented in Chapter \ref{chap:Resultados-BNE}.

    Associated with DEM, we are interested in simulate granular materials carried by fluids and characterize the transport mode of bedload, using Computer Fluid Dynamics (CFD). The model of the fluid we chose flows homogeneously in layers in one direction and but with different profiles layer by layer, making the simulation of the fluid been in quasi-1D. In this thesis we describe the technique to simulate 1D flow in viscous regime and how to link it with the granular phase in Chapter \ref{chap:CFD} and a turbulent steady-state approach in Appendix \ref{chap:Turbulence}.

    We also are interested in the phenomenon of sediment transport. The sediment transport if the transport of grains immersed in fluids. The Navier-Stokes \cite{Physical_Hydrodynamics, Fluid_Mechanics} equation is used in this work to model the fluid that flows and carries part of the granular materials with. There are some transport modes that are characterized by the way the grains are transported by the fluid, which are briefly described in the Chapter \ref{chap:Transporte-Sedimentos}. In this thesis, we focused in the bedload transportation mode, and the results of our work are presented in Chapter \ref{chap:Resultados-CFD}.

\section{Justification}
\label{sec:justificativa}
    Because we realize that there is still a lack of understanding in the phenomena involving granular materials, we propose to study in this thesis the techniques that can predict the behavior of the grains, conglomerated with low speed and in contact most of the time, or to characterize some emerging properties of the system, like resonance effect, that has not yet been reported or documented, as well as to use results already known from the materials, and try to apply them in other areas where measurement is difficult.

    The main importance of the numerical simulations is related the simplicity of data extraction. One can change the range of parameters without the need to rebuild the experimental apparatus, like repositioning sensors or rearranging the experimental setup. All the information in the simulations are at hand, since to simulate the system one needs to access the parameters that will fill the dynamics of the equations. Depending on the precision of the model used, very detailed information can be extracted if the model is precise, or lots of ensembles could give a better approach to averaging process, as if many different experiments were done.

    Specifically, we will deal with a subject that has not yet been reproduced in the literature: BNE in two dimensions with periodic boundary conditions. To analyze this problem we use computer simulations and the Large Deviation Function (LDF) \cite{Large_Deviations_in_Physics} technique.

    We also computed by simulations the saturation time scale\footnote{Saturation time in sediment transport indicates the characteristic time it takes for the material to enter a steady state, leaving a configuration and arriving at the final configuration.} and the saturation length scale\footnote{Saturation length in the sediment transport indicates the characteristic length it takes for the material to enter its normal flux regime, when there is a difference of concentration of the material according to the space.} of the material for the viscous bedload regime when transported by a fluid. Such measurement is quite difficult to achieve in experiments due to the scales we are treating.

\section{Motivation}
\label{sec:motivacao}
    In the context of engineering, it is necessary to understand how the processes are elaborated, in order to adjust them to optimize the production costs, transportation and storage of materials essential to human activities, such as food and ores. In this sense, the understanding of the behavior of these materials, when subjected to certain conditions, allows them to be manipulated in the way of greatest interest, whether due to the need to conserve the material, either due to faster transport or the efficiency of another parameter in which it is intended spend less resources or have the greatest financial, energy or social return.

    In grain segregation, problems related to clogging may occur depending on the geometry of the materials \cite{Caio-Tese}. When these materials are subjected to vibration, or when they flow, the larger ones are naturally separated of the smaller ones, thus facilitating the filtration, but making it difficult to mix. The recent experiments shows the behavior of the shape and size contribution in the agitated grains, known as BNE \cite{Size_segregation_of_irregular_granular_materials_captured_by_time-resolved_3D_imaging}. These sets of agglomerates can have negative consequences for industrial process, such as wear of silos related to the resistance of materials, biochemical transformation of the silo or granular material, rotting or ageing of stored food, etc. \cite{Silo_failures}.

    Thus, understanding how the granular materials interact and how they are transported, whether transported on a conveyor belt, or carried by the currents of a river, gives the possibility to control its possible effects, or to predict its consequences, such as the amount of residues remaining in in the rivers due to the collapse of the Fundão dams in Mariana-MG in November 2015 \cite{Mariana_en, Mariana_pt, Mariana_fr}, and the Brumadinho dam in January 2019 \cite{Brumadinho_en, Brumadinho_pt, Brumadinho_fr}.

\section{Workflow}
\label{sec:organizacaoTrabalho}
    This work is divided into a bibliographic review on granular materials, in Chapter \ref{chap:Trabalhos-Relacionados}, with phenomenology on the study of granular materials. The descriptions of the Chapter \ref{chap:DEM} relates to the equation of motion and modelling of the granular phase, concerning the interaction forces between grains and how the governing equations are implemented in this system. The Chapter \ref{chap:BNE} refers in particular to the phenomenon known as BNE and the proposition of this work in studying dry granular systems. In Chapter \ref{chap:Resultados-BNE} we present the first part of the results of this thesis. The Chapter \ref{chap:CFD} relates to the equation of motion and modelling the fluid phase, relating to the interaction between grains and fluid in different regimes. The Chapter \ref{chap:Transporte-Sedimentos} represents the characterizations of each transport mode, and the descriptions of the transport of materials by fluids in laminar regime. The Chapter \ref{chap:Resultados-CFD} contains discussions on the results obtained, referring to the transport of sediments. The Chapter \ref{chap:Conclusao} presents the conclusions of this thesis and perspectives for the continuation of this work.
