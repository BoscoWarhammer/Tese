% -----------------------------------------------------------------------------
% Introdução
% -----------------------------------------------------------------------------

\chapter{Introduction}
\label{chap:Introducao}

%    Materiais granulares estão presentes em vários contextos da natureza e das atividades humanas \cite{Sands_Powders_and_Grains, The_Physics_of_Granular_Media, Granular_Physics, Micromechanics_of_Granular_Materials, Granular_Media_Between_Fluid_and_Solid}. Atividades econômicas, como produção agrícola, mineração e tecnologia de construção, são essencialmente ligadas ao uso de materiais granulares \cite{Sands_Powders_and_Grains}. Por muitos anos, os estudos em materiais granulares estiveram presentes principalmente nas engenharias \cite{Versuche_uber_Getreidedruck_in_Silozellen, Janssen}, com o intuito de otimizar os processos de produção, armazenagem, escoamento e aplicações estruturais destes materiais. Hoje, algumas áreas da física, como a mecânica estatística \cite{Unifying_Concepts_in_Granular_Media_and_Glasses}, estudam intensamente a caracterização do comportamento destes materiais e suas aplicações, pela riqueza dos fenômenos observados. Sua ubiquidade reflete a importância dos estudos acerca de seu conhecimento, para que haja a manipulação destes elementos nas mais diversas situações.

    Granular materials are present in various contexts of nature and in many human activities \cite{Sands_Powders_and_Grains, The_Physics_of_Granular_Media, Granular_Physics, Micromechanics_of_Granular_Materials, Granular_Media_Between_Fluid_and_Solid}. Economic activities, like agricultural production, mining and building technology, are essentially linked to the usage of granular materials \cite{Sands_Powders_and_Grains}. For many years, research in granular materials were linked manly to engineering \cite{Versuche_uber_Getreidedruck_in_Silozellen, Janssen}, in order to optimize production process, storing, flowing, and structural applications to these materials. Nowadays, some areas of physics, such as statistical mechanics \cite{Unifying_Concepts_in_Granular_Media_and_Glasses}, study intensively the characterization and behavior of these materials, and their applications, due to the richness of observed phenomena. Its ubiquity reflects the importance of studies about its knowledge, so that there is manipulation of these elements in the most diverse situations.

%    Materiais granulares podem ser caracterizados como um aglomerado de corpos maiores que algumas centenas de micrometros até o tamanho de asteroides \cite{Sands_Powders_and_Grains, The_Physics_of_Granular_Media}. Além do tamanho, outra característica dos corpos é se apresentarem no estado sólido. Suas interações resultam em dissipação de energia, seja por atrito, seja pela inelasticidade da interação. Não estão sujeitos à variações no movimento causadas por flutuações térmicas, e portanto, não exibem movimentos Brownianos. Mais caracterizações dos materiais granulares podem ser encontradas no capítulo \ref{chap:Trabalhos-Relacionados} desta tese.

    Granular materials can be characterized as a cluster of bodies larger than a few hundred micrometers up to the size of asteroids \cite{Sands_Powders_and_Grains, The_Physics_of_Granular_Media}. In addition to the size, another feature of bodies is that they are individually in solid state. Their interactions result in energy dissipation, either by friction or by inelastic collision interaction. They are not subject to various movement caused by thermal fluctuations, and therefore, do not exhibit Brownian movements. More characterizations of granular materials can be found in the chapter \ref{chap:Trabalhos-Relacionados} of this thesis.

%    A proposta de estudos deste trabalho baseia-se na realização de simulações computacionais de materiais granulares, utilizando o Método de Elemento Discreto, ou \textit{Discrete Element Method} (DEM), baseado no método de Dinâmica Molecular, ou \textit{Molecular Dynamics} (MD) \cite{Computer_Simulation_of_Liquids}. As simulações estão em 2D, os grãos tem geometria circular, estão sob a ação da gravidade, e possuem potencial de repulsão quando estão em contato. No contato também levamos em conta o atrito entre as partículas. Definidas as propriedades dos materiais, como dureza, atrito, massa, posição e raio, aplicamos as leis de Newton para realizar a simulação. Detalhamos tal equacionamento e peculiaridades da simulação no capítulo \ref{chap:DEM}. Em relação ao fluido, a descrição detalhada do equacionamento, considerações do fluido no problema de transporte e da Fluidodinâmica Computacional, ou \textit{Computational Fluid Dynamics (CFD)}, pode ser encontrada também no capítulo \ref{chap:DEM}.

    The aim of this work is to computationally simulate granular materials, using Discrete Element Method (DEM), based on the Molecular Dynamics (MD) \cite{Computer_Simulation_of_Liquids}. The simulations are in 2D, with grains that have circular geometry, have potential repulsion when in contact, and are under the action of gravity. In contact, we also take into account the friction between the grains. Once the properties of the materials are defined, such as hardness, friction, mass, position and radius, we apply Newton's laws to perform the simulation. We detail these equations and peculiarities of the simulation in the chapter \ref{chap:DEM}. The detailed description of the fluid equation, and its considerations, can also be found in chapter \ref{chap:DEM}, likewise the transport law and Computational Fluid Dynamics (CFD).

%    Dos fenômenos apresentados pelos materiais granulares, estudamos nesta tese, o efeito castanha do Pará, ou \textit{Brazil Nut Effect} (BNE), relacionado à segregação de grãos confinados quando são submetidos à vibração e em presença de um campo gravitacional. Grãos maiores segregam-se no topo, enquanto grãos menores afundam. O capítulo \ref{chap:BNE} fornece mais detalhes a respeito do \textit{BNE}, tanto do ponto de vista fenomenológico, quanto das simulações propostas.

    From the phenomena presented by the dry granular materials, we researched in this thesis, the Brazil Nut Effect (BNE), related to the segregation of confined grains when they are submitted to the vibration and in the presence of a gravitational field. Larger grains segregate at the top, while smaller grains sink. The chapter \ref{chap:BNE} provides more details about BNE, both from a phenomenological point of view and from the proposed simulations.

%    Estudamos também o fenômeno do transporte de sedimentos imersos em fluidos. Com as equações que regem os fluidos, a equação de Navier-Stokes \cite{Physical_Hydrodynamics, Fluid_Mechanics} é utilizada neste trabalho para modelar o fluido que escoa e carrega consigo parte do material granular. Existem alguns modos de transporte que são caracterizados pela maneira como os grãos são trasportados pelo fluido, os quais estão descritos no capítulo \ref{chap:Transporte-Sedimentos}.

    We also researched the phenomenon of sediment transport immersed in fluids. With the equations that govern fluids, the Navier-Stokes \cite{Physical_Hydrodynamics, Fluid_Mechanics} equation is used in this work to model the fluid that flows and carries part of the granular materials with. There are some transport modes that are characterized by the way the grains are transported by the fluid, which are briefly described in the chapter \ref{chap:Transporte-Sedimentos}. In this thesis, we focused in the bedload transportation mode.

\section{Justification}
\label{sec:justificativa}

%    No contexto da engenharia, necessita-se compreender como os processos são elaborados, de forma a ajustá-los para otimizar os custos de produção, transporte e armazenamento de materiais vitais às atividades humanas, como alimentos e minérios. Neste sentido, o entendimento do comportamento dos materiais, quando submetidos a certas condições, permite manipulá-los da forma de maior interesse, seja por uma necessidade de conservação do material, seja pelo transporte mais rápido ou pela eficiência de outro parâmetro na qual se pretende gastar menos recursos ou ter o maior retorno financeiro, energético ou social.

    In the context of engineering, it is necessary to understand how the processes are elaborated, in order to adjust them to optimize the production costs, transportation and storage of materials vital to human activities, such as food and ores. In this sense, the understanding of behavior of materials, when subjected to certain conditions, allows them to be manipulated in the way of greatest interest, whether due to the need to conserve the material, either due to faster transport or the efficiency of another parameter in which it is intended spend less resources or have the greatest financial, energy or social return.

%    Na segregação de grãos, problemas relacionados a entupimentos podem ocorrer dependendo da geometria dos materiais \cite{Caio-Tese}. Quando estes materiais são submetidos à vibração, ou quando escoam, naturalmente separam-se os maiores dos menores, facilitando assim a filtração, porém dificultando a mistura. Estes conjuntos de aglomerados podem trazer consequências negativas aos processos industriais, como desgaste de silos relacionado à resistência dos materiais, corrosão do silo ou do granular, apodrecimento ou envelhecimento de alimentos estocados, etc. \cite{Silo_failures}.

    In grain segregation, problems related to clogging may occur depending on the geometry of the materials \cite{Caio-Tese}. When these materials are subjected to vibration, or when they flow, the largest are naturally separated of the smallest, thus facilitating the filtration, but making it difficult to mix. These sets of agglomerates can have negative consequences for industrial process, such as wear of silos related to the resistance of materials, corrosion of the silo or granular, rotting or aging of stored food, etc. \cite{Silo_failures}.

%    Assim, compreender como os materiais interagem e como são transportados, sejam transportados em uma esteira, sejam levados pelas correntezas de um rio, dá a possibilidade de controlar seus possíveis efeitos, ou prever suas consequências, como por exemplo a quantidade de resíduos remanescentes nos rios devido ao rompimento das barragens de Fundão em Mariana-MG em novembro de 2015 \cite{Mariana_en, Mariana_pt, Mariana_fr}, e a barragem de Brumadinho em janeiro de 2019 \cite{Brumadinho_en, Brumadinho_pt, Brumadinho_fr}.

    Thus, understanding how the materials interact and how they are transported, whether transported on a conveyor belt, or carried by the currents of a river, gives the possibility to control its possible effects, or to predict its consequences, such as the amount of residues remaining in in the rivers due to the collapse of the Fundão dams in Mariana-MG in November 2015 \cite{Mariana_en, Mariana_pt, Mariana_fr}, and the Brumadinho dam in January 2019 \cite{Brumadinho_en, Brumadinho_pt, Brumadinho_fr}.

\section{Motivation}
\label{sec:motivacao}

%    Por percebermos que ainda falta compreensão nos fenômenos envolvendo materiais granulares, propomos estudar nesta tese as técnicas que possam predizer o comportamento do aglomerado, ou caracterizar alguma propriedade emergente do sistema que ainda não tenha sido relatada ou documentada, bem como utilizar resultados já conhecidos dos materiais e procurar aplicá-los em outras áreas que a medição faz-se dificultada.
    
    Because we realize that there is still a lack of understanding in the phenomena involving granular materials, we propose to study in this thesis the techniques that can predict the behavior of the conglomerate, or to characterize some emerging properties of the system that has not yet been reported or documented, as well as to use results already known from the materials, and try to apply them in other areas where measurement is difficult.

%    Especificamente, trataremos de um assunto que ainda não havia sido reproduzido na literatura: o BNE em duas dimensões com condição periódica de contorno. Para analisar este problema utilizamos a técnica de \textit{Large Deviation Function} (LDF) \cite{Large_Deviations_in_Physics}.

    Specifically, we will deal with a subject that has not yet been reproduced in the literature: BNE in two dimensions with periodic boundary conditions. To analyze this problem we use the Large Deviation Function (LDF) \cite{Large_Deviation_in_Physics} technique.

%    Medimos também a medida de escala de tempo de saturação\footnote{Tempo de saturação em transporte de sedimentos indica o tempo característico que o material leva para entrar em regime permanente, saindo de uma configuração e chegando na configuração final.} do material quando transportado por um fluido.

    We also measure the saturation time scale\footnote{Saturation time in sediment transport indicates the characteristic time it takes for the material to enter a steady state, leaving a configuration and arriving at the final configuration.} of the material when transported by a fluid.

\section{Organização do trabalho}
\label{sec:organizacaoTrabalho}

    Este trabalho divide-se em um capítulo de revisão bibliográfica sobre materiais granulares, descrito no capítulo \ref{chap:Trabalhos-Relacionados}, com fenomenologia do estudo sobre o estudo de materiais granulares. As descrições do capítulo \ref{chap:DEM} dizem respeito ao equacionamento e da modelagem do sistema, subdividido em duas partes principais: a primeira diz respeito das forças de interação dos grãos e como são implementadas as equações regentes neste sistema, enquanto a segunda parte trata do fluido e como implementá-lo. O capítulo \ref{chap:BNE} refere-se em especial ao fenômeno conhecido como \textit{BNE}. o capítulo \ref{chap:Transporte-Sedimentos} traz as descrições do transporte de materiais por fluidos, com as caracterizações de cada modo de transporte. O capítulo \ref{chap:Conclusao} apresenta as conclusões parciais deste projeto de tese.

    %O capítulo \ref{chap:Metodologia} descreve a implantação dos modelos para conseguir observar e caracterizar, tanto o BNE quanto o transporte de grãos, através dos parâmetros do modelo já descritos no capítulo \ref{chap:DEM}.

    %Já o capítulo \ref{chap:Resultados}, mostra os resultados obtidos através das simulações realizadas, bem como uma discussão e validação do modelo proposto.

    %Finalmente, o capítulo \ref{chap:Conclusao} encerra este projeto de tese, concluindo os resultados obtidos com a literatura.
