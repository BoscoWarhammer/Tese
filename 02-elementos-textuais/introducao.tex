% -----------------------------------------------------------------------------
% Introdução
% -----------------------------------------------------------------------------

\chapter{Introduction}
\label{chap:Introducao}

%    Materiais granulares estão presentes em vários contextos da natureza e das atividades humanas \cite{Sands_Powders_and_Grains, The_Physics_of_Granular_Media, Granular_Physics, Micromechanics_of_Granular_Materials, Granular_Media_Between_Fluid_and_Solid}. Atividades econômicas, como produção agrícola, mineração e tecnologia de construção, são essencialmente ligadas ao uso de materiais granulares \cite{Sands_Powders_and_Grains}. Por muitos anos, os estudos em materiais granulares estiveram presentes principalmente nas engenharias \cite{Versuche_uber_Getreidedruck_in_Silozellen, Janssen}, com o intuito de otimizar os processos de produção, armazenagem, escoamento e aplicações estruturais destes materiais. Hoje, algumas áreas da física, como a mecânica estatística \cite{Unifying_Concepts_in_Granular_Media_and_Glasses}, estudam intensamente a caracterização do comportamento destes materiais e suas aplicações, pela riqueza dos fenômenos observados. Sua ubiquidade reflete a importância dos estudos acerca de seu conhecimento, para que haja a manipulação destes elementos nas mais diversas situações.

    Granular materials are present in various contexts of nature and in many human activities \cite{Sands_Powders_and_Grains, The_Physics_of_Granular_Media, Granular_Physics, Micromechanics_of_Granular_Materials, Granular_Media_Between_Fluid_and_Solid}. Economic activities, like agricultural production, mining and building technology, are essentially linked to the usage of granular materials \cite{Sands_Powders_and_Grains}. For many years, research in granular materials were linked manly to engineering \cite{Versuche_uber_Getreidedruck_in_Silozellen, Janssen}, in order to optimize production process, storing, flowing, and structural applications to these materials. Nowadays, some areas of physics, such as statistical mechanics \cite{Unifying_Concepts_in_Granular_Media_and_Glasses}, study intensively the characterization and behavior of these materials, and their applications, due to the richness of observed phenomena. Its ubiquity reflects the importance of studies about its knowledge, so that there is manipulation of these elements in the most diverse situations.

%    Materiais granulares podem ser caracterizados como um aglomerado de corpos maiores que algumas centenas de micrometros até o tamanho de asteroides \cite{Sands_Powders_and_Grains, The_Physics_of_Granular_Media}. Além do tamanho, outra característica dos corpos é se apresentarem no estado sólido. Suas interações resultam em dissipação de energia, seja por atrito, seja pela inelasticidade da interação. Não estão sujeitos à variações no movimento causadas por flutuações térmicas, e portanto, não exibem movimentos Brownianos. Mais caracterizações dos materiais granulares podem ser encontradas no capítulo \ref{chap:Trabalhos-Relacionados} desta tese.

    Granular materials can be characterized as a cluster of bodies larger than a few hundred micrometers up to the size of asteroids \cite{Sands_Powders_and_Grains, The_Physics_of_Granular_Media}. In addition to the size, another feature of bodies is that they are in solid state. Their interactions result in energy dissipation, either by friction or by inelastic of the interaction. They are not subject to various movement caused by thermal fluctuations, and therefore, do not exhibit Brownian movements. More characterizations of granular materials can be found in the chapter \ref{chap:Trabalhos-Relacionados} of this thesis.

    A proposta de estudos deste trabalho baseia-se na realização de simulações computacionais de materiais granulares, utilizando o Método de Elemento Discreto, ou \textit{Discrete Element Method} (DEM), baseado no método de Dinâmica Molecular, ou \textit{Molecular Dynamics (MD)} \cite{Computer_Simulation_of_Liquids}. A simulação utiliza a definição de que os grãos são representados por discos  que sofrem gravidade, e que possuem potencial de repulsão quando estão em contato. No contato também levamos em conta o atrito entre as partículas. Definidas as propriedades dos materiais, como dureza, atrito, massa, posição e raio, dispomos das equações de movimento de Newton e das equações da cinemática para realizar a simulação. Detalhamos tal equacionamento e peculiaridades da simulação no capítulo \ref{chap:DEM}. Em relação ao fluido, a descrição detalhada do equacionamento, considerações do fluido no problema de transporte e da Fluidodinâmica Computacional, ou \textit{Computational Fluid Dynamics (CFD)}, pode ser encontrada também no capítulo \ref{chap:DEM}.



    Dos fenômenos apresentados pelos materiais granulares, estudaremos nesse projeto de tese, o efeito castanha do Pará, ou \textit{Brazil Nut Effect (BNE)}, relacionado à segregação de grãos confinados quando são submetidos à vibração e em presença de um campo gravitacional. Grãos maiores segregam-se no topo, enquanto grãos menores afundam. O capítulo \ref{chap:BNE} fornece mais detalhes a respeito do \textit{BNE}, tanto no ponto de vista fenomenológico, quanto ao experimento proposto.

    Estudaremos também o fenômeno do transporte de sedimentos imersos em fluidos. Com as equações que regem os fluidos, a equação de Navier-Stokes \cite{Physical_Hydrodynamics, Fluid_Mechanics} é utilizada neste trabalho para modelar o fluido que escoa e carrega consigo parte do material granular. Existem alguns modos de transportes que são caracterizados pela maneira em que os grãos são trasportados pelo fluido e estão descritos no capítulo \ref{chap:Transporte-Sedimentos}.

\section{Justificativa}
\label{sec:justificativa}

    No contexto da engenharia, necessita-se compreender como os processos são elaborados, de forma a ajustá-los para otimizar os custos de produção, transporte e armazenamento de materiais vitais as atividades humanas, como alimentos e minérios. Neste sentido, o entendimento do comportamento dos materiais, quando submetidos à certas condições, permite manipulá-los da forma de maior interesse, seja por uma necessidade de conservação do material, seja pelo transporte mais rápido ou pela eficiência de outro parâmetro na qual pretende-se gastar menos recursos ou ter o maior retorno financeiro, energético ou social.

    Na segregação de grãos, problemas relacionados à entupimentos podem ocorrer dependendo da geometria dos materiais \cite{Caio-Tese}. Quando estes materiais são submetidos à vibração, ou quando escoam, naturalmente separam-se os maiores dos menores, facilitando assim a filtração, porém dificultando a mistura. Estes conjuntos de aglomerados podem trazer consequências aos processos industriais, como desgaste de silos relacionados à resistência dos materiais, corrosão do silo ou do granular, apodrecimento ou envelhecimento de alimentos estocados etc. \cite{Silo_failures}.

    Assim, compreender como os materiais interagem e como são transportados, seja de forma forçada em uma esteira, seja naturalmente pelas correntezas de um rio, dá a possibilidade de controlar seus possíveis efeitos, ou prever sua consequência, como por exemplo a quantidade de resíduos remanescentes nos rios devido ao rompimento das barragens de Fundão em Mariana-MG em novembro de 2015 \cite{Mariana} e a barragem de Barcarena-PA em fevereiro de 2018 \cite{Barcarena}.

\section{Motivação}
\label{sec:motivacao}

    Por percebermos que ainda falta compreensão nos fenômenos envolvendo granulares, propomos estudar técnicas que possam predizer o comportamento do aglomerado, ou caracterizar alguma propriedade emergente do sistema que ainda não tenha sido relatada ou documentada, bem como utilizar resultados já conhecidos dos materiais e procurar aplicá-los em outras áreas que a medição faz-se dificultada.
    
    Especificamente, trataremos de um assunto que ainda não havia sido reproduzido na literatura: o \textit{BNE} em duas dimensões com condição periódica de contorno.

    Tentaremos também realizar a medida de escala de tempo de saturação\textit{Tempo de saturação em transporte de sedimentos indica o tempo característico que o material leva para entrar em regime permanente, saindo de uma configuração e chegando na configuração final.} do material quando transportado por um fluido.

\section{Organização do trabalho}
\label{sec:organizacaoTrabalho}

    Este trabalho divide-se em um capítulo de revisão bibliográfica sobre materiais granulares, descrito no capítulo \ref{chap:Trabalhos-Relacionados}, com fenomenologia do estudo sobre o estudo de materiais granulares. As descrições do capítulo \ref{chap:DEM} dizem respeito ao equacionamento e da modelagem do sistema, subdividido em duas partes principais: a primeira diz respeito das forças de interação dos grãos e como são implementadas as equações regentes neste sistema, enquanto a segunda parte trata do fluido e como implementá-lo. O capítulo \ref{chap:BNE} refere-se em especial ao fenômeno conhecido como \textit{BNE}. o capítulo \ref{chap:Transporte-Sedimentos} traz as descrições do transporte de materiais por fluidos, com as caracterizações de cada modo de transporte. O capítulo \ref{chap:Conclusao} apresenta as conclusões parciais deste projeto de tese.

    %O capítulo \ref{chap:Metodologia} descreve a implantação dos modelos para conseguir observar e caracterizar, tanto o BNE quanto o transporte de grãos, através dos parâmetros do modelo já descritos no capítulo \ref{chap:DEM}.

    %Já o capítulo \ref{chap:Resultados}, mostra os resultados obtidos através das simulações realizadas, bem como uma discussão e validação do modelo proposto.

    %Finalmente, o capítulo \ref{chap:Conclusao} encerra este projeto de tese, concluindo os resultados obtidos com a literatura.
